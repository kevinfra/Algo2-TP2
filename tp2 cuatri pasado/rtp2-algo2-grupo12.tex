\documentclass[10pt, a4paper, spanish]{article}
\usepackage[paper=a4paper, left=1.5cm, right=1.5cm, bottom=1.5cm, top=3.5cm]{geometry}
\usepackage[spanish]{babel}
\selectlanguage{spanish}
\usepackage[utf8]{inputenc}
\usepackage[T1]{fontenc}
\usepackage{indentfirst}
\usepackage{fancyhdr}
\usepackage{latexsym}
\usepackage{lastpage}
\usepackage{aed2-symb,aed2-itef,aed2-tad,caratula}
\usepackage[colorlinks=true, linkcolor=blue]{hyperref}
\usepackage{calc}
\usepackage{ifthen}

\usepackage{xspace}
\usepackage{xargs}
\usepackage{algorithm}% http://ctan.org/pkg/algorithms
\usepackage{algpseudocode}% http://ctan.org/pkg/algorithmicx
\usepackage{verbatim}
\usepackage{listings}

% Estilo para Algoritmos
\lstdefinestyle{alg}{tabsize=4, frame=single, escapeinside=\'\', framesep=10pt}

\newcommand{\alg}[3]{\hangindent=\parindent#1 (#2) \ifx#3\empty\else$\rightarrow$ res: #3\fi}
\newcommand\ote[1]{\hspace*{\fill}~\mbox{O(#1)}\penalty -9999 }
\newcommand\ofi[1]{\ensuremath{\textbf{Complejidad}: #1}}

% Afanado
\newcommand{\moduloNombre}[1]{\textbf{#1}}

\let\NombreFuncion=\textsc
\let\TipoVariable=\texttt
\let\ModificadorArgumento=\textbf
\newcommand{\res}{$res$\xspace}
\newcommand{\tab}{\hspace*{7mm}}

\newcommandx{\TipoFuncion}[3]{%
  \NombreFuncion{#1}(#2) \ifx#3\empty\else $\to$ \res\,: \TipoVariable{#3}\fi%
}
\newcommand{\In}[2]{\ModificadorArgumento{in} \ensuremath{#1}\,: \TipoVariable{#2}\xspace}
\newcommand{\Out}[2]{\ModificadorArgumento{out} \ensuremath{#1}\,: \TipoVariable{#2}\xspace}
\newcommand{\Inout}[2]{\ModificadorArgumento{in/out} \ensuremath{#1}\,: \TipoVariable{#2}\xspace}
\newcommand{\Aplicar}[2]{\NombreFuncion{#1}(#2)}

\newlength{\IntFuncionLengthA}
\newlength{\IntFuncionLengthB}
\newlength{\IntFuncionLengthC}
%InterfazFuncion(nombre, argumentos, valor retorno, precondicion, postcondicion, complejidad, descripcion, aliasing)
\newcommandx{\InterfazFuncion}[9][4=true,6,7,8,9]{%
  \hangindent=\parindent
  \TipoFuncion{#1}{#2}{#3}\\%
  \textbf{Pre} $\equiv$ \{#4\}\\%
  \textbf{Post} $\equiv$ \{#5\}%
  \ifx#6\empty\else\\\textbf{Complejidad:} #6\fi%
  \ifx#7\empty\else\\\textbf{Descripción:} #7\fi%
  \ifx#8\empty\else\\\textbf{Aliasing:} #8\fi%
  \ifx#9\empty\else\\\textbf{Requiere:} #9\fi%
}

\newenvironment{Interfaz}{%
  \parskip=2ex%
  \noindent\textbf{\Large Interfaz}%
  \par%
}{}

\newenvironment{Representacion}{%
  \vspace*{2ex}%
  \noindent\textbf{\Large Representación}%
  \vspace*{2ex}%
}{}

\newenvironment{Algoritmos}{%
  \vspace*{2ex}%
  \noindent\textbf{\Large Algoritmos}%
  \vspace*{2ex}%
}{}


\newcommand{\Titulo}[1]{
  \vspace*{1ex}\par\noindent\textbf{\large #1}\par
}

\newenvironmentx{Estructura}[2][2={estr}]{%
  \par\vspace*{2ex}%
  \TipoVariable{#1} \textbf{se representa con} \TipoVariable{#2}%
  \par\vspace*{1ex}%
}{%
  \par\vspace*{2ex}%
}%

\newboolean{EstructuraHayItems}
\newlength{\lenTupla}
\newenvironmentx{Tupla}[1][1={estr}]{%
    \settowidth{\lenTupla}{\hspace*{3mm}donde \TipoVariable{#1} es \TipoVariable{tupla}$($}%
    \addtolength{\lenTupla}{\parindent}%
    \hspace*{3mm}donde \TipoVariable{#1} es \TipoVariable{tupla}$($%
    \begin{minipage}[t]{\linewidth-\lenTupla}%
    \setboolean{EstructuraHayItems}{false}%
}{%
    $)$%
    \end{minipage}
}

\newcommandx{\tupItem}[3][1={\ }]{%
    %\hspace*{3mm}%
    \ifthenelse{\boolean{EstructuraHayItems}}{%
        ,#1%
    }{}%
    \emph{#2}: \TipoVariable{#3}%
    \setboolean{EstructuraHayItems}{true}%
}

\newcommandx{\RepFc}[3][1={estr},2={e}]{%
  \tadOperacion{Rep}{#1}{bool}{}%
  \tadAxioma{Rep($#2$)}{#3}%
}%

\newcommandx{\Rep}[3][1={estr},2={e}]{%
  \tadOperacion{Rep}{#1}{bool}{}%
  \tadAxioma{Rep($#2$)}{true \ssi #3}%
}%

\newcommandx{\Abs}[5][1={estr},3={e}]{%
  \tadOperacion{Abs}{#1/#3}{#2}{Rep($#3$)}%
  \settominwidth{\hangindent}{Abs($#3$) \igobs #4: #2 $\mid$ }%
  \addtolength{\hangindent}{\parindent}%
  Abs($#3$) \igobs #4: #2 $\mid$ #5%
}%

\newcommandx{\AbsFc}[4][1={estr},3={e}]{%
  \tadOperacion{Abs}{#1/#3}{#2}{Rep($#3$)}%
  \tadAxioma{Abs($#3$)}{#4}%
}%


%FIN Afanado

\newcommand{\f}[1]{\text{#1}}
\renewcommand{\paratodo}[2]{$\forall~#2$: #1}

\sloppy

\hypersetup{%
 % Para que el PDF se abra a página completa.
 pdfstartview= {FitH \hypercalcbp{\paperheight-\topmargin-1in-\headheight}},
 pdfauthor={Grupo 12 - 1c2015 - Algoritmos y Estructuras de Datos II - DC - UBA},
 pdfsubject={TP2}
}

\parskip=5pt % 5pt es el tamaño de fuente

% Pongo en 0 la distancia extra entre ítemes.
\let\olditemize\itemize
\def\itemize{\olditemize\itemsep=0pt}

% Acomodo fancyhdr.
\pagestyle{fancy}
\thispagestyle{fancy}
\addtolength{\headheight}{1pt}
\lhead{Algoritmos y Estructuras de Datos II}
\rhead{$1^{\mathrm{er}}$ cuatrimestre de 2015}
\cfoot{\thepage /\pageref{LastPage}}
\renewcommand{\footrulewidth}{0.4pt}

% Encabezado
\lhead{Algoritmos y Estructuras de Datos II}
\rhead{Grupo 12}
% Pie de pagina
\renewcommand{\footrulewidth}{0.4pt}
\lfoot{Facultad de Ciencias Exactas y Naturales}
\rfoot{Universidad de Buenos Aires}

\begin{document}

% Datos de caratula
\materia{Algoritmos y Estructuras de Datos II}
\titulo{Trabajo Práctico II}
\subtitulo{Reentrega}
\grupo{Grupo: 12}

\integrante{Pondal, Iván}{078/14}{ivan.pondal@gmail.com}
\integrante{Paz, Maximiliano Leon}{251/14}{m4xileon@gmail.com}
\integrante{Mena, Manuel}{313/14}{manuelmena1993@gmail.com}
\integrante{Demartino, Francisco}{348/14}{demartino.francisco@gmail.com}

\maketitle
\pagebreak

%Indice
\tableofcontents

\pagebreak
\section{Módulo DCNet}

\subsection{Interfaz}

\textbf{se explica con}: \tadNombre{DCNet}.

\textbf{géneros}: \TipoVariable{dcnet}.

\subsubsection{Operaciones básicas de DCNet}

\InterfazFuncion{IniciarDCNet}{\In{r}{red}}{dcnet}
[true]
{$res$ $\igobs$ iniciarDCNet($red$)}
[$O(n * (n + L))$ donde n es es la cantidad de computadoras y L es la longitud de nombre de computadora mas larga]
[crea una DCNet nueva tomando una red]
[]

~

\InterfazFuncion{CrearPaquete}{\Inout{dcn}{dcnet}, \In{p}{paquete}}{}
[\\ \- $\qquad$ $dcn$ $\igobs$ $dcn_0$ $\land$
\\ \- $\qquad$$\neg$( ($\exists$ $p'$:paquete)(	paqueteEnTransito($dcn$, $p'$) $\land$ id($p$) = id($p'$) $\land$ origen($p$) $\in$ computadoras(red($dcn$)) $\yluego$
\\ \- $\qquad$ \- $\qquad$ destino($p$) $\in$ computadoras(red($dcn$)) $\yluego$ hayCamino?(red($dcn$), origen($p$), destino($p$)))) \\]
{$dcn$ $\igobs$ crearPaquete($dcn_0$)}
[$O(L + log(k))$ donde L es la longitud de nombre de computadora mas larga y k es la longitud de la cola de paquetes mas larga]
[crea un nuevo paquete]
[]

~

\InterfazFuncion{AvanzarSegundo}{\Inout{dcn}{dcnet}}{}
[$dcn$ $\igobs$ $dcn_0$]
{$dcn$ $\igobs$ avanzarSegundo($dcn_0$)}
[$O(n * (L + log(k)))$ donde n es es la cantidad de computadoras, L es la longitud de nombre de computadora mas larga y k es la longitud de la cola de paquetes mas larga]
[envia los paquetes con mayor prioridad a la siguiente compu]
[]

~

\InterfazFuncion{Red}{\In{dcn}{dcnet}}{red}
[true]
{alias($res$ $\igobs$ red($dcn$))}
[$O(1)$]
[devuelve la red de una DCNet]
[res es una referencia no modificable]

~

\InterfazFuncion{CaminoRecorrido}{\In{dcn}{dcnet}, \In{p}{paquete}}{secu(compu)}
[paqueteEnTransito?($dcn$, $p$)]
{alias($res$ $\igobs$ caminoRecorrido($dcn$, $p$))}
[$O(n * log(k))$ donde n es es la cantidad de computadoras y k es la longitud de la cola de paquetes mas larga]
[devuelve el camino recorrido por un paquete]
[res es una referencia no modificable]

\pagebreak

\InterfazFuncion{CantidadEnviados}{\In{dcn}{dcnet}, \In{c}{compu}}{nat}
[c $\in$ computadoras(red($dcn$))]
{$res$ $\igobs$ cantidadEnviados($dcn$, $c$)}
[$O(L)$ donde L es la longitud de nombre de computadora mas larga]
[devuelve la cantidad de paquetes enviados por una compu]
[]

~

\InterfazFuncion{EnEspera}{\In{dcn}{dcnet}, \In{c}{compu}}{conj(paquete)}
[c $\in$ computadoras(red($dcn$))]
{alias($res$ $\igobs$ enEspera($dcn$, $c$))}
[$O(L)$ donde L es la longitud de nombre de computadora mas larga]
[devuelve el conjunto de paquetes encolados en una compu]
[res es una referencia no modificable]

~

\InterfazFuncion{PaqueteEnTransito}{\In{dcn}{dcnet}, \In{p}{paquete}}{bool}
[true]
{$res$ $\igobs$ paqueteEnTransito($dcn$, $p$)}
[$O(n * log(k))$ donde n es es la cantidad de computadoras y k es la longitud de la cola de paquetes mas larga]
[indica si el paquete está en transito]
[]

~

\InterfazFuncion{LaQueMasEnvio}{\In{dcn}{dcnet}}{compu}
[true]
{alias($res$ $\igobs$ laQueMasEnvio($dcn$))}
[$O(1)$]
[devuelve la compu que mas paquetes envió]
[res es una referencia no modificable]

~

\InterfazFuncion{$\puntito$ = $\puntito$}{\In{dcn_1}{dcnet}, \In{dcn_2}{dcnet}}{bool}
[true]
{$res$ $\igobs$ $dcn_1$ = $dcn_2$}
[$O(n * k^3 * (k + n))$]
[compara $dcn_1$ y $dcn_2$ por igualdad]
[]

\pagebreak

\subsection{Representación}

\subsubsection{Representación de dcnet}

\begin{Estructura}{dcnet}[estr]
	\begin{Tupla}[estr]
		\tupItem{topología}{red}%
		\tupItem{\\ vectorCompusDCNet}{vector(compuDCNet)}%
		\tupItem{\\ diccCompusDCNet}{diccString(puntero(compuDCNet))}%
		\tupItem{\\ conjPaquetesDCNet}{conj(paqueteDCNet)}%
		\tupItem{\\ laQueMásEnvió}{puntero(compuDCNet)}%
	\end{Tupla}

	~

	\begin{Tupla}[compuDCNet]
		\tupItem{pc}{puntero(compu)}%
		\tupItem{\\ conjPaquetes}{conj(paquete)}%
		\tupItem{\\ diccPaquetesDCNet}{diccLog(itConj(paqueteDCNet))}%
		\tupItem{\\ colaPaquetesDCNet}{colaPrioridad(nat, itConj(paqueteDCNet))}%
		\tupItem{\\ paqueteAEnviar}{itConj(paqueteDCNet)}%
		\tupItem{enviados}{nat}%
	\end{Tupla}

	~

	\begin{Tupla}[paqueteDCNet]
		\tupItem{it}{itConj(paquete)}%
		\tupItem{recorrido}{lista(compu)}%
	\end{Tupla}

	~

	\begin{Tupla}[paquete]
		\tupItem{id}{nat}%
		\tupItem{prioridad}{nat}%
		\tupItem{origen}{compu}%
		\tupItem{destino}{compu}%
	\end{Tupla}

	~

	\begin{Tupla}[compu]
		\tupItem{ip}{string}%
		\tupItem{interfaces}{conj(nat)}%
	\end{Tupla}

\end{Estructura}

\subsubsection{Invariante de Representación}

\renewcommand{\labelenumi}{(\Roman{enumi})}

\begin{enumerate}
	\item Las compus de los elementos de vectorCompusDCNet son punteros a todas las compus de
		la topología
	\item Las claves de diccCompusDCNet son todos los hostnames de la topología
	\item Los significados de diccCompusDCNet son punteros que apuntan a las
		compuDCNet cuyo hostname equivale a su clave en vectorCompusDCNet
	\item laQueMásEnvió es un puntero a la compuDCNet en vectorCompusDCNet que
		más paquetes enviados tiene. Si no hay compus es NULL
	\item El conjPaquetesDCNet contiene tuplas con iteradores a todos los
		paquetes en tránsito en la red y sus recorridos
	\item Todos los paquetes en conjPaquetes de cada compuDCNet tienen id único
		y tanto su origen como destino existen en la topología
	\item El paquete en conjPaquetes tiene que tener en su recorrido a la
		compuDCNet en la que se encuentra y esta no puede ser igual al
		destino del recorrido
	\item Las claves de diccPaquetesDCNet son los id de los paquetes en
		conjPaquetes
	\item Los significados de diccPaquetesDCNet son un iterador al
		paqueteDCNet de conjPaquetesDCNet que contiene un iterador al
		paquete con el id equivalente a su clave y un recorrido que es uno de
		los caminos mínimos del origen del paquete a la compu en la que se
		encuentra
	\item La cantidad de enviados de una compuDCNet es igual o mayor a la
		cantidad de apariciones de esa compu en los caminos recorridos de
		paquetes en la red
	\item El paquete a enviar de cada compuDCNet es un iterador que no tiene siguiente
\end{enumerate}

\pagebreak

\Rep[estr][e]{
	\\
	($\forall c$: compu)($c$ $\in$ computadoras($e$.topologia) $\Leftrightarrow$ \\
	\- ( \\
	\- \- ($\exists cd$: compuDCNet)
	(está?($cd$, $e$.vectorCompusDCNet) $\land$ ($cd$.pc = puntero($c$)) $\land$ \\
	\- \- ($\exists s$: string)(def?($s$, $e$.diccCompusDCNet) $\land$ ($s$ = $c$.ip))) \\
	\- ) \\
	) $\yluego$\\
	($\forall cd$: compuDCNet)(está?($cd$, $e$.vectorCompusDCNet) $\Leftrightarrow$ \\
	\- ($\exists s$: string)(($s$ = $cd$.pc$\rightarrow$ip) $\land$
		def?($s$, $e$.diccCompusDCNet) $\yluego$ \\
	\- obtener($s$, $e$.diccCompusDCNet) = puntero($cd$)) \\
	) $\yluego$\\
	($\exists cd$: compuDCNet)(está?($cd$, $e$.vectorCompusDCNet) $\yluego$ \\
	*($cd$.pc) = compuQueMásEnvió($e$.vectorCompusDCNet) $\land$ $e$.laQueMásEnvió = puntero($cd$)) $\yluego$\\
	($\forall cd_1$: compuDCNet)(está?($cd_1$, $e$.vectorCompusDCNet) $\implies$ \\
	\- ($\forall p_1$: paquete)($p_1$ $\in$ $cd_1$.conjPaquetes $\implies$ \\
	\- \- ($\forall cd_2$: compuDCNet)((está?($cd_2$, $e$.vectorCompusDCNet)
		$\land$ $cd_1$ $\neq$ $cd_2$) $\implies$ \\
	\- \- \- ($\forall p_2$: paquete)($p_2$ $\in$ $cd_2$.conjPaquetes $\implies$
		$p_1$.id $\neq$ $p_2$.id) \\
	\- \- ) \\
	\- ) \\
	) $\yluego$ \\
	($\forall cd$: compuDCNet)(está?($cd$, $e$.vectorCompusDCNet) $\implies$ \\
	\- ( \\
	\- \- ($\forall p$: paquete)($p$ $\in$ $cd$.conjPaquetes $\Leftrightarrow$ \\
	\- \- \- ( \\
	\- \- \- \- (($p$.origen $\in$ computadoras($e$.topologia) $\land$ $p$.destino
		$\in$ computadoras($e$.topologia) $\land$ \\
	\- \- \- \- $p$.destino $\neq$ *($cd$.pc)) $\yluego$ \\
	\- \- \- \- ($\exists sc$: secu(compu))($sc$ $\in$
		caminosMinimos($e$.topologia, $p$.origen, $p$.destino) $\land$
		está(*($cd$.pc), $sc$))) $\land$ \\
	\- \- \- \- ($\exists n$: nat)
		((def?($n$, $cd$.diccPaquetesDCNet) $\land$ $p$.id = $n$) $\yluego$ \\
	\- \- \- \- \- ($\exists pdn$: paqueteDCNet)($pdn$ $\in$ $e$.conjPaquetesDCNet $\land$ Siguiente($pdn$.it) = $p$ $\land$ \\
	\- \- \- \- \- \- (($p$.origen = *($cd$.pc) $\land$ $pdn$.recorrido = *($cd$.pc) $\puntito$ <>) $\lor$ \\
	\- \- \- \- \- \- ($p$.origen $\neq$ *($cd$.pc) $\land$
		$pdn$.recorrido $\in$ caminosMinimos($e$.topologia, $p$.origen,
		*($cd$.pc)))) $\land$ \\
	\- \- \- \- \- \- Siguiente(obtener($n$, $cd$.diccPaquetesDCNet)) = $pdn$ \\
	\- \- \- \- \- ) \\
	\- \- \- \- ) \\
	\- \- \- ) \\
	\- \- ) $\yluego$ \\
	\- \- ($\neg$vacía?($cd$.colaPaquetesDCNet) $\Leftrightarrow$ \\
	\- \- \- ($\exists p$: paquete)(($p$ $\in$ $cd$.conjPaquetes) $\land$
		($p$ = paqueteMásPrioridad($cd$.conjPaquetes)) $\land$ \\
	\- \- \- \- ($\exists pdn$: paqueteDCNet)(($pdn$ $\in$
		$e$.conjPaquetesDCNet) $\land$ (Siguiente($pdn$.it) = $p$) $\land$ \\
	\- \- \- \-	(Siguiente(proximo($cd$.colaPaquetesDCNet)) = $pdn$)) \\
	\- \- \- ) \\
	\- \- ) $\yluego$ \\
	\- \- ($cd$.enviados $\geq$ enviadosCompu(*($cd$.pc), $e$.vectorCompusDCNet)) $\land$\\
	\- \- ($\neg$HaySiguiente?(cd.paqueteAEnviar))
	\- ) \\
	)
}\mbox{}

\tadOperacion{compuQueMásEnvió}{secu(compuDCNet)/scd}{compu}{$\neg$vacía?($scd$)}
\tadOperacion{maxEnviado}{secu(compuDCNet)/scd}{nat}{$\neg$vacía?($scd$)}
\tadOperacion{enviaronK}{secu(compuDCNet),nat}{conj(compu)}{}
\tadOperacion{paqueteMásPrioridad}{conj(paquete)/cp}{paquete}{$\neg \emptyset?(cp)$}
\tadOperacion{paquetesConPrioridadK}{conj(paquete),nat}{conj(paquete)}{}
\tadOperacion{altaPrioridad}{conj(paquetes)/cp}{nat}{$\neg \emptyset?(cp)$}
\tadOperacion{enviadosCompu}{compu,secu(compuDCNet)}{nat}{}
\tadOperacion{aparicionesCompu}{compu,conj(nat)/cn,dicc(nat,itConj(paqueteDCNet))/dp}{nat}{claves($dp$) $\subseteq$ $cn$}

~

\tadAxioma{compuQueMásEnvió($scd$)}{dameUno(enviaronK($scd$, maxEnviado($scd$)))}
\tadAxioma{maxEnviado($scd$)}{
	\IF vacía?(fin($scd$)) THEN
		prim($scd$).enviados
	ELSE
		max(prim($scd$), maxEnviado(fin($scd$)))
	FI
}
\tadAxioma{enviaronK($scd$, $k$)}{
	\IF vacía?($scd$) THEN
		$\emptyset$
	ELSE {
		\IF prim($scd$).enviados = $k$ THEN
			Ag(*(prim($scd$).pc), enviaronK(fin($scd$), $k$))
		ELSE
			enviaronK(fin($scd$), $k$)
		FI
		}
	FI
}
\tadAxioma{paqueteMásPrioridad($dcn$, $cp$)}{dameUno(paquetesConPrioridadK($cp$, altaPrioridad($cp$)))}

\tadAxioma{altaPrioridad($cp$)}{
	\IF $\emptyset$?(sinUno($cp$)) THEN
		dameUno($cp$).prioridad
	ELSE
		min(dameUno($cp$).prioridad, altaPrioridad(sinUno($cp$)))
	FI
}

\tadAxioma{paquetesConPrioridadK($cp$, $k$)}{
	\IF $\emptyset$?($cp$) THEN
		$\emptyset$
	ELSE {
		\IF dameUno($cp$).prioridad = $k$ THEN
			 Ag(dameUno($cp$), paquetesConPrioridadK(sinUno($cp$), $k$))
		ELSE
			paquetesConPrioridadK(sinUno($cp$), $k$)
		FI
		}
	FI
}

\tadAxioma{enviadosCompu($c$, $scd$)}{
	\IF vacía?($scd$) THEN
		0
	ELSE {
			\IF prim($scd$) = $c$ THEN
				enviadosCompu($c$, fin($scd$))
			ELSE {
				aparicionesCompu($c$, claves(prim($scd$).diccPaquetesDCNet), \\
				prim($scd$).diccPaquetesDCNet) + enviadosCompu($c$, fin($scd$))
				}
			FI
		}
	FI
}

\tadAxioma{aparicionesCompu($c$, $cn$, $dpd$)}{
	\IF	$\emptyset$?($cn$) THEN
		0
	ELSE {
			\IF está?($c$, Siguiente(obtener(dameUno($cn$), $dpd$)).recorrido) THEN
				1 + aparicionesCompu($c$, sinUno($cn$), $dpd$)
			ELSE {
				aparicionesCompu($c$, sinUno($cn$), $dpd$)
				}
			FI
		}
	FI
}

\pagebreak

\subsubsection{Funci\'on de Abstracci\'on}

\Abs[estr]{dcnet}[e]{dcn}{
red($dcn$) = $e$.topología $\land$ \\
($\forall cdn$: compuDCNet)(está?($cdn$, $e$.vectorCompusDCNet) $\impluego$ \\
\- enEspera($dcn$, *($cdn$.pc)) = $cdn$.conjPaquetes $\land$ \\
\- cantidadEnviados($dcn$, *($cdn$.pc)) = $cdn$.enviados $\land$ \\
\- ($\forall p$: paquete)($p$ $\in$ $cdn$.conjPaquetes $\impluego$ \\
\- \- caminoRecorrido($dcn$, $p$) = Siguiente(obtener($p$.id,
	$cdn$.diccPaquetesDCNet)).recorrido \\
\- ) \\
)
}

\subsection{Algoritmos}

\lstset{style=alg}

\begin{lstlisting}[mathescape]
'\alg{iIniciarDCNet}{\In{topo}{red}}{estr}'

    res.topologia $\leftarrow$ Copiar(topo) '\ote{n! * $n^6$}'
    res.vectorCompusDCNet $\leftarrow$ Vacia() '\ote{1}'
    res.diccCompusDCNet $\leftarrow$ CrearDicc() '\ote{1}'
    res.laQueMasEnvio $\leftarrow$ NULL '\ote{1}'
    res.conjPaquetesDCNet $\leftarrow$ Vacio() '\ote{1}'

    itConj(compu): it $\leftarrow$ CrearIt(Computadoras(topo)) '\ote{1}'

    if(HaySiguiente?(it)) then '\ote{1}'
    	res.laQueMasEnvio $\leftarrow$ puntero(Siguiente(it)) '\ote{1}'
    end if

    while HaySiguiente?(it) do '\ote{1}'
    	compuDCNet: compudcnet $\leftarrow$ <puntero(Siguiente(it)), Vacio(), CrearDicc(),
    		Vacia(), CrearIt(Vacio()), 0> '\ote{1}'
    	AgregarAtras(res.vectorCompusDCNet, compudcnet) '\ote{n}'
    	Definir(res.diccCompusDCNet, Siguiente(it).ip, puntero(compudcnet)) '\ote{L}'
    	Avanzar(it) '\ote{1}'
    end while '\ote{n * (n + L)}'

'\ofi{O(n * (n + L))}'
\end{lstlisting}

\begin{lstlisting}[mathescape]
'\alg{iCrearPaquete}{\Inout{dcn}{dcnet}, \In{p}{paquete}}{}'

	puntero(compuDCNet): compudcnet $\leftarrow$
		Significado(dcn.diccCompusDCNet, p.origen.ip) '\ote{L}'
	itConj(paquete): itPaq $\leftarrow$ AgregarRapido(compudcnet$\rightarrow$conjPaquetes, p) '\ote{1}'
	lista(compu): recorr $\leftarrow$ AgregarAtras(Vacia(), p.origen) '\ote{1}'
	paqueteDCNet: paqDCNet $\leftarrow$ <itPaq, recorr> '\ote{1}'

	itConj(paqueteDCNet): itPaqDCNet $\leftarrow$
		AgregarRapido(dcn.conjPaquetesDCNet, paqDCNet) '\ote{1}'
	Definir(compudcnet$\rightarrow$diccPaquetesDCNet, p.id, itPaqDCNet) '\ote{log(k)}'
	Encolar(compudcnet$\rightarrow$colaPaquetesDCNet, p.prioridad, itPaqDCNet) '\ote{log(k)}'

'\ofi{O(log(k) + L)}'
\end{lstlisting}

\begin{lstlisting}[mathescape]
'\alg{iAvanzarSegundo}{\Inout{dcn}{dcnet}}{}'

	nat: maxEnviados $\leftarrow$ 0
	nat: i $\leftarrow$ 0 '\ote{1}'
	while i < Longitud(dcn.vectorCompusDCNet) do '\ote{1}'
		if($\neg$EsVacia?(dcn.vectorCompusDCNet[i].colaPaquetesDCNet)) then
			dcn.vectorCompusDCNet[i].paqueteAEnviar $\leftarrow$
				Desencolar(dcn.vectorCompusDCNet[i].colaPaquetesDCNet) '\ote{log(k)}'
		end if
		i++ '\ote{1}'
	end while '\ote{n * log(k)}'

	i $\leftarrow$ 0 '\ote{1}'
	while i < Longitud(dcn.vectorCompusDCNet) do '\ote{1}'
		if(HaySiguiente?(dcn.vectorCompusDCNet[i].paqueteAEnviar)) then '\ote{1}'

			dcn.vectorCompusDCNet[i].enviados++ '\ote{1}'
			if(dcn.vectorCompusDCNet[i].enviados > maxEnviados) then '\ote{1}'
				dcn.laQueMasEnvio $\leftarrow$ puntero(dcn.vectorCompusDCNet[i]) '\ote{1}'
			end if

			paquete: pAEnviar $\leftarrow$
				Siguiente(Siguiente(dcn.vectorCompusDCNet[i].paqueteAEnviar).it) '\ote{1}'
			itConj(lista(compu)): itercaminos $\leftarrow$
				CrearIt(CaminosMinimos(dcn.topologia,
				*(dcn.vectorCompusDCNet[i].pc), pAEnviar.destino)) '\ote{1}'
			compu: siguientecompu $\leftarrow$ Siguiente(itercaminos)[1] '\ote{1}'

			if(pAEnviar.destino $\neq$ siguientecompu) then '\ote{1}'

				compuDCNet: siguientecompudcnet $\leftarrow$
					*(Obtener(dcn.diccCompusDCNet, siguientecompu.ip)) '\ote{L}'

				itConj(paquete): itpaquete $\leftarrow$
					AgregarRapido(siguientecompudcnet.conjPaquetes, pAEnviar) '\ote{1}'

				itConj(paqueteDCNet): paqAEnviar $\leftarrow$
					Obtener(dcn.vectorCompusDCNet[i].diccPaquetesDCNet,
					pAEnviar.id) '\ote{log(k)}'

				AgregarAtras(Siguiente(paqAEnviar).recorrido, siguientecompu) '\ote{1}'

				Encolar(siguientecompudcnet.colaPaquetesDCNet,
					pAEnviar.prioridad, paqAEnviar) '\ote{log(k)}'
				Definir(siguientecompudcnet.diccPaquetesDCNet,
					pAEnviar.id, paqAEnviar) '\ote{log(k)}'
			end if

			Borrar(dcn.vectorCompusDCNet[i].diccPaquetesDCNet,
				Siguiente(dcn.vectorCompusDCNet[i].paqueteAEnviar$\rightarrow$it).id) '\ote{log(k)}'
			EliminarSiguiente(Siguiente(dcn.vectorCompusDCNet[i].paqueteAEnviar).it)
				'\ote{1}'
			EliminarSiguiente(dcn.vectorCompusDCNet[i].paqueteAEnviar) '\ote{1}'

			dcn.vectorCompusDCNet[i].paqueteAEnviar $\leftarrow$ CrearIt(Vacio()) '\ote{1}'

		end if
		i++ '\ote{1}'
	end while '\ote{n * (L + log(k))}'

'\ofi{O(n * (L + log(k)))}'
\end{lstlisting}

\begin{lstlisting}[mathescape]
'\alg{Red}{\In{dcn}{dcnet}}{red}'

	res $\leftarrow$ dcn.topologia '\ote{1}'

'\ofi{O(1)}'
\end{lstlisting}

\begin{lstlisting}[mathescape]
'\alg{CaminoRecorrido}{\In{dcn}{dcnet}, \In{p}{paquete}}{lista(compu)}'

	nat: i $\leftarrow$ 0 '\ote{1}'
	while i < Longitud(dcn.vectorCompusDCNet) do '\ote{1}'
		if Definido?(dcn.vectorCompusDCNet[i].diccPaquetesDCNet, p.id) then '\ote{log(k)}'
			res $\leftarrow$ Siguiente(Obtener(dcn.vectorCompusDCNet[i].diccPaquetesDCNet,
				p.id)).recorrido '\ote{log(k)}'
		end if
		i++ '\ote{1}'
	end while '\ote{n * log(k)}'

'\ofi{O(n * log(k))}'
\end{lstlisting}

\begin{lstlisting}[mathescape]
'\alg{CantidadEnviados}{\In{dcn}{dcnet}, \In{c}{compu}}{nat}'

	res $\leftarrow$ Obtener(dcn.diccCompusDCNet, c.ip)$\rightarrow$enviados '\ote{L}'

'\ofi{O(L)}'
\end{lstlisting}

\begin{lstlisting}[mathescape]
'\alg{EnEspera}{\In{dcn}{dcnet}, \In{c}{compu}}{nat}'

	res $\leftarrow$ Obtener(dcn.diccCompusDCNet, c.ip)$\rightarrow$conjPaquetes '\ote{L}'

'\ofi{O(L)}'
\end{lstlisting}

\begin{lstlisting}[mathescape]
'\alg{PaqueteEnTransito}{\In{dcn}{dcnet}, \In{p}{paquete}}{bool}'

	res $\leftarrow$ false
	nat: i $\leftarrow$ 0 '\ote{1}'
	while i < Longitud(dcn.vectorCompusDCNet) do '\ote{1}'
		if Definido?(dcn.vectorCompusDCNet[i].diccPaquetesDCNet, p.id) then '\ote{log(k)}'
			res $\leftarrow$ true '\ote{1}'
		end if
		i++ '\ote{1}'
	end while '\ote{n * log(k)}'

'\ofi{O(n * log(k))}'
\end{lstlisting}

\begin{lstlisting}[mathescape]
'\alg{LaQueMasEnvio}{\In{dcn}{dcnet}}{compu}'

	res $\leftarrow$ *(dcn.laQueMasEnvio$\rightarrow$pc) '\ote{1}'

'\ofi{O(1)}'
\end{lstlisting}

\begin{lstlisting}[mathescape]
'\alg{$\puntito$ $=_i$ $\puntito$}{\In{dcn_1}{dcnet}, \In{dcn_2}{dcnet}}{bool}'

	bool: boolTopo $\leftarrow$ $dcn_1$.topologia = $dcn_2$.topologia '\ote{n + $L^2$}'
	bool: boolVec $\leftarrow$ $dcn_1$.vectorCompusDCNet = $dcn_2$.vectorCompusDCNet '\ote{n * k * (k + n)}'
	bool: boolConj $\leftarrow$ $dcn_1$.conjPaquetesDCNet = $dcn_2$.conjPaquetesDCNet '\ote{$k^3$ * (k + n)}'
	bool: boolMasEnvio $\leftarrow$ *($dcn_1$.laQueMasEnvio) = *($dcn_2$.laQueMasEnvio) '\ote{1}'

	res $\leftarrow$ boolTopo $\land$ boolVec $\land$ boolTrie $\land$ boolConj $\land$ boolMasEnvio '\ote{1}'

'\ofi{O(n * k^3 * (k + n))}'
\end{lstlisting}

\begin{lstlisting}[mathescape]
'\alg{$\puntito$ $=_{compudcn}$ $\puntito$}{\In{c_1}{compuDCNet}, \In{c_2}{compuDCNet}}{bool}'

	bool: boolPC $\leftarrow$ *($c_1$.pc) = *($c_2$.pc) '\ote{1}'
	bool: boolConj $\leftarrow$ $c_1$.conjPaquetes = $c_1$.conjPaquetes '\ote{$k^2$}'
	bool: boolAVL $\leftarrow$ true '\ote{1}'
	bool: boolCola $\leftarrow$ true '\ote{1}'
	bool: boolPaq $\leftarrow$ Siguiente($c_1$.paqueteAEnviar) $=_{paqdcn}$ Siguiente($c_2$.paqueteAEnviar)
		'\ote{n}'
	bool: boolEnviados $\leftarrow$ $c_1$.enviados = $c_2$.enviados '\ote{1}'

	if boolConj then '\ote{1}'
		itConj: $itconj_1$ $\leftarrow$ CrearIt($c_1$.conjPaquetes) '\ote{1}'
		while HaySiguiente?($itconj_1$) do '\ote{1}'
			if Definido?($c_2$.diccPaquetesDCNet, Siguiente($itconj_1$)).id then '\ote{log(n)}'
				if $\neg$(Siguiente(Obtener($c_1$.diccPaquetesDCNet, Siguiente($itconj_1$).id))
					$=_{paqdcn}$
					Siguiente(Obtener($c_1$.diccPaquetesDCNet, Siguiente($itconj_1$).id)))
					then '\ote{n}'
					boolAVL $\leftarrow$ false '\ote{1}'
				end if
			else
				boolAVL $\leftarrow$ false '\ote{1}'
			end if
			Avanzar($itconj_1$) '\ote{1}'
		end while '\ote{n * k}'
	end if

	if EsVacia($c_1$.colaPrioridad) then '\ote{1}'
		if $\neg$EsVacia($c_2$.colaPrioridad) then '\ote{1}'
			boolCola $\leftarrow$ false '\ote{1}'
		end if
	else
		if EsVacia($c_1$.colaPrioridad) then '\ote{1}'
			boolCola $\leftarrow$ false '\ote{1}'
		else
			if $\neg$(Siguiente(Proximo($c_1$.colaPrioridad)) $=_{paqdcn}$
				Siguiente(Proximo($c_2$.colaPrioridad))) then '\ote{n}'
				boolCola $\leftarrow$ false '\ote{1}'
			end if
		end if
	end if

	res $\leftarrow$ boolPC $\land$ boolConj $\land$ boolAVL $\land$ boolCola $\land$ boolPaq $\land$ boolEnviados '\ote{1}'

'\ofi{O(k^2 + n * k) = O(k * (k + n))}'
\end{lstlisting}

\begin{lstlisting}[mathescape]
'\alg{$\puntito$ $=_{paqdcn}$ $\puntito$}{\In{p_1}{paqueteDCNet}, \In{p_2}{paqueteDCNet},}{bool}'

	bool: boolPaq $\leftarrow$ Siguiente($p_1$.it) = Siguiente($p_2$.it) '\ote{1}'
	bool: boolRecorrido $\leftarrow$ $p_1$.recorrido = $p_2$.recorrido '\ote{n}'

	res $\leftarrow$ boolPaq $\land$ boolRecorrido '\ote{1}'

'\ofi{O(n)}'
\end{lstlisting}

\pagebreak
\section{Módulo Red}

\subsection{Interfaz}

\textbf{se explica con}: \tadNombre{red}.

\textbf{géneros}: \TipoVariable{red}.

  ~

  \InterfazFuncion{IniciarRed}{}{red}
  [true]
  {$res$ $\igobs$ iniciarRed}
  [$O(1)$]
  [Crea una red nueva]


  ~

  \InterfazFuncion{AgregarComputadora}{\Inout{r}{red}, \In{c}{compu}}{}
  [($r$ \igobs $r_0$) $\land$ (($\forall$ $c'$: compu) ($c'$ $\in$ computadoras($r$) $\Rightarrow$  ip($c$) $\neq$  ip($c'$)))  ]
  {$r$ $\igobs$ agregarComputadora($r_0$, $c$)) }
  [$O((n*L)$]
  [Agrega una computadora a la red]
  [La compu se agrega por copia]

  ~

  \InterfazFuncion{Conectar}{\Inout{r}{red}, \In{c}{compu}, \In{c'}{compu}, \In{i}{compu}, \In{i'}{compu}}{}
  [($r$ \igobs $r_0$) $\land$ ($c$ $\in$ computadoras($r$)) $\land$ ($c'$ $\in$ computadoras(r)) $\land$ (ip($c$) $\neq$ ip($c'$)) \\
   $\land$ ($\neg$conectadas?($r$, $c$, $c'$)) $\land$ ($\neg$usaInterfaz?($r$, $c$, $i$) $\land$ $\neg$usaInterfaz?($r$, $c'$, $i'$))]
  {$r$ $\igobs$ conectar($r_0$, $c$, $i$, $c'$, $i'$))}
  [$O(n!*(n^4))$]
  [Conecta dos computadoras y recalcula los caminos mínimos de la red.]

  ~


  \InterfazFuncion{Computadoras}{\In{r}{red}}{conj(compu)}
  [true]
  {alias($res$ \igobs computadoras($r$))}
  [$O(1)$]
  [Devuelve el conjunto de computadoras de la red.]
  [El conjunto se da por referencia, y es modificable si y solo si la red es modificable.]

  ~

  \InterfazFuncion{Conectadas?}{\In{r}{red}, \In{c}{compu}, \In{c'}{compu}}{bool}
  [($c$ $\in$ computadoras($r$)) $\land$ ($c'$ $\in$ computadoras($r$))]
  {$res$ \igobs conectadas?($r$, $c$, $c'$)}
  [$O(1)$]
  [Indica si dos computadoras de la red estan conectadas]

  ~

  \InterfazFuncion{InterfazUsada}{\In{r}{red}, \In{c}{compu},  \In{c'}{compu}}{interfaz}
  [conectadas?($r$, $c$, $c'$)]
  {$res$ \igobs interfazUsada($r$, $c$, $c'$)}
  [$O(L + n)$]
  [Devuelve la interfaz con la cual se conecta c con c']

  ~

  \InterfazFuncion{Vecinos}{\In{r}{red}, \In{c}{compu}}{conj(compu)}
  [$c$ $\in$ computadoras($r$)]
  {$res$ \igobs vecinos($r$, $c$)}
  [$O(n^2)$]
  [Devuelve el conjunto de computadoras conectadas con c]
  [Devuelve una copia de las computadoras conectadas a c]

  ~

  \InterfazFuncion{usaInterfaz?}{\In{r}{red}, \In{c}{compu}, \In{i}{interfaz}}{bool}
  [$c$ $\in$ computadoras($r$)]
  {$res$ \igobs usaInterfaz?($r$, $c$, $i$)}
  [$O(L + n)$]
  [Indica si la interfaz i es usada por la computadora c]

  ~

  \InterfazFuncion{CaminosMinimos}{\In{r}{red}, \In{c}{compu}, \In{c'}{compu}}{conj(lista(compu))}
  [($c$ $\in$ computadoras($r$)) $\land$ ($c'$ $\in$ computadoras(r))]
  {alias($res$ \igobs caminosMinimos($r$, $c$, $i$))}
  [$O(L)$]
  [Devuelve el conjunto de caminos minimos de c a c']
  [Devuelve una refencia no modificable]


  ~

  \InterfazFuncion{HayCamino?}{\In{r}{red}, \In{c}{compu}, \In{c'}{compu}}{bool}
  [($c$ $\in$ computadoras($r$)) $\land$ ($c'$ $\in$ computadoras(r))]
  {$res$ \igobs hayCamino?($r$, $c$, $i$)}
  [$O(L)$]
  [Indica si existe algún camino entre c y c']

  ~

  \InterfazFuncion{copiar}{\In{r}{red}}{red}
  [true]
  {$res$ \igobs $r$}
  [$O(n!*(n^6))$]
  [Devuelve una copia la red]

  ~

  \InterfazFuncion{• = •}{\In{r}{red}, \In{r'}{red}}{bool}
  [true]
  {$res$ \igobs ($r$ \igobs $r'$)}
  [$O(n +L^2)$]
  [Indica si r es igual a r']

  ~


\subsection{Representación}

  \subsubsection{Estructura}

    \begin{Estructura}{red}[estr]

      \begin{Tupla}[estr]
        \tupItem{compus}{conj(compu)}
        \tupItem{\\dns}{diccString(nodoRed)}
      \end{Tupla}

      ~

      \begin{Tupla}[nodoRed]
        \tupItem{pc}{puntero(compu)}
        \tupItem{\\caminos}{diccString(conj(lista(compu)))}
        \tupItem{\\conexiones}{diccLineal(nat, puntero(nodoRed))}
      \end{Tupla}

      ~

    	\begin{Tupla}[compu]
    		\tupItem{ip}{string}%
    		\tupItem{interfaces}{conj(nat)}%
    	\end{Tupla}

    \end{Estructura}
\pagebreak

\subsubsection{Invariante de Representación}
  \begin{enumerate}

  \item Todas los elementos de $compus$ deben tener IPs distintas.

  \item Para cada compu, el diccionario de strings $dns$ define para la clave $<$IP de esa compu$>$ un \TipoVariable{nodoRed} cuyo $pc$ es puntero a esa compu.

  \item \TipoVariable{nodoRed}.$conexiones$ contiene como claves todas las
        interfaces usadas de la compu $c$ (que tienen que estar en $pc$.interfaces)

  \item Ningun nodo se conecta con si mismo.

  \item Ningun nodo se conecta a otro a traves de dos interfaces distintas.

  \item Para cada \TipoVariable{nodoRed} en $dns$, $caminos$ tiene como claves todas las
        IPs de las compus de la red (\TipoVariable{estr}.$compus$), y los significados corresponden a todos los caminos
        mínimos desde la compu $pc$ hacia la compu cuya IP es clave.

  \end{enumerate}

  \Rep[estr][e]{ (\\
    \\
    (($\forall c1, c2$: compu) ($c1 \neq c2$ $\land$ $c1 \in e$.compus $\land$ $c2 \in e$.compus) $\implies$ $c1$.ip $\neq$ $c2$.ip) $\land$ \\

    (($\forall c$: compu)($c \in e$.compus $\implies$ \\
    \- \- ( def?($c$.ip, $e$.dns) $\yluego$ obtener($c$.ip, $e$.dns).pc = puntero($c$) ) \\
    )) $\land$ \\

    (($\forall i$: string, $n$: nodoRed) ((def?($i$, $e$.dns) $\yluego$ $n$ = obtener($i$, $e$.dns)) $\implies$ \\
    \- \- ($\exists c$: compu) ($c \in e$.compus $\land$ ($n$.pc = puntero($c$))) \\
    )) $\land$ \\

    (($\forall i$: string, $n$: nodoRed) ((def?($i$, $e$.dns) $\yluego$ $n$ = obtener($i$, $e$.dns)) $\implies$ \\
    \- \- (($\forall t$: nat) (def?($t$, $n$.conexiones) $\implies$ ($t$ $\in$ $n$.pc$\rightarrow$interfaces))) \\
    )) $\land$ \\

    (($\forall i$: string, $n$: nodoRed) ((def?($i$, $e$.dns) $\yluego$ $n$ = obtener($i$, $e$.dns)) $\implies$ \\
    \- \- (($\forall t$: nat) (def?($t$, $n$.conexiones) $\impluego$ (obtener($t$, $n$.conexiones) $\neq$ puntero($n$))) ) \\
    )) $\land$ \\

    (($\forall i$: string, $n$: nodoRed) ((def?($i$, $e$.dns) $\yluego$ $n$ = obtener($i$, $e$.dns)) $\implies$ \\
    \- \- (($\forall t1, t2$: nat) (($t1 \neq t2$ $\land$ def?($t1$, $n$.conexiones) $\land$ def?($t2$, $n$.conexiones)) $\impluego$ \\
    \- \- \- \- (obtener($t1$, $n$.conexiones) $\neq$ obtener($t2$, $n$.conexiones)) \\
    \- \- )) \\
    )) $\land$ \\

    (($\forall i1, i2$: string, $n1, n2$: nodoRed) (( \\
    \- \- (def?($i1$, $e$.dns) $\yluego$ $n1$ = obtener($i1$, $e$.dns)) $\land$ \\
    \- \- (def?($i2$, $e$.dns) $\yluego$ $n2$ = obtener($i2$, $e$.dns)) \\
    \- ) $\implies$ (def?($i2$, $n1$.caminos) $\yluego$ obtener($i2$, $n1$.caminos) = darCaminosMinimos($n1$, $n2$)) \\
    )) \\


    ) \\
  }
\mbox{}

\tadAlinearFunciones{darCaminoMasCorto}{topologia, conj(pc), pc/ip, conj(pc), secu(nodoRed)}

\tadOperacion{vecinas}{nodoRed}{conj(nodoRed)}{}
\tadOperacion{auxVecinas}{nodoRed, dicc(nat, puntero(nodoRed))}{conj(nodoRed)}{}
\tadOperacion{secusDeLongK}{conj(secu($\alpha$)), nat}{conj(secu($\alpha$))}{}
\tadOperacion{longMenorSec}{conj(secu($\alpha$))/secus}{nat}{$\neg \emptyset?(secus)$}

\tadOperacion{darRutas}{nodoRed/nA, nodoRed/nB, conj(pc), secu(nodoRed)}{conj(secu(nodoRed))}{}
\tadOperacion{darRutasVecinas}{conj(pc)/vec, nodoRed/n, conj(pc), secu(nodoRed)}{conj(secu(nodoRed))}{}
\tadOperacion{darCaminosMinimos}{nodoRed/n1, nodoRed/n1}{conj(secu(compu))}{}

~


\tadAlinearAxiomas{darRutasEEEEEEWACHOOOOOOOOO}
\tadAxioma{vecinas($n$)}{auxVecinas($n$, $n$.conexiones)}

\tadAxioma{auxVecinas($n$, $cs$)}{
	\IF $\emptyset$?($cs$) THEN
		$\emptyset$
	ELSE
    Ag(obtener(dameUno(claves(cs)), cs), auxVecinas($n$, sinUno($cs$)))
	FI
}


\tadAxioma{secusDeLongK($secus$, $k$)}{
	\IF $\emptyset$?($secus$) THEN
		$\emptyset$
	ELSE{
		\IF long(dameUno($secus$)) = $k$ THEN
			dameUno($secus$) $\cup$ secusDeLongK(sinUno($secus$), $k$)
		ELSE
			secusDeLongK(sinUno($secus$), $k$)
		FI
	}
	FI
}

\tadAxioma{longMenorSec($secus$)}{
	\IF $\emptyset$?(sinUno($secus$)) THEN
		long(dameUno($secus$))
	ELSE
		min(long(dameUno($secus$)), \\
		longMenorSec(sinUno($secus$)))
	FI
}

~


\tadAxioma{darRutas($nA$, $nB$, $rec$, $ruta$)}{
	\IF $nB$ $\in$ vecinas($nA$) THEN
		Ag($ruta$ \circulito $nB$, $\emptyset$)
	ELSE{
		\IF $\emptyset$?(vecinas($nA$) - $rec$) THEN
			$\emptyset$
		ELSE
			darRutas(dameUno(vecinas($nA$) - $rec$), \\ $nB$, Ag($nA$, $rec$),\\
							$ruta$ \circulito dameUno(vecinas($nA$) - $rec$)) $\cup$ \\
			darRutasVecinas(sinUno(vecinas($nA$) - $rec$), \\ $nB$, Ag($nA$, $rec$), \\
							$ruta$ \circulito dameUno(vecinas($nA$) - $rec$))
		FI
	}
	FI
}

\tadAxioma{darRutasVecinas($vecinas$, $n$, $rec$, $ruta$)}{
	\IF $\emptyset$?($vecinas$) THEN
		$\emptyset$
	ELSE
		darRutas(dameUno($vecinas$), $n$, $rec$, $ruta$) $\cup$ \\
		darRutasVecinas(sinUno($vecinas$), $n$, $rec$, $ruta$)
	FI
}

\tadAxioma{darCaminosMinimos($nA$, $nB$)}{
	secusDeLongK(darRutas($nA$, $nB$, $\emptyset$, <>), \\
	longMenorSec(darRutas($nA$, $nB$, $\emptyset$, <>)))
}

\pagebreak

  \subsubsection{Función de Abstracción}

  \tadAlinearFunciones{Abs}{Estr/e}
  \Abs[estr]{red}[e]{r}{
    $e$.compus $\igobs$ computadoras($r$) $\land$ \\
    (($\forall$ $c1$, $c2$: compu, $i1$, $i2$: string, $n1$, $n2$: nodoRed) ( \\
        \- \- ($c1$ $\in$ $e$.compus $\land$ $i1$ = $c1$.ip $\land$ def?($i1$, $e$.dns) $\yluego$ $n1$ = obtener($i1$, $e$.dns) $\land$ $c1$ = *$n1$.pc) $\land$ \\
        \- \- ($c2$ $\in$ $e$.compus $\land$ $i2$ = $c2$.ip $\land$ def?($i2$, $e$.dns) $\yluego$ $n2$ = obtener($i2$, $e$.dns) $\land$ $c2$ = *$n2$.pc) $\land$ \\
        \- \- ($c1$ $\neq$ $c2$))
        ) $\impluego$ \\
        \- \- \- \- (conectadas?($r$, $c1$, $c2$) $\Leftrightarrow$ ( ($\exists$ $t1$, $t2$: nat)  (\\
        \- \- \- \- \- \- $t1$ = interfazUsada($r$, $c1$, $c2$) $\land$ $t2$ = interfazUsada($r$, $c2$, $c1$) $\land$ \\
        \- \- \- \- \- \- def?($t1$, $n1$.conexiones) $\land$ def?($t2$, $n2$.conexiones) $\yluego$ (\\
        \- \- \- \- \- \- \- \- $\&n2$ = obtener($t1$, $n1$.conexiones) $\land$ $\&n1$ = obtener($t2$, $n2$.conexiones) \\
        \- \- \- \- ))))
   }

\subsection{Algoritmos}
\lstset{style=alg,columns=fixed,basewidth=.5em}

\begin{lstlisting}[mathescape]
'\alg{iIniciarRed}{}{red}'
    res.compus $\leftarrow$ Vacio() '\ote{1}'
    res.dns $\leftarrow$ Vacio() '\ote{1}'
  '\ofi{O(1)}'
\end{lstlisting}

\begin{lstlisting}[mathescape]
'\alg{iAgregarComputadora}{\Inout{r}{red}, \In{c}{compu}}{}'
    itCompus:itConj(compu) $\leftarrow$ CrearIt(r.compus)   '\ote{1}'
    while HaySiguiente?(itCompus) do '\ote{1}'
      nr:nodoRed $\leftarrow$ Significado(r.dns, Siguiente(itCompus).ip) '\ote{L}'
      Definir(nr.caminos, c.ip, Vacio()) '\ote{L}'
      Avanzar(itCompus) '\ote{1}'
    end while  '\ote{n*L}'

    AgregarRapido(r.compus, c) '\ote{1}'

    Definir(r.dns, compu.ip, Tupla<&c,Vacio(),Vacio()>) '\ote{L}'
    InicializarConjCaminos(r, c) '\ote{n*L}'
  '\ofi{O(n*L)}'
\end{lstlisting}

\begin{lstlisting}[mathescape]
'\alg{InicializarConjCaminos}{\Inout{r}{red}, \In{c}{compu}}{}'
    itCompus:itConj(compu) $\leftarrow$ CrearIt(r.compus) '\ote{1}'
    cams:diccTrie(ip,conj(lista(compu))) $\leftarrow$
     Significado(r.dns, c.ip).caminos   '\ote{L}'
    while HaySiguiente?(itCompus) do   '\ote{1}'
      Definir(cams, Siguiente(itCompus).ip, Vacio())  '\ote{L}'
      Avanzar(itCompus) '\ote{1}'
    end while '\ote{n*L}'
  '\ofi{O(n*L)}'
\end{lstlisting}

\pagebreak

\begin{lstlisting}[mathescape]
'\alg{iConectar}{\Inout{r}{red}, \In{c_0}{compu}, \In{c_1}{compu}, \In{i_0}{compu}, \In{i_1}{compu}}{}'
    nr0:nodoRed $\leftarrow$ Significado(r.dns, c0.ip) '\ote{L}'
    nr1:nodoRed $\leftarrow$ Significado(r.dns, c1.ip) '\ote{L}'
    DefinirRapido(nr0.conexiones, i0, nr1)'\ote{1}'
    DefinirRapido(nr1.conexiones, i1, nr0) '\ote{1}'
    CrearTodosLosCaminos(r) '\ote{n! * (n$^3$ *(n + L))}'
  '\ofi{O(n! * (n^3 *(n + L)))}'
\end{lstlisting}

\begin{lstlisting}[mathescape]
'\alg{CrearTodosLosCaminos}{\Inout{r}{red}}{}'
  itCompuA:itConj(compu) $\leftarrow$ CrearIt(r.compus) '\ote{1}'
  while HaySiguiente?(itCompuA) do  '\ote{1}'
    nr:nodoRed $\leftarrow$ Significado(r.dns, Siguiente(itCompuA).ip) '\ote{L}'

    itCompuB:itConj(compu) $\leftarrow$ CrearIt(r.compus) '\ote{1}'
    while HaySiguiente?(itCompuB) do  '\ote{1}'

      caminimos:conj(lista(compu)) $\leftarrow$ Minimos(Caminos
        (nr, Siguiente(itCompuB).ip) '\ote{ n! * n*(n + L)}'
      Definir(nr.caminos, Siguiente(itCompuB).ip, caminimos)  '\ote{L}'

      Avanzar(itCompuB) '\ote{1}'
    end while '\ote{n! * (n$^2$ *(n + L))}'

    Avanzar(itCompuA) '\ote{1}'
  end while '\ote{n! * (n$^3$ *(n + L))}'
'\ofi{O(n! * (n^3 *(n + L)))}'
\end{lstlisting}

\begin{lstlisting}[mathescape]
'\alg{Caminos}{\In{c1}{nodoRed}, \In{ipDestino}{string}}{conj(lista(compu))}'
  res $\leftarrow$ Vacio()       '\ote{1}'

  frameRecorrido:pila(lista(compu)) $\leftarrow$ Vacia() '\ote{1}'
  frameCandidatos:pila(lista(nodoRed)) $\leftarrow$ Vacia() '\ote{1}'

  iCandidatos:lista(nodoRed) $\leftarrow$ listaNodosVecinos(c1) '\ote{n}'
  iRecorrido:lista(compu) $\leftarrow$ Vacia()  '\ote{1}'
  AgregarAdelante(iRecorrido, *(c1.pc)) '\ote{1}'

  Apilar(frameRecorrido, iRecorrido) '\ote{1}'
  Apilar(frameCandidatos, iCandidatos) '\ote{1}'

  pCandidatos:compu '\ote{1}'
  fCandidatos:lista(nodoRed) '\ote{1}'

  while $\neg$EsVacia?(frameRecorrido) do  '\ote{1}'
    iRecorrido $\leftarrow$ Tope(frameRecorrido) '\ote{1}'
    iCandidatos $\leftarrow$ Tope(frameCandidatos)  '\ote{1}'

    Desapilar(frameRecorrido) '\ote{1}'
    Desapilar(frameCandidatos)  '\ote{1}'

    pCandidatos $\leftarrow$ Primero(iCandidatos) '\ote{1}'

    // ... sigue


    if $\neg$EsVacio?(iCandidatos) then   '\ote{1}'
      Fin(iCandidatos)  '\ote{1}'
      fCandidatos $\leftarrow$ iCandidatos  '\ote{n}'

      if ult(iRecorrido).pc$\rightarrow$ip = ipDestino then   '\ote{L}'
        AgregarRapido(res, iRecorrido)  '\ote{n}'
      else
        Apilar(frameRecorrido, iRecorrido)  '\ote{1}'
        Apilar(frameCandidatos, fCandidatos)  '\ote{1}'

        if $\neg$nodoEnLista(pCandidatos, iRecorrido) then '\ote{n*(n + L)}'
          iRecorrido $\leftarrow$ Copiar(iRecorrido)  '\ote{n}'
          AgregarAtras(iRecorrido, *(pCandidatos)) '\ote{n}'
          Apilar(frameRecorrido, iRecorrido)  '\ote{1}'
          Apilar(frameCandidatos, listaNodosVecinos(pCandidatos)) '\ote{n}'
        fi '\ote{n*(n + L)}'
      fi '\ote{n*(n + L)}'
    fi '\ote{n*(n + L)}'

  end while '\ote{n! * n*(n + L)}'
'\ofi{O(n! * n*(n + L))}'
\end{lstlisting}

\begin{lstlisting}[mathescape]
'\alg{Minimos}{\In{caminos}{conj(lista(compu))}}{conj(lista(compu))}'
  res $\leftarrow$ Vacio() '\ote{1}'
  longMinima:int '\ote{1}'
  itCaminos:itConj(lista(compu)) $\leftarrow$ CrearIt(caminos) '\ote{1}'
  if HaySiguiente?(itCaminos) then  '\ote{1}'
    longMinima $\leftarrow$ Longitud(Siguiente(itCaminos)) '\ote{1}'
    Avanzar(itCaminos)  '\ote{1}'
    while HaySiguiente?(itCaminos)  '\ote{1}'
      if Longitud(Siguiente(itCaminos)) < longMinima then
        longMinima $\leftarrow$ Longitud(Siguiente(itCaminos)) '\ote{1}'
      Avanzar(itCaminos)  '\ote{1}'
    end while '\ote{n}'
    itCaminos $\leftarrow$ CrearIt(caminos) '\ote{1}'
    while HaySiguiente?(itCaminos)  '\ote{1}'
      if Longitud(Siguiente(itCaminos)) = longMinima then '\ote{1}'
        AgregarRapido(res, Siguiente(itCaminos))  '\ote{1}'
      Avanzar(itCaminos)  '\ote{1}'
    end while '\ote{n}'
  end if '\ote{1}'
'\ofi{O(n)}'

\end{lstlisting}

\begin{lstlisting}[mathescape]
'\alg{listaNodosVecinos}{\In{n}{nodoRed}}{lista(nodoRed)}'
  res $\leftarrow$ Vacia()  '\ote{1}'
  itVecinos :itDicc(interfaz, puntero(nodoRed))) $\leftarrow$ CrearIt(n,conexiones) '\ote{1}'
  while HaySiguiente?(itVecinos) do '\ote{1}'
    AgregarAdelante(res, *SiguienteSignificado(itVecinos)) '\ote{1}'
    Avanzar(itVecinos) '\ote{1}'
  end while '\ote{n}'
'\ofi{O(n)}'
\end{lstlisting}

\pagebreak

\begin{lstlisting}[mathescape]
'\alg{nodoEnLista}{\In{n}{nodoRed}, \In{ns}{lista(nodoRed)}}{bool}'
  res $\leftarrow$ false  '\ote{1}'
  itNodos: itLista(lista(nodoRed)) $\leftarrow$ CrearIt(ns) '\ote{1}'
  while HaySiguiente?(itNodos) do '\ote{1}'
    if Siguiente(itNodos) = n then '\ote{n + L}'
      res $\leftarrow$ true   '\ote{1}'
    end if  '\ote{1}'
    Avanzar(itNodos) '\ote{1}'
  end while '\ote{n*(n + L)}'
'\ofi{O(n*(n + L))}'
\end{lstlisting}


\begin{lstlisting}[mathescape]
'\alg{iComputadoras}{\In{r}{red}}{conj(compu)}'
    res  $\leftarrow$ r.compus  '\ote{1}'
  '\ofi{O(1)}'
\end{lstlisting}

\begin{lstlisting}[mathescape]
'\alg{iConectadas?}{\In{r}{red}, \In{c_0}{compu}, \In{c_1}{compu}}{bool}'
    nr0:nodoRed $\leftarrow$ Significado(r.dns, c0.ip) '\ote{L}'
    it :itDicc(interfaz, puntero(nodoRed))) $\leftarrow$ CrearIt(nr0.conexiones) '\ote{1}'
    res $\leftarrow$ false  '\ote{1}'
    while HaySiguiente?(it) do  '\ote{1}'
      if c1.ip = SiguienteSignificado(it)->pc->ip then  '\ote{1}'
        res $\leftarrow$ true   '\ote{1}'
      end if '\ote{1}'
      Avanzar(it) '\ote{1}'
    end while '\ote{n}'
  '\ofi{O(L + n)}'
\end{lstlisting}

\begin{lstlisting}[mathescape]
'\alg{iInterfazUsada}{\In{r}{red}, \In{c_0}{compu},  \In{c_1}{compu}}{interfaz}'
    nr0:nodoRed $\leftarrow$ Significado(r.dns, c0.ip)  '\ote{L}'
    it :itDicc(interfaz, puntero(nodoRed))
      $\leftarrow$ CrearIt(nr0.conexiones)  '\ote{1}'
    while HaySiguiente?(it) do  '\ote{1}'
      if c1.ip = SiguienteSignificado(it)->pc->ip then '\ote{1}'
        res $\leftarrow$ SiguienteClave(it)   '\ote{1}'
      end if '\ote{1}'
      Avanzar(it)   '\ote{1}'
    end while '\ote{n}'
  '\ofi{O(L + n)}'
\end{lstlisting}

\begin{lstlisting}[mathescape]
'\alg{iVecinos}{\In{r}{red}, \In{c}{compu}}{conj(compu)}'
    nr:nodoRed $\leftarrow$ Significado(r.dns, c.ip)  '\ote{L}'
    res:conj(compu) $\leftarrow$ Vacio()  '\ote{1}'
    it :itDicc(interfaz, puntero(nodoRed))
      $\leftarrow$ CrearIt(nr.conexiones)  '\ote{1}'
    while HaySiguiente?(it) do  '\ote{1}'
      AgregarRapido(res,*(SiguienteSignificado(it)->pc))  '\ote{1}'
      Avanzar(it)  '\ote{1}'
    end while '\ote{n}'
  '\ofi{O(L + n)}'
\end{lstlisting}

\begin{lstlisting}[mathescape]
'\alg{iUsaInterfaz?}{\In{r}{red}, \In{c}{compu}, \In{i}{interfaz}}{bool}'
    nr:nodoRed $\leftarrow$ Significado(r.dns, c.ip)   '\ote{L}'
    res $\leftarrow$ Definido?(pnr.conexiones,i)  '\ote{n}'
  '\ofi{O(L + n)}'
\end{lstlisting}

\begin{lstlisting}[mathescape]
'\alg{iCaminosMinimos}{\In{r}{red}, \In{c_0}{compu}, \In{c_1}{compu}}{conj(secu(compu))}'
    nr:nodoRed $\leftarrow$ Significado(r.dns, c0.ip) '\ote{L}'
    res $\leftarrow$  Significado(pnr.caminos, c1.ip) '\ote{L}'
  '\ofi{O(L)}'
\end{lstlisting}

\begin{lstlisting}[mathescape]
'\alg{HayCamino?}{\In{r}{red}, \In{c_0}{compu}, \In{c_1}{compu}}{bool}'
    nr:nodoRed $\leftarrow$ Significado(r.dns, c0.ip)  '\ote{L}'
    res $\leftarrow$ $\neg$EsVacio?(Significado(pnr.caminos, c1.ip))  '\ote{L}'
  '\ofi{O(L)}'
\end{lstlisting}

\begin{lstlisting}[mathescape]
'\alg{iCopiar}{\In{otraRed}{red}}{red}'
    res $\leftarrow$ iIniciarRed '\ote{1}'
    // copia el conjunto de tuplas
    res.compus $\leftarrow$ Copiar(otraRed.compus) '\ote{n}'
    // rearma los nodos (con conexiones en blanco) del diccionario dns
    itCompus:itConj(compu) $\leftarrow$ CrearIt(res.compus)   '\ote{1}'
    while HaySiguiente?(itCompus) do '\ote{1}'

      c:compu $\leftarrow$ Siguiente(itCompus) '\ote{1}'
      nodoAux:nodoRed $\leftarrow$ Obtener(otraRed.dns, c.ip) '\ote{L}'
      copiaCaminos:diccString(conj(lista(compu))) $\leftarrow$ Copiar(nodoAux.caminos) '\ote{n}'
      Definir(res.dns, c.ip, Tupla<&c, copiaCaminos, Vacio()>) '\ote{L}'

      Avanzar(itCompus) '\ote{1}'
    end while '\ote{n$^2$ + n*L}'

    // rearma las conexiones
    itCompus:itConj(compu) $\leftarrow$ CrearIt(res.compus)   '\ote{1}'
    while HaySiguiente?(itCompus) do '\ote{1}'
      nodoMio:nodoRed $\leftarrow$ Obtener(res.dns, c.ip) '\ote{L}'
      nodoOtra:nodoRed $\leftarrow$ Obtener(otraRed.dns, c.ip) '\ote{L}'

      itInterfs:itConj(nat) $\leftarrow$ CrearIt(nodoMio.pc $\rightarrow$ interfaces)   '\ote{1}'

      while HaySiguiente?(itInterfs) do '\ote{1}'
        interf:nat $\leftarrow$ Siguiente(itInterfs) '\ote{1}'
        ip:string $\leftarrow$ Obtener(nodoOtra.conexiones, interf) '\ote{n}'
        Definir(nodoMio.conexiones, interf, &Obtener(res.dns, ip)) '\ote{L}'

        Avanzar(itInterfs) '\ote{1}'
      end while  '\ote{$n^2$ + n*L}'

      Avanzar(itCompus) '\ote{1}'
    end while '\ote{n$^3$ + n$^2$ * L}'

  '\ofi{O(n^3 + n^2 * L))}'
\end{lstlisting}

\begin{lstlisting}[mathescape]
'\alg{• = •}{\In{r_0}{red}, \In{r_1}{red}}{bool}'
    res $\leftarrow$ (r0.compus = r1.compus) $\land$ (r0.dns = r1.dns) '\ote{n + L$^2$}'
  '\ofi{O(n + L(L + n))}'
\end{lstlisting}

\pagebreak
\section{Módulo Cola de mínima prioridad($\alpha$)}

El módulo cola de mínima prioridad consiste en una cola de prioridad de
elementos del tipo $\alpha$ cuya prioridad está determinada por un $nat$ de forma
tal que el elemento que se ingrese con el menor $nat$ será el de mayor prioridad.

\subsection{Especificación}

	\begin{tad}{\tadNombre{Cola de mínima prioridad($\alpha$)}}
	\tadIgualdadObservacional{c}{c'}{colaMinPrior($\alpha$)}{vacía?($c$) $\igobs$ vacía?($c'$) $\yluego$ \\
							($\neg$vacía?($c$) $\impluego$ (próximo($c$) $\igobs$ próximo($c'$) $\land$ \\
							\phantom{($\neg$vacía?($c$) $\impluego$ (}desencolar($c$) $\igobs$ desencolar($c'$))}

	\tadParametrosFormales{
		\tadEncabezadoInline{géneros}{$\alpha$}\\
		\tadEncabezadoInline{operaciones}{
			\tadOperacionInline{\argumento $<$ \argumento}{$\alpha$, $\alpha$}{bool} \hfill Relación de orden total estricto\footnotemark
		}
	}

	\footnotetext{\noindent Una relación es un orden total estricto cuando se cumple:
	\begin{description}
	 \item[Antirreflexividad:] $\neg$ $a < a$ para todo $a: \alpha$
	 \item[Antisimetría:] $(a < b \implies \neg\ b < a)$ para todo $a, b: \alpha$, $a \neq b$
	 \item[Transitividad:] $((a < b \land b < c) \implies a < c)$ para todo $a, b, c: \alpha$
	 \item[Totalidad:] $(a < b \lor b < a)$ para todo $a, b: \alpha$
	\end{description}
	}

	\tadGeneros{colaMinPrior($\alpha$)}
	\tadExporta{colaMinPrior($\alpha$), generadores, observadores}
	\tadUsa{\tadNombre{Bool}}

	\tadAlinearFunciones{desencolar}{$\alpha$,colaMinPrior($\alpha$)}

	\tadObservadores
	\tadOperacion{vacía?}{colaMinPrior($\alpha$)}{bool}{}
	\tadOperacion{próximo}{colaMinPrior($\alpha$)/c}{$\alpha$}{$\neg$ vacía?($c$)}
	\tadOperacion{desencolar}{colaMinPrior($\alpha$)/c}{colaMinPrior($\alpha$)}{$\neg$ vacía?($c$)}

	\tadGeneradores
	\tadOperacion{vacía}{}{colaMinPrior($\alpha$)}{}
	\tadOperacion{encolar}{$\alpha$,colaMinPrior($\alpha$)}{colaMinPrior($\alpha$)}{}

	\tadOtrasOperaciones
	\tadOperacion{tamaño}{colaMinPrior($\alpha$)}{nat}{}

	\tadAxiomas[\paratodo{colaMinPrior($\alpha$)}{c}, \paratodo{$\alpha$}{e}]
	\tadAlinearAxiomas{desencolar(encolar($e$, $c$))}{}

	\tadAxioma{vacía?(vacía)}{true}
	\tadAxioma{vacía?(encolar($e$, $c$))}{false}

	\tadAxioma{próximo(encolar($e$, $c$))}{\IF vacía?($c$) $\oluego$\ proximo($c$) $> e$ THEN $e$ ELSE próximo($c$) FI}

	\tadAxioma{desencolar(encolar($e$, $c$))}{\IF vacía?($c$) $\oluego$\ proximo($c$) $> e$ THEN $c$ ELSE encolar($e$, desencolar($c$)) FI}

	\end{tad}

\subsection{Interfaz}
	\tadParametrosFormales{
		\tadEncabezadoInline{géneros}{$\alpha$}
	}

	\textbf{se explica con}: \tadNombre{Cola de mínima prioridad(nat)}.

	\textbf{géneros}: \TipoVariable{colaMinPrior($\alpha$)}.

\subsubsection{Operaciones básicas de Cola de mínima prioridad}

	\InterfazFuncion{Vacía}{}{colaMinPrior($\alpha$)}
	[true]
	{$res$ $\igobs$ vacía}
	[O(1)]
	[Crea una cola de prioridad vacía]

	~

	\InterfazFuncion{Vacía?}{\In{c}{colaMinPrior($\alpha$)}}{bool}
	[true]
	{$res$ $\igobs$ vacía?(c)}
	[O(1)]
	[Devuelve \TipoVariable{true} si y sólo si la cola está vacía]

	~

	\InterfazFuncion{Próximo}{\In{c}{colaMinPrior($\alpha$)}}{$\alpha$}
	[$\neg$vacía?($c$)]
	{alias($res$ $\igobs$ próximo($c$))}
	[O(1)]
	[Devuelve el próximo elemento a desencolar]
	[$res$ es modificable si y sólo si $c$ es modificable]

	~

	\InterfazFuncion{Desencolar}{\Inout{c}{colaMinPrior($\alpha$)}}{$\alpha$}
	[$\neg$vacía?($c$) $\land$ $c$ $\igobs$ $c_0$]
	{$res$ $\igobs$ próximo($c_0$) $\land$ $c$ $\igobs$ desencolar($c_0$)}
	[O(log(tamaño(c)))]
	[Quita el elemento más prioritario]
	[Se devuelve el elemento por copia]

	~

	\InterfazFuncion{Encolar}{\Inout{c}{colaMinPrior($\alpha$)}, \In{p}{nat}, \In{a}{$\alpha$}}{}
	[$c$ $\igobs$ $c_0$]
	{$c$ $\igobs$ encolar($p$,$c_0$)}
	[O(log(tamaño(c)))]
	[Agrega al elemento $\alpha$ con prioridad $p$ a la cola]
	[Se agrega el elemento por copia]

	~

	\InterfazFuncion{• = •}{\In{c}{colaMinPrior($\alpha$)}, \In{c'}{colaMinPrior($\alpha$)}}{bool}
	[true]
	{$res$ $\igobs$ ($c$ $\igobs$ $c'$)}
	[O(min(tamaño($c$), tamaño($c'$)))]
	[Indica si $c$ es igual $c'$]

\subsection{Representación}

	\subsubsection{Representación de colaMinPrior}

		\begin{Estructura}{colaMinPrior($\alpha$)}[estr]
			\- \- \- \- donde \TipoVariable{estr} es \TipoVariable{diccLog(nodoEncolados)}

			\- \- \- \- donde \TipoVariable{nodoEncolados} es
			\TipoVariable{tupla}($encolados$: \TipoVariable{cola($\alpha$)},
			$prioridad$: \TipoVariable{nat})
		\end{Estructura}

	\subsubsection{Invariante de Representación}

		\renewcommand{\labelenumi}{(\Roman{enumi})}

		\begin{enumerate}
			\item Todos los significados del diccionario tienen como clave
			el valor de $prioridad$
			\item Todos los significados del diccionario no pueden tener una
			cola vacía
		\end{enumerate}

	\tadAlinearFunciones{Rep}{estr}
	\tadAlinearAxiomas{Rep(e)}

	\Rep[estr][e]{
		\\($\forall n:$ nat) def?($n$, $e$) $\impluego$
		((obtener($n$, $e$).prioridad = $n$) $\land$
		$\neg$vacía?(obtener($n$, $e$).encolados))
	}\mbox{}

	\subsubsection{Función de Abstracción}

	\tadAlinearFunciones{Abs}{Estr/e}

		\Abs[estr]{colaMinPrior}[e]{cmp}{
			(vacía?($cmp$) $\Leftrightarrow$ (\#claves($e$) = 0)) $\land$ \\
			\- $\neg$vacía?($cmp$) $\impluego$ \\
			\- \- ((próximo($cmp$) = próximo(mínimo($e$).encolados)) $\land$ \\
			\- \- (desencolar($cmp$) = desencolar(mínimo($e$).encolados)))
		}

\subsection{Algoritmos}
	\lstset{style=alg}

	\begin{lstlisting}[mathescape]
	'\alg{iVacía}{}{colaMinPrior($\alpha$)}'

	res $\leftarrow$ CrearDicc() '\ote{1}'

	'\ofi{O(1)}'
	\end{lstlisting}

	\begin{lstlisting}[mathescape]
	'\alg{iVacía?}{\In{c}{colaMinPrior($\alpha$)}}{bool}'

	res $\leftarrow$ Vacio?($c$) '\ote{1}'

	'\ofi{O(1)}'
	\end{lstlisting}

	\begin{lstlisting}[mathescape]
	'\alg{iPróximo}{\Inout{c}{colaMinPrior($\alpha$)}}{$\alpha$}'

	res $\leftarrow$ Proximo(Minimo($c$).encolados) '\ote{1}'

	'\ofi{O(1)}'
	\end{lstlisting}

	\begin{lstlisting}[mathescape]
	'\alg{iDesencolar}{\Inout{c}{colaMinPrior($\alpha$)}}{$\alpha$}'

	res $\leftarrow$ Copiar(Proximo(Minimo($c$).encolados)) '\ote{copy($\alpha$)}'
	Desencolar(Minimo($c$).encolados) '\ote{log(tamaño($c$))}'
	if EsVacia?(Minimo($c$).encolados) then '\ote{1}'
		Borrar($c$, Minimo($c$).prioridad) '\ote{log(tamaño($c$))}'
	end if

	'\ofi{O(log(tamano(c)) + O(copy(\alpha))}'
	\end{lstlisting}

	\begin{lstlisting}[mathescape]
	'\alg{iEncolar}{\Inout{c}{colaMinPrior($\alpha$)}, \In{p}{nat}, \In{a}{$\alpha$}}{}'
	if Definido?($c$, $p$) then '\ote{log(tamaño($c$))}'
		Encolar(Obtener($c$, $p$).encolados, $a$) '\ote{log(tamaño($c$)) + copy($\alpha$)}'
	else
		nodoEncolados $nuevoNodoEncolados$ '\ote{1}'
		$nuevoNodoEncolados$.encolados $\leftarrow$ Vacia() '\ote{1}'
		$nuevoNodoEncolados$.prioridad $\leftarrow$ $p$ '\ote{1}'
		Encolar($nuevoNodoEncolados$.encolados, $a$) '\ote{copy($a$)}'
		Definir($c$, $p$, $nuevoNodoEncolados$) '\ote{log(tamaño($c$)) + copy($nodoEncolados$)}'
	end if

	'\ofi{O(log(tamano(c)) + O(copy(\alpha))}'
	\end{lstlisting}

	\begin{lstlisting}[mathescape]
	'\alg{• = •}{\In{c_0}{colaMinPrior($\alpha$)}, \In{c_1}{colaMinPrior($\alpha$)}}{bool}'
	res $\leftarrow$ $c_0$ = $c_1$ '\ote{min(tamano($c_0$), tamano($c_1$))}'
	'\ofi{O(min(tamano(c_0), tamano(c_1)))}'
	\end{lstlisting}



\pagebreak
\section{Módulo Diccionario logarítmico($\alpha$)}

\subsection{Interfaz}

\textbf{se explica con}: \tadNombre{Diccionario(nat, $\alpha$)}.

\textbf{géneros}: \TipoVariable{diccLog($\alpha$)}.

\subsubsection{Operaciones básicas de Diccionario logarítimico($\alpha$)}

	\InterfazFuncion{CrearDicc}{}{diccLog($\alpha$)}
	[true]
	{$res$ $\igobs$ vacío}%
	[$O(1)$]
	[Crea un diccionario vacío]
	[]

	~

	\InterfazFuncion{Definido?}{\In{d}{diccLog($\alpha$)}, \In{c}{nat}}{bool}
	[true]
	{$res$ $\igobs$ def?($c$, $d$)}
	[$O(log(\#claves(d)))$]
	[Devuelve \TipoVariable{true} si y sólo si la clave fue previamente definida en el diccionario]
	[]

	~

	\InterfazFuncion{Definir}{\Inout{d}{diccLog($\alpha$)}, \In{c}{nat}, \In{s}{$\alpha$}}{}
	[$d$ $\igobs$ $d_0$]
	{$d$ $\igobs$ definir($c$, $s$, $d_0$)}
	[$O(log(\#claves(d)) + copy(s))$]
	[Define la clave $c$ con el significado $s$ en $d$]
	[]

	~

	\InterfazFuncion{Obtener}{\Inout{d}{diccLog($\alpha$)}, \In{c}{nat}}{$\alpha$}
	[def?($c$, $d$)]
	{alias($res$ $\igobs$ obtener($c$, $d$))}
	[$O(log(\#claves(d)))$]
	[Devuelve el significado correspondiente a la clave en el diccionario]
	[$res$ es modificable si y sólo si $d$ es modificable]

	~

	\InterfazFuncion{Borrar}{\Inout{d}{diccLog($\alpha$)}, \In{c}{nat}}{}
	[def?($c$, $d$)]
	{$res$ $\igobs$ borrar($c$, $d$))}
	[$O(log(\#claves(d)))$]
	[Borra el elemento con la clave dada]
	[]

	~

	\InterfazFuncion{Vacío?}{\In{d}{diccLog($\alpha$)}}{bool}
	[true]
	{res $\igobs$ $\emptyset$?(claves($d$))}
	[$O(1)$]
	[Devuelve \TipoVariable{true} si y sólo si el diccionario está vacío]
	[]

	~

	\InterfazFuncion{Mínimo}{\Inout{d}{diccLog($\alpha$)}}{$\alpha$}
	[$\neg\emptyset$?(claves($d$))]
	{alias($res$ $\igobs$ obtener(claveMínima($d$), $d$))}
	[$O(1)$]
	[Devuelve el significado correspondiente a la clave de mínimo valor en el diccionario]
	[$res$ es modificable si y sólo si $d$ es modificable]

	~

	\InterfazFuncion{• = •}{\In{d}{diccLog($\alpha$)}, \In{d'}{diccLog($\alpha$)}}{bool}
	[true]
	{$res$ $\igobs$ (d $\igobs$ d')}
	[$O(max($\#$claves(d), $\#$claves(d')))$]
	[Devuelve \TipoVariable{true} si y sólo si ambos diccionarios son iguales]
	[]

\subsubsection{Operaciones auxiliares del TAD}

\tadAlinearFunciones{darClaveMínima}{dicc(nat, $\alpha$)/d,conj(nat)/c}

	\tadOperacion{claveMínima}{dicc(nat, $\alpha$)/d}{nat}{$\neg\emptyset$?(claves($d$))}
	\tadOperacion{darClaveMínima}{dicc(nat, $\alpha$)/d,conj(nat)/c}{nat}{$\neg\emptyset$?(claves($d$)) $\land$ (c $\subseteq$ claves($d$))}

\tadAlinearAxiomas{darClaveMínima($d$, $c$)}

	\tadAxioma{claveMínima($d$)}{darClaveMínima($d$, claves($d$))}

	\tadAxioma{darClaveMínima($d$, $c$)}{
		\IF $\emptyset$?(sinUno($c$)) THEN
			dameUno($c$)
		ELSE {
			min(dameUno($c$), darClaveMínima($d$, sinUno($c$)))
			}
		FI
	}

\subsection{Representación}

	\subsubsection{Representación de diccLog($\alpha$)}

	\begin{Estructura}{diccLog($\alpha$)}[estr]
		\begin{Tupla}[estr]
			\tupItem{abAvl}{ab(nodoAvl)}%
			\tupItem{mínimo}{puntero(ab(nodoAvl))}%
		\end{Tupla}

		\begin{Tupla}[nodoAvl]
			\tupItem{clave}{nat}%
			\tupItem{data}{$\alpha$}%
			\tupItem{balance}{int}%
		\end{Tupla}
	\end{Estructura}

\lstset{style=alg,columns=fixed,basewidth=.5em}

	\subsubsection{Invariante de Representación}
	\begin{enumerate}
		\item{Se mantiene el invariante de árbol binario de búsqueda para las claves de los nodos.}
		\item{Cada nodo tiene $balance$ $\in$ \{-1, 0, 1 \} donde $balance$ es:$\newline$
			* 0 si el arbol esta balanceado $\newline$
			* 1 si la diferencia en altura entre el hijo derecho y el izquierdo
		es de uno $\newline$
			* -1 si la diferencia en altura entre el hijo izquierdo y el derecho
		es de uno}
		\item{Todas las claves son distintas.}
		\item{El $minimo$ apunta al árbol con el la clave de menor
			valor, si el diccionario está vacío vale \TipoVariable{NULL}.}

	\end{enumerate}

	\tadAlinearFunciones{Rep}{estr}
	\tadAlinearAxiomas{Rep(e)}

	\Rep[estr][e]{esABB($e$.abAvl) $\land$ balanceadoBien($e$.abAvl) $\land$
		clavesÚnicas($e$.abAvl, vacío) $\yluego$ \\
		$e$.mínimo = árbolClaveMínima($e$.abAvl)}

		\tadAlinearFunciones{alturaBienwachooooo}{puntero(nodoAvl), conj(nat)}
		\tadAlinearAxiomas{árbolClaveMínima($a$)}

		\tadOperacion{esABB}{ab(nodoAvl)}{bool}{}
		\tadOperacion{balanceadoBien}{ab(nodoAvl)}{bool}{}
		\tadOperacion{clavesEnÁrbol}{ab(nodoAvl)}{conj(nat)}{}
		\tadOperacion{clavesÚnicas}{ab(nodoAvl)}{bool}{}
		\tadOperacion{árbolClaveMínima}{ab(nodoAvl)}{puntero(ab(nodoAvl))}{}
		\tadOperacion{darSignificado}{ab(nodoAvl)/a, nat/c}{$\alpha$}{c $\in$ clavesEnÁrbol($a$) $\land$ esABB($a$)}

		~

		\tadAxioma{esABB($a$)}{
			($\neg$nil?($a$)) $\impluego$ ( \\
			\- ($\neg$nil?(izq($a$)) $\impluego$
				(raíz($a$).clave $>$ raíz(izq($a$)).clave $\land$
				esABB(izq($a$)))) $\land$ \\
			\- ($\neg$nil?(der($a$)) $\impluego$
				(raíz($a$).clave $<$ raíz(der($a$)).clave $\land$
				esABB(der($a$)))) \\
			)
		}

		~

		\tadAxioma{balanceadoBien($a$)}{
			\IF (nil?($a$)) THEN
				true
			ELSE{
				(abs(altura(der($a$)) - altura(izq($a$))) $<$ 2) $\land$ \\
				(raíz($a$)$\rightarrow$balance = altura(der($a$)) - altura(izq($a$))) $\land$ \\
				balanceadoBien(izq($a$)) $\land$ balanceadoBien(der($a$))
			} FI
		}

		~

		\tadAxioma{clavesEnÁrbol($a$)}{
			\IF (nil?($a$)) THEN
				$\emptyset$
			ELSE{
				Ag(raíz($a$).clave, (clavesEnÁrbol(izq($a$)) $\cup$
					clavesEnÁrbol(der($a$))))
			} FI
		}

		~

		\tadAxioma{clavesÚnicas($a$)}{
			tamaño($a$) = $\#$(clavesEnÁrbol($a$))
		}

		~

		\tadAxioma{árbolClaveMínima($a$)}{
			\IF (nil?($a$)) THEN
				NULL
			ELSE{
				\IF (nil?(izq($a$))) THEN
					puntero($a$)
				ELSE{
					árbolClaveMínima(izq($a$))
				} FI
			} FI
		}

		~

		\tadAxioma{darSignificado($a$, $c$)}{
			\IF (raíz($a$).clave = $c$) THEN
				raíz($a$).data
			ELSE{
				\IF (raíz($a$).clave < $c$) THEN
					darSignificado(izq($a$), $c$)
				ELSE{
					darSignificado(der($a$), $c$)
				} FI
			} FI
		}

	\subsubsection{Función de Abstracción}
	\tadAlinearFunciones{Abs}{Estr/e}
	\Abs[estr]{dicc(nat, $\alpha$)}[e]{d}{
		($\forall n$: nat) ( \\
		\- (def?($n$, $d$) $\Leftrightarrow$ $n$ $\in$ clavesEnÁrbol($e$.abAvl)) $\yluego$ \\
		\- (def?($n$, $d$) $\impluego$ obtener($n$, $d$) = darSignificado($e$.abAvl, $n$)) \\
		)
	}

\subsection{Algoritmos}
\begin{lstlisting}[mathescape]
'\alg{iCrearDicc}{}{diccLog($\alpha$)}'
	res $\leftarrow$ <Nil, NULL> '\ote{1}'
'\ofi{O(1)}'
\end{lstlisting}

\pagebreak
\begin{lstlisting}[mathescape]
'\alg{iDefinir}{\Inout{diccLog(\alpha)}{d}, \In{nat}{c}, \In{\alpha}{s}}{}'
	if (Nil?(d.abAvl)) then '\ote{1}'
		d.abAvl $\leftarrow$ crearArbol(c, s) '\ote{copy(s)}'
		d.minimo $\leftarrow$ puntero(d.abAvl) '\ote{1}'
	else
		it: ab(nodoAvl) $\leftarrow$ d.abAvl '\ote{1}'
		up: pila(puntero(ab(nodoAvl))) '\ote{1}'
		upd: pila(bool) '\ote{1}'

		Apilar(upd, (Raiz(it).clave < c)) '\ote{1}'
		Apilar(up, puntero(it)) '\ote{1}'

		while($\neg$Nil?(subArbol(it, Tope(upd)))) '\ote{1}'
			it $\leftarrow$ subArbol(it, Tope(upd)) '\ote{1}'

			Apilar(upd, (Raiz(it).clave < c)) '\ote{1}'
			Apilar(up, puntero(it)) '\ote{1}'
		end while '\ote{log($\#$claves(d))}'

		subArbol(it, Tope(upd)) $\leftarrow$ crearArbol(c, s) '\ote{copy(s)}'
		if(c < Raiz(*d.minimo).clave) then '\ote{1}'
			d.minimo $\leftarrow$ puntero(subArbol(it, Tope(upd))) '\ote{1}'
		end if

		break $\leftarrow$ false '\ote{1}'
		while((Tamano(up) > 0) $\land$ $\neg$break) '\ote{1}'
			if(Tope(upd)) then '\ote{1}'
				Raiz(*Tope(up)).balance $\leftarrow$ Raiz(*Tope(up)).balance + 1 '\ote{1}'
			else
				Raiz(*Tope(up)).balance $\leftarrow$ Raiz(*Tope(up)).balance - 1 '\ote{1}'
			end if

			if(Raiz(*Tope(up)).balance = 0) then '\ote{1}'
				break $\leftarrow$ true '\ote{1}'
			else
				if(abs(Raiz(*Tope(up)).balance > 1)) then '\ote{1}'
					*Tope(up) $\leftarrow$ puntero(insertarBalance(*Tope(up), Tope(upd))) '\ote{1}'

					if(Tamano(up) > 1) then '\ote{1}'
						upTope: puntero(ab(nodoAvl)) $\leftarrow$ copy(Tope(up)) '\ote{1}'
						Desapilar(up) '\ote{1}'
						Desapilar(upd) '\ote{1}'
						subArbol(*Tope(up), Tope(upd)) $\leftarrow$ *upTope '\ote{1}'
					else
						d.abAvl $\leftarrow$ *Tope(up) '\ote{1}'
					end if

					break $\leftarrow$ true '\ote{1}'
				else
					Desapilar(up) '\ote{1}'
					Desapilar(upd) '\ote{1}'
				end if
			end if
		end while '\ote{log($\#$claves(d))}'
end if
'\ofi{O(log($\#$claves(d))) + O(copy(s))}'
\end{lstlisting}

\begin{lstlisting}[mathescape]
'\alg{crearArbol}{\In{nat}{c}, \In{\alpha}{s}}{ab(nodoAvl)}'
	res $\leftarrow$ Bin(Nil, <c, copy(s), 0>, Nil) '\ote{copy(s)}'
	'\ofi{O(copy(s))}'
\end{lstlisting}

\begin{lstlisting}[mathescape]
'\alg{subArbol}{\Inout{ab(nodoAvl)}{a}, \In{bool}{dir}}{ab(nodoAvl)}'
	if(dir) then '\ote{1}'
		res $\leftarrow$ Der(a) '\ote{1}'
	else
		res $\leftarrow$ Izq(a) '\ote{1}'
	end if
	'\ofi{O(1)}'
\end{lstlisting}

\begin{lstlisting}[mathescape]
'\alg{insertarBalance}{\Inout{ab(nodoAvl)}{root}, \In{bool}{dir}}{ab(nodoAvl)}'
	hijo: ab(nodoAvl) $\leftarrow$ subArbol(root, dir) '\ote{1}'

	if(dir) then '\ote{1}'
		bal: int $\leftarrow$ 1 '\ote{1}'
	else
		bal: int $\leftarrow$ -1 '\ote{1}'
	end if

	if(Raiz(hijo).balance = bal) then '\ote{1}'
		Raiz(root).balance $\leftarrow$ 0 '\ote{1}'
		Raiz(hijo).balance $\leftarrow$ 0 '\ote{1}'
		root $\leftarrow$ rotacionSimple(root, $\neg$dir) '\ote{1}'
	else
		ajustarBalance(root, dir, bal) '\ote{1}'
		root $\leftarrow$ rotacionDoble(root, $\neg$dir) '\ote{1}'
	end if

	res $\leftarrow$ root '\ote{1}'
'\ofi{O(1)}'
\end{lstlisting}

\begin{lstlisting}[mathescape]
'\alg{rotacionSimple}{\Inout{ab(nodoAvl)}{root}, \In{bool}{dir}}{ab(nodoAvl)}'
	hijo: ab(nodoAvl) $\leftarrow$ subArbol(root, $\neg$dir) '\ote{1}'

	subArbol(root, $\neg$dir) $\leftarrow$ subArbol(hijo, dir) '\ote{1}'
	subArbol(hijo, dir) $\leftarrow$ root '\ote{1}'

	res $\leftarrow$ hijo '\ote{1}'
'\ofi{O(1)}'
\end{lstlisting}

\begin{lstlisting}[mathescape]
'\alg{rotacionDoble}{\Inout{ab(nodoAvl)}{root}, \In{bool}{dir}}{ab(nodoAvl)}'
	nieto: ab(nodoAvl) $\leftarrow$ subArbol(subArbol(root, $\neg$dir), dir) '\ote{1}'

	subArbol(subArbol(root, $\neg$dir), dir) $\leftarrow$ subArbol(nieto, $\neg$dir) '\ote{1}'
	subArbol(nieto, $\neg$dir) $\leftarrow$ subArbol(root, $\neg$dir) '\ote{1}'
	subArbol(root, $\neg$dir) $\leftarrow$ nieto '\ote{1}'

	// ... sigue


	nieto $\leftarrow$ subArbol(root, $\neg$dir) '\ote{1}'
	subArbol(root, $\neg$dir) $\leftarrow$ subArbol(nieto, dir) '\ote{1}'
	subArbol(nieto, dir) $\leftarrow$ root '\ote{1}'

	res $\leftarrow$ nieto '\ote{1}'
'\ofi{O(1)}'
\end{lstlisting}

\begin{lstlisting}[mathescape]
'\alg{ajustarBalance}{\Inout{ab(nodoAvl)}{root}, \In{bool}{dir}, \In{int}{bal}}{}'
	hijo: ab(nodoAvl) $\leftarrow$ subArbol(root, dir) '\ote{1}'
	nieto: ab(nodoAvl) $\leftarrow$ subArbol(hijo, $\neg$dir) '\ote{1}'

	if(Raiz(nieto).balance = 0) then '\ote{1}'
		Raiz(root).balance $\leftarrow$ 0 '\ote{1}'
		Raiz(hijo).balance $\leftarrow$ 0 '\ote{1}'
	else
		if(Raiz(nieto).balance = bal) then '\ote{1}'
			Raiz(root).balance $\leftarrow$ -bal '\ote{1}'
			Raiz(hijo).balance $\leftarrow$ 0 '\ote{1}'
		else
			Raiz(root).balance $\leftarrow$ 0 '\ote{1}'
			Raiz(hijo).balance $\leftarrow$ bal '\ote{1}'
		end if
	end if

	Raiz(nieto).balance $\leftarrow$ 0 '\ote{1}'
'\ofi{O(1)}'
\end{lstlisting}

\begin{lstlisting}[mathescape]
'\alg{iBorrar}{\Inout{diccLog(\alpha)}{d}, \In{nat}{c}}{}'
	it: ab(nodoAvl) $\leftarrow$ d.abAvl '\ote{1}'
	padre: ab(nodoAvl) $\leftarrow$ Nil '\ote{1}'
	padreDir: bool $\leftarrow$ false '\ote{1}'
	up: pila(puntero(ab(nodoAvl))) '\ote{1}'
	upd: pila(bool) '\ote{1}'

	while(Raiz(it).clave $\neq$ c) '\ote{1}'
		Apilar(upd, (Raiz(it).clave < c)) '\ote{1}'
		Apilar(up, puntero(it)) '\ote{1}'

		padre $\leftarrow$ it '\ote{1}'
		padreDir $\leftarrow$ Tope(upd) '\ote{1}'

		it $\leftarrow$ subArbol(it, Tope(upd)) '\ote{1}'
	end while '\ote{log($\#$claves(d))}'

	if(Raiz(it).clave = Raiz(*d.minimo).clave) then '\ote{1}'
		if(Nil?(padre)) then '\ote{1}'
			d.minimo $\leftarrow$ NULL '\ote{1}'
		else
			d.minimo $\leftarrow$ puntero(padre) '\ote{1}'
		end if
	end if

	// ... sigue

	if(Nil?(Izq(it)) $\lor$ Nil?(Der(it))) then '\ote{1}'
		dir: bool $\leftarrow$ Nil?(Izq(it)) '\ote{1}'

		if(Tamano(up) > 1) then '\ote{1}'
			SubArbol(*Tope(up), Tope(upd)) $\leftarrow$ subArbol(it, dir) '\ote{1}'
		else
			d.abAvl $\leftarrow$ subArbol(it, dir) '\ote{1}'
		end if
	else
		heredero: ab(nodoAvl) $\leftarrow$ Der(it) '\ote{1}'

		Apilar(Tope(upd), true) '\ote{1}'
		Apilar(Tope(up), puntero(it)) '\ote{1}'

		while($\neg$Nil?(Izq(heredero)) '\ote{1}'
			Apilar(upd, false) '\ote{1}'
			Apilar(up, puntero(heredero)) '\ote{1}'
			heredero $\leftarrow$ Izq(heredero) '\ote{1}'
		end while '\ote{log($\#$claves(d))}'

		subArbol(*Tope(up), Tope(up) = puntero(it)) $\leftarrow$ Der(heredero) '\ote{1}'
		Izq(heredero) $\leftarrow$ Izq(it) '\ote{1}'
		Der(heredero) $\leftarrow$ Der(it) '\ote{1}'
		if($\neg$Nil?(padre)) then '\ote{1}'
			subArbol(padre, padreDir) $\leftarrow$ heredero '\ote{1}'
		end if
	end if

	break: bool $\leftarrow$ false '\ote{1}'
	while((Tamano(up) > 0) $\land$ $\neg$break) '\ote{1}'
		if(Tope(upd)) then '\ote{1}'
			Raiz(*Tope(up)).balance $\leftarrow$ Raiz(*Tope(up)).balance + 1 '\ote{1}'
		else
			Raiz(*Tope(up)).balance $\leftarrow$ Raiz(*Tope(up)).balance - 1 '\ote{1}'
		end if

		if(abs(Raiz(*Tope(up)).balance) = 1) then '\ote{1}'
			break $\leftarrow$ true '\ote{1}'
		else
			if(abs(Raiz(*Tope(up)).balance) > 1) then '\ote{1}'
				*Tope(up) $\leftarrow$ removerBalanceo(*Tope(up), Tope(upd), $\&$break) '\ote{1}'
				if(Tamano(up) > 1) then '\ote{1}'
					upTope: puntero(ab(nodoAvl)) $\leftarrow$ copy(Tope(up)) '\ote{1}'
					Desapilar(up) '\ote{1}'
					Desapilar(upd) '\ote{1}'
					subArbol(*Tope(up), Tope(upd)) $\leftarrow$ *upTope '\ote{1}'
				else
					d.abAvl $\leftarrow$ *Tope(up) '\ote{1}'
				end if
			else
				Desapilar(up) '\ote{1}'
				Desapilar(upd) '\ote{1}'
			end if
		end if
	end while '\ote{log($\#$claves(d))}'
	'\ofi{O(log($\#$claves(d)) + copy(data))}'
\end{lstlisting}

\begin{lstlisting}[mathescape]
'\alg{removerBalanceo}{\Inout{ab(nodoAvl)}{root}, \In{bool}{dir}, \Inout{puntero(bool)}{done}}{ab(nodoAvl)}'
	hijo: ab(nodoAvl) $\leftarrow$ subArbol(root, $\neg$dir) '\ote{1}'

	if(dir) then '\ote{1}'
		bal $\leftarrow$ 1 '\ote{1}'
	else
		bal $\leftarrow$ -1 '\ote{1}'
	end if

	if(Raiz(hijo).balance = -bal) then '\ote{1}'
		Raiz(root).balance $\leftarrow$ 0 '\ote{1}'
		Raiz(hijo).balance $\leftarrow$ 0 '\ote{1}'
		root $\leftarrow$ rotacionSimple(root, dir) '\ote{1}'
	else
		if(Raiz(hijo).balance) = bal) then '\ote{1}'
			ajustarBalance(root, $\neg$dir, -bal) '\ote{1}'
			root $\leftarrow$ rotacionDoble(root, dir) '\ote{1}'
		else
			Raiz(root).balance $\leftarrow$ -bal '\ote{1}'
			Raiz(hijo).balance $\leftarrow$ bal '\ote{1}'
			root $\leftarrow$ rotacionSimple(root, dir) '\ote{1}'
			*done $\leftarrow$ true '\ote{1}'
		end if
	end if

	res $\leftarrow$ root '\ote{1}'
'\ofi{O(1)}'
\end{lstlisting}

\begin{lstlisting}[mathescape]
'\alg{iMínimo}{\In{diccLog(\alpha)}{d}}{$\alpha$}'
	res $\leftarrow$ Raiz(*d.minimo).data '\ote{1}'
'\ofi{O(1)}'
\end{lstlisting}

\begin{lstlisting}[mathescape]
'\alg{iDefinido?}{\In{diccLog(\alpha)}{d}, \In{nat}{c}}{bool}'
	definido:bool $\leftarrow$ false '\ote{1}'
	it:ab(nodoAvl) $\leftarrow$ d.abAvl '\ote{1}'

	while($\neg$Nil?(it) $\land$ $\neg$definido) do '\ote{1}'
		definido $\leftarrow$ (Raiz(it).clave = c) '\ote{1}'
		it $\leftarrow$ subArbol(it, (Raiz(it).clave < c)) '\ote{1}'
	end while '\ote{log($\#$claves(d))}'

	res $\leftarrow$ definido '\ote{1}'
'\ofi{O(log($\#$claves(d)))}'
\end{lstlisting}

\pagebreak

\begin{lstlisting}[mathescape]
'\alg{iObtener}{\Inout{diccLog(\alpha)}{d}, \In{nat}{c}}{$\alpha$}'
	it:ab(nodoAvl) $\leftarrow$ d.abAvl '\ote{1}'

	while(Raiz(it).clave $\neq$ c) do '\ote{1}'
		it $\leftarrow$ subArbol(it, (Raiz(it).clave < c)) '\ote{1}'
	end while '\ote{log($\#$claves(d))}'

	res $\leftarrow$ Raiz(it).data '\ote{1}'
'\ofi{O(log($\#$claves(d)))}'
\end{lstlisting}

\begin{lstlisting}[mathescape]
'\alg{iVacio?}{\In{diccLog(\alpha)}{d}}{bool}'
	res $\leftarrow$ Nil?(d.abAvl) '\ote{1}'
'\ofi{O(1)}'
\end{lstlisting}

\begin{lstlisting}[mathescape]
'\alg{inorder}{\In{diccLog(\alpha)}{d}}{lista(tupla(nat, $\alpha$))}'
	root:ab(nodoAvl) $\leftarrow$ d.abAvl '\ote{1}'
	p:pila(puntero(ab(nodoAvl))) $\leftarrow$ Vacia() '\ote{1}'
	done:bool $\leftarrow$ false '\ote{1}'
	res $\leftarrow$ Vacia() '\ote{1}'

	while (!done) do '\ote{1}'
		if ($\neg$Nil?(root)) then '\ote{1}'
			Apilar(p, puntero(root)) '\ote{1}'
			root $\leftarrow$ Izq(root) '\ote{1}'
		else
			if $\neg$EsVacia?(p) then '\ote{1}'
				AgregarAtras(res, <Raiz(*Tope(p)).clave, Raiz(*Tope(p)).data>) '\ote{1}'
				root $\leftarrow$ Der(*Tope(p)) '\ote{1}'
			else
				done $\leftarrow$ true '\ote{1}'
			end if
		end if
	end while '\ote{$\#$claves(d)}'
'\ofi{O($\#$claves(d))}'
\end{lstlisting}

\begin{lstlisting}[mathescape]
'\alg{• = •}{\In{diccLog(\alpha)}{d1}, \In{diccLog(\alpha)}{d2}}{bool}'
	res $\leftarrow$ inorder(d1) = inorder(d2) '\ote{max($\#$claves(d1), $\#$claves(d2))}'
'\ofi{O(max($\#$claves(d1), $\#$claves(d2)))}'
\end{lstlisting}

\pagebreak
\section{Módulo Árbol binario($\alpha$)}

\subsection{Interfaz}

\textbf{se explica con}: \tadNombre{Árbol binario($\alpha$)}.

\textbf{géneros}: \TipoVariable{ab($\alpha$)}.

\subsubsection{Operaciones básicas de Árbol binario($\alpha$)}

	\InterfazFuncion{Nil}{}{ab($\alpha$)}
	[true]
	{$res$ $\igobs$ nil}%
	[$O(1)$]
	[Crea un árbol binario nulo]
	[]

	~

	\InterfazFuncion{Bin}{\In{i}{ab($\alpha$)}, \In{r}{$\alpha$}, \In{d}{ab($\alpha$)}}{ab($\alpha$)}
	[true]
	{$res$ $\igobs$ bin($i$, $r$, $d$)}%
	[$O(copy(r) + copy(i) + copy(d))$]
	[Crea un árbol binario con hijo izquierdo $i$, hijo derecho $d$ y raíz de valor $r$]
	[]

	~

	\InterfazFuncion{Raíz}{\Inout{a}{ab($\alpha$)}}{$\alpha$}
	[$\neg$nil?($a$)]
	{alias($res$ $\igobs$ raíz($a$))}%
	[$O(1)$]
	[Devuelve el valor de la raíz del árbol]
	[$res$ es modificable si y sólo si $a$ lo es]

	~

	\InterfazFuncion{Izq}{\Inout{a}{ab($\alpha$)}}{ab($\alpha$)}
	[$\neg$nil?($a$)]
	{alias($res$ $\igobs$ izq($a$))}%
	[$O(1)$]
	[Devuelve el hijo izquierdo]
	[$res$ es modificable si y sólo si $a$ lo es]

	~

	\InterfazFuncion{Der}{\Inout{a}{ab($\alpha$)}}{ab($\alpha$)}
	[$\neg$nil?($a$)]
	{alias($res$ $\igobs$ der($a$))}%
	[$O(1)$]
	[Devuelve el hijo derecho]
	[$res$ es modificable si y sólo si $a$ lo es]

	~

	\InterfazFuncion{Nil?}{\Inout{a}{ab($\alpha$)}}{bool}
	[true]
	{$res$ $\igobs$ nil?($a$)}%
	[$O(1)$]
	[Devuelve \TipoVariable{true} si $res$ es un árbol vacío]
	[]

\subsection{Representación}

	\subsubsection{Representación de ab($\alpha$)}

	\begin{Estructura}{ab($\alpha$)}[estr]
		\- \- \- \- donde \TipoVariable{estr} es \TipoVariable{puntero(nodoAb)}

		\- \- \- \- donde \TipoVariable{nodoAb} es
			\TipoVariable{tupla}(
				$raiz$: \TipoVariable{$\alpha$},
				$hijos$: \TipoVariable{arreglo[2] de ab($\alpha$)}
			)

	\end{Estructura}

	\subsubsection{Invariante de Representación}

		\renewcommand{\labelenumi}{(\Roman{enumi})}

		\begin{enumerate}
			\item No puede haber ciclos en el árbol
			\item Los hijos no pueden apuntar a un mismo árbol
		\end{enumerate}

	\subsubsection{Función de Abstracción}

		\tadAlinearFunciones{Abs}{Estr/e}
		\Abs[estr]{ab($\alpha$)}[e]{abn}{
			(nil?($abn$) $\Leftrightarrow$ $e$ = NULL) $\land$ \\
			($\neg$nil?($abn$) $\impluego$ \\
			\- (raíz($abn$) = $e$ $\rightarrow$ raíz $\land$
				izq($abn$) = $e$ $\rightarrow$ hijos[0] $\land$
				der($abn$) = $e$ $\rightarrow$ hijos[1]) \\
			)
		}

\subsection{Algoritmos}
	\lstset{style=alg}

	\begin{lstlisting}[mathescape]
	'\alg{iNil}{}{ab($\alpha$)}'

	res $\leftarrow$ NULL '\ote{1}'

	'\ofi{O(1)}'
	\end{lstlisting}

	\begin{lstlisting}[mathescape]
	'\alg{iBin}{\In{i}{ab($\alpha$)}, \In{r}{$\alpha$},	\In{d}{ab($\alpha$)}}{ab($\alpha$)}'

	nuevoAb:nodoAb '\ote{1}'
	nuevoAb.raiz $\leftarrow$ copy(r) '\ote{copy(r)}'
	nuevoAb.hijos[0] $\leftarrow$ copy(i) '\ote{copy(i)}'
	nuevoAb.hijos[1] $\leftarrow$ copy(d) '\ote{copy(d)}'

	res $\leftarrow$ puntero(nuevoAb) '\ote{1}'

	'\ofi{O(copy(r) + copy(i) + copy(d))}'
	\end{lstlisting}

	\begin{lstlisting}[mathescape]
	'\alg{iRaíz}{\Inout{a}{ab($\alpha$)}}{$\alpha$}'

	res $\leftarrow$ ($a$ $\rightarrow$ raiz) '\ote{1}'

	'\ofi{O(1)}'
	\end{lstlisting}
\pagebreak
	\begin{lstlisting}[mathescape]
	'\alg{iIzq}{\Inout{a}{ab($\alpha$)}}{ab($\alpha$)}'

	res $\leftarrow$ ($a$ $\rightarrow$ hijos[0]) '\ote{1}'

	'\ofi{O(1)}'
	\end{lstlisting}

	\begin{lstlisting}[mathescape]
	'\alg{iDer}{\Inout{a}{ab($\alpha$)}}{ab($\alpha$)}'

	res $\leftarrow$ ($a$ $\rightarrow$ hijos[1]) '\ote{1}'

	'\ofi{O(1)}'
	\end{lstlisting}

	\begin{lstlisting}[mathescape]
	'\alg{iNil?}{\In{a}{ab($\alpha$)}}{bool}'

	res $\leftarrow$ ($a$ = NULL) '\ote{1}'

	'\ofi{O(1)}'
	\end{lstlisting}

\pagebreak
\section{Módulo Diccionario String($\alpha$)}

\subsection{Interfaz}

\textbf{se explica con}: \tadNombre{Diccionario(string, $\alpha$)}.
\textbf{géneros}: \TipoVariable{diccString$(\alpha)$}.

Se representa mediante un árbol n-ario con invariante de trie

~

\InterfazFuncion{CrearDicc}{}{diccString$(\alpha)$}%
[true]
{$res$ $\igobs$ vacío()}
[$O(1)$]
[Crea un diccionario vacío.]
[]

~

\InterfazFuncion{Definido?}{\In{d}{diccString$(\alpha)$}, \In{c}{string})}{bool}
[true]
{$res$ $\igobs$ def?($d$, $c$)}
[$O(L)$]
[Devuelve true si la clave está definida en el diccionario y false en caso contrario.]
[]

~

\InterfazFuncion{Definir}{\In{d}{diccString$(\alpha)$}, \In{c}{string}, \In{s}{$\alpha$}}{}
[$ d \igobs d_0 $]
{$ d \igobs$ definir($c$, $s$, $d_0$)}
[$O(L)$ ]
[Define la clave $c$ con el significado $s$]
[Almacena una copia de $s$.]

~

\InterfazFuncion{Obtener}{\In{d}{diccString$(\alpha)$}, \In{c}{string}}{$\alpha$}
[def?($c$, $d$)]
{alias($res$ $\igobs$ obtener($c$, $d$))}
[$O(L)$]
[Devuelve el significado correspondiente a la clave $c$.]
[Devuelve el significado almacenado en el diccionario, por lo que $res$ es modificable si y sólo si $d$ lo es.]

~

\InterfazFuncion{• = •}{\Inout{d}{diccString($\alpha$)}, \Inout{d'}{diccString($\alpha$)}}{bool}
[true]
{$res$ $\igobs$ (d $\igobs$ d')}
[$O(L*n*(\alpha  \igobs \alpha'))$]
[Indica si d es igual d']
[]

~

\InterfazFuncion{Copiar}{\In{dicc}{diccString$(\alpha)$}}{diccString$(\alpha)$}
[true]
{$res$ $\igobs$ dicc}
[$O(n * L * copy(\alpha))$]
[Devuelve una copia del diccionario]
[]



\end{document}
