\section{Módulo Diccionario Nat Fijo}

\subsection{Interfaz}

\textbf{se explica con}: \tadNombre{Diccionario(nat, $\alpha$ )}.

\textbf{géneros}: \TipoVariable{DiccNat($\alpha$)}, \TipoVariable{itDiccNat($\alpha$)}.

\subsubsection{Operaciones básicas de DiccNat($\alpha$)}

\InterfazFuncion{crearDiccionario}{\In{v}{vector(tupla(clave : nat, significado :  $\alpha$))}}{DiccNat($\alpha$)}
[true]
{(($\forall$ $t$ : tupla(nat, $\alpha$)) esta?($t$, v)) $\Rightarrow$ ((definido?(t.clave, res))
	$\land_L$ obtener(t.clave,res) $=_{obs}$ t.significado) $\land$ cantClaves(res) $=_{obs}$ longitud(v)}
[$O(copy(\alpha) * n)$ donde n es el largo del vector]
[Agrega las tuplas de clave-significado pasadas por parametro al diccionario $d$]
[Los elementos pasados por parámetros se copian al diccionario]

~

\InterfazFuncion{redefinir}{\Inout{d}{DiccNat($\alpha$), \In{n}{nat}, \In{a}{$\alpha$}}}{}
[definido?(n,d)]
{obtener(n,d) $=_{obs}$ a}
[$O(1)$ en caso promedio, $O(n)$ en peor caso]
[]

~

\InterfazFuncion{obtener}{\In{n}{nat}, \In{d}{DiccNat($\alpha$)}}{puntero($\alpha$)}
[true]
{res $=_{obs}$ obtener(n,d)}
[$O(1)$ en caso promedio, $O(n)$ en peor caso]
[Devuelve un puntero al significado de la clave pasada por parametro. Si no está definido, devuelve NULL]
[El puntero va ser una referencia al significado almacenado en el diccionario]

~

\InterfazFuncion{definido?}{\In{n}{nat}, \In{d}{DiccNat($\alpha$)}}{bool}
[true]
{res $=_{obs}$ def?(n,d)}
[$O(1)$ en caso promedio, $O(n)$ en peor caso]
[Dice si está definida una clave en el diccionario]
[]

~

\InterfazFuncion{cantClaves}{\In{d}{DiccNat($\alpha$)}}{Nat}
[true]
{res $=_{obs} \ \# claves$($d$)}
[$O(1)$]
[Devuelve la cantidad de claves definidas en el diccionario.]
[]

~

\subsubsection{Operaciones básicas del iterador}

Este iterador permite recorrer la tabla de hash sobre la que est\'{a} implementado el diccionario para obtener cada clave con su respectivo significado sin modificar ning\'{u}n dato del diccionario.

\InterfazFuncion{CrearIt}{\In{d}{DiccNat($\alpha$)}}{itDiccNat($\alpha$)}
[true]
{alias(esPermutación(SecuSuby($res$), $d$)) $\land$ vacia?(Anteriores($res$))}
[$\Theta(1)$]
[crea un iterador bidireccional del diccionario, de forma tal que HayAnterior evalúe a false (i.e.,
que se pueda recorrer los elementos aplicando iterativamente Siguiente).]
[El iterador se invalida si y sólo si se elimina el elemento siguiente del iterador sin utilizar la función
EliminarSiguiente. Además, anteriores(res) y siguientes(res) podrían cambiar completamente ante cualquier
operación que modifique d sin utilizar las funciones del iterador.]

~

\InterfazFuncion{HaySiguiente}{\In{it}{itDiccNat($\alpha$)}}{bool}
[true]
{res $=_{obs}$ haySiguiente?($it$)}
[$\Theta(1)$]
[devuelve true si y sólo si en el iterador todavía quedan elementos para avanzar.]
[]

~

\InterfazFuncion{Siguiente}{\In{it}{itDiccNat($\alpha$)}}{tupla(nat,$\alpha$)}
[haySiguiente?(it)]
{alias(res $=_{obs}$ Siguiente($it$))}
[$\Theta(1)$]
[devuelve el elemento siguiente del iterador.]
[res.significado es modificable si y sólo si it es modificable. En cambio, res.clave no es modificable.]

~

\InterfazFuncion{SiguienteSignificado}{\In{it}{itDiccNat($\alpha$)}}{$\alpha$}
[haySiguiente?(it)]
{alias(res $=_{obs}$ Siguiente($it$).significado)}
[$\Theta(1)$]
[devuelve el significado del elemento siguiente del iterador]
[res es modificable si y sólo si it es modficable.]

~

\InterfazFuncion{Avanzar}{\Inout{it}{itDiccNat($\alpha$)}}{}
[it $=$ it$_0$ $\land$ haySiguiente?(it)]
{it $=_{obs}$ Avanzar(it$_0$)}
[$\Theta(1)$]
[avanza a la posicion siguiente del iterador.]
[]

~

\subsubsection{Especificación de las operaciones auxiliares utilizadas en la interfaz}

\begin{tad}{\tadNombre{DiccNat($\alpha$) extendido}}
	\tadExtiende{Diccionario(nat,$\alpha$)}
	\textbf{otras operaciones (no exportadas)}

	\tadOperacion{esPermutacion?}{secu(tupla(nat,$\alpha$)), diccNat($\alpha$)}{bool}{}
	\tadOperacion{secuADiccNat}{secu(tupla(nat,$\alpha$))}{diccNat($\alpha$)}{}

	\textbf{axiomas}

	\tadAxioma{esPermutacion?(s,d)}{d = secuADiccNat $\land$ $\#$claves = long(s)}
	\tadAxioma{secuADiccNat(s)}{\IF vacia?(s) THEN vacio ELSE definir(prim(s).clave, prim(s).significado, secuADiccNat(fin(s))) FI }
\end{tad}

\pagebreak

\subsection{Representación de DiccNat($\alpha$)}

\begin{Estructura}{DiccNat}[estr]
	\begin{Tupla}[estr]
		\tupItem{tabla}{vector(lista(tupla(clave : nat, significado : $\alpha$)))}
		\tupItem{\\ listaIterable}{lista(puntero(tupla(clave : nat, significado : $\alpha$)))}
	\end{Tupla}
\end{Estructura}

\begin{enumerate}
	\item No existe dos veces el mismo nat $n$ en dos posiciones distintas del vector y ese nat va a estar en la posicion $n$ mod longitud(tabla)
	\item La suma del largo de todas las listas enlazadas que salen del vector, tiene que ser igual al largo del vector.
	\item Toda tupla de la tabla es apuntado por un elemento de listaIterable.
	\item El largo de listaIterable es igual al largo del vector.
\end{enumerate}

\subsubsection{Invariante de Representación}

\renewcommand{\labelenumi}{(\Roman{enumi})}

\Rep[estr][e]{
	\\
	($\forall$ $l$ : lista(tupla(nat, $\alpha$))) (( $\exists$ $k$ : nat ) ((e.tabla[k] = l)
	  \\ $\Rightarrow$ ($\not \exists$ $q$ : nat) (k $\neq$ q $\land$ e.tabla[q] = l)) $\land$ ($\forall$ $t1$, $t2$ : tupla(nat, $\alpha$)) (esta?($t1$,l) $\land$ esta?($t2$,l)
		\\ $\Rightarrow$ ($\forall$ n,m : nat) ($t1$.clave = n $\land$ $t2$.clave = m
		\\ $\Rightarrow$ n $\neq$ m $\land$ (n mod longitud(e.tabla) = k $\land$ m mod longitud(e.tabla) = k) ) ))
	$\\ \land \\$
	largosDeListas(e.tabla) $=$ longitud(e.tabla)
	$\\ \land \\$
	($\forall$ l : lista(tupla(nat, $\alpha$)))(esta?(l, e.tabla)
		\\ $\Rightarrow$ ($\forall$ t : tupla(nat, $\alpha$))(esta?(t, l)
		\\ $\Rightarrow$ ($\exists$ p : puntero($\alpha$)($\&$t $=$ p $\land$ esta?(p, e.listaIterable)))))
	$\\ \land \\$
	long(e.tabla) $=_{obs}$ long(e.listaIterable)
}\mbox{}

\tadOperacion{largosDeListas}{secu(secu(tupla(nat,$\alpha$)))}{nat}{}

~

\tadAxioma{largoDeListas($vector$)}{
	\IF vacía?($vector$)) THEN
		0
	ELSE
		longitud(prim($vector$)) + largoDeLista(fin($vector$))
	FI
}

\subsubsection{Funci\'on de Abstracci\'on}

\Abs[estr]{Diccionario}[e]{dicc}{
	($\forall$ $l$ : lista(tupla(nat, $\alpha$))) ((esta?($l$,e.tabla) $\\$
	\- \- $\Rightarrow$ ($\forall$ $t$ : tupla(nat, $\alpha$)) esta?($t$,l) $\\$
	\- $\Rightarrow_{L}$ ($\forall$ $n$ : nat) $n =_{obs} t$.clave $\Leftrightarrow$ $\\$ def?($n$, dicc) $\land_L$ obtener($n$, dicc) $=_{obs}$ $t$.significado))
}

\subsection{Representaci\'on del iterador de DiccNat}

\begin{Estructura}{itDiccNat($\alpha$)}[itDiccNat donde itDiccNat es $\newline$ \- \- \- \- itLista(tupla(nat,$\alpha$))]

\end{Estructura}

\subsection{Algoritmos}

\lstset{style=alg}

\begin{lstlisting}[mathescape]
'\alg{icrearDiccionario}{\In{v}{vector(tupla(clave : nat, significado :  $\alpha$))}}{estr}'
{$\textbf{Pre}$ $\equiv$ true}
		nat : i $\leftarrow$ 0 '\ote{1}'
		while i < longitud(v) do '\ote{1}'
			AgregarAtras(res.tabla, vacia()) '\ote{1}'
		end while '\ote{1}'
		i $\leftarrow$ 0 '\ote{1}'
		while i < longitud(v) do '\ote{1}'
			nat : k $\leftarrow$ v[i].clave mod longitud(v) '\ote{1}'
			AgregarAtras(res.tabla[k], v[i]) '\ote{copy($\alpha$)}'
			nat : q $\leftarrow$ longitud(res.tabla[k]) '\ote{longitud(res.tabla[k])}'
			AgregarAtras(res.listaIterable, puntero(res.tabla[k][q-1])) '\ote{q}'
			i++ '\ote{1}'
		end while '\ote{longitud(v) * (copy($\alpha$) + q)}'
{$\textbf{Post}$ $\equiv$ (($\forall$ $t$ : tupla(nat, $\alpha$)) esta?($t$, v)) $\Rightarrow$ ((definido?(t.clave, res))
 $\land_L$ obtener(t.clave,res) $=_{obs}$ t.significado) $\land$ cantClaves(res) $=_{obs}$ longitud(v)}
'\ofi{O(longitud(v) * (copy(\alpha) + q))} donde q es la cantidad de elementos que pueden encontrarse en una misma posicion de la tabla'
$\textbf{Justificacion:}$ las veces que se va a realizar la operacion de copia
	del elemento $\alpha$ es equivalente al largo del vector, ya que va a hacer
	una vez cada una. El valor de $q$ va a ser 1 en caso promedio gracias a la funcion
	de Hash, pero en el peor caso va a ser igual a la longitud del vector de entrada.
\end{lstlisting}

\begin{lstlisting}[mathescape]
'\alg{iredefinir}{\Inout{d}{estr}, \In{n}{nat}, \In{a}{$\alpha$}}{}'
{$\textbf{Pre}$ $\equiv$ definido?(n,d)}
	nat : k $\leftarrow$ n mod longitud(d) '\ote{1}'
	itLista($\alpha$) : it $\leftarrow$ crearIt(d.tabla[k]) '\ote{1}'
	while haySiguiente?(it) do '\ote{1}'
		IF siguiente(it).clave = n THEN '\ote{1}'
			it.significado $\leftarrow$ a '\ote{1}'
		FI
	end while '\ote{longitud(d.tabla[k])}'
{$\textbf{Post}$ $\equiv$ obtener(n,d) $=_{obs}$ a}
'\ofi{O(i)} siendo i = 1 en caso promedio y i = $\#$claves(d) en peor caso'
// Preguntar como renombrar longitud(d.tabla[k]) a i, para que la complejidad quede
// prolija
$\textbf{Justificacion:}$ En el peor caso, van a estar todos los elementos
	en la misma posicion de la tabla y el elemento que queremos redefinir
	va a estar en la ultima posicion. Ademas, cada acceso a una posicion de
	una lista, es O(i) siendo i la posicion y vamos a recorrer todos los elementos
	de la tabla (o sea de la lista) hasta encontrar el que queremos. Entonces,
	en el peor caso, i va a ser igual al numero de claves del diccionario,
	pero en caso promedio, i va a ser 1 gracias a la funcion de hash.
\end{lstlisting}

\begin{lstlisting}[mathescape]
'\alg{iobtener}{\In{n}{nat}, \In{d}{estr}}{puntero($\alpha$)}'
{$\textbf{Post}$ $\equiv$ true}
	nat : k $\leftarrow$ n mod longitud(d) '\ote{1}'
	itLista($\alpha$) : it $\leftarrow$ crearIt(d.tabla[k]) '\ote{1}'
	res $\leftarrow$ NULL '\ote{1}'
	while haySiguiente?(it) do '\ote{1}'
		IF Siguiente(it).clave = n THEN '\ote{1}'
			res $\leftarrow$ it.significado '\ote{1}'
		FI
	end for '\ote{i}'
{$\textbf{Post}$ $\equiv$ res $=_{obs}$ obtener(n,d)}
'\ofi{O(i)} siendo i = 1 en caso promedio y i = $\#$claves(d) en peor caso'
$\textbf{Justificacion:}$ Misma justificacion que para redefinir(d,n,a)
\end{lstlisting}

\begin{lstlisting}[mathescape]
'\alg{idefinido?}{\In{n}{nat}, \In{d}{estr}}{bool}'
{$\textbf{Pre}$ $\equiv$ true}
	nat : k $\leftarrow$ n mod longitud(d) '\ote{1}'
	itLista($\alpha$) : it $\leftarrow$ crearIt(d.tabla[k]) '\ote{1}'
	res $\leftarrow$ false '\ote{1}'
	while haySiguiente?(it) do '\ote{1}'
		IF Siguiente(it).clave $=$ n THEN '\ote{i}'
			res $\leftarrow$ true '\ote{1}'
		FI
	end while '\ote{i}'
{$\textbf{Post}$ $\equiv$ res $=_{obs}$ def?(n,d)}
'\ofi{O(i)} siendo i = 1 en caso promedio y i = $\#$claves(d) en peor caso'
$\textbf{Justificacion:}$ Misma justificacion que para redefinir(d,n,a)
\end{lstlisting}

\begin{lstlisting}[mathescape]
'\alg{icantClaves}{\In{d}{estr}}{nat}'
{$\textbf{Pre}$ $\equiv$ true}
	res $\leftarrow$ longitud(d.tabla) '\ote{1}'
{$\textbf{Post}$ $\equiv$ res $=_{obs}$ $\#$claves(d)}
'\ofi{O(1)}'
$\textbf{Justificacion:}$ Saber la longitud de un vector es O(1) y por el InvRep,
	sabemos que el largo del vector representa la cantidad de claves del dicc.
\end{lstlisting}

\begin{lstlisting}[mathescape]
'\alg{iCrearIt}{\In{d}{DiccNat($\alpha$)}}{itDiccNat($\alpha$)}'
{$\textbf{Pre}$ $\equiv$ true}
	res $\leftarrow$ crearIt(d.listaIterable) '\ote{1}'
{$\textbf{Post}$ $\equiv$ alias(esPermutacion(SecuSuby(res), d)) $\land$ vacia?(Anteriores(res))}
'\ofi{\Theta (1)}'
{$\textbf{Justificacion:}$ De acuerdo a la interfaz de Lista, crear un itLista($\alpha$) es $\Theta (1)$.
\end{lstlisting}

\begin{lstlisting}[mathescape]
'\alg{iHaySiguiente}{\In{it}{itDiccNat($\alpha$)}}{bool}'
{$\textbf{Pre}$ $\equiv$ true}
	res $\leftarrow$ HaySiguiente(it)
{$\textbf{Pre}$ $\equiv$ res $=_{obs}$ haySiguiente?(it)}
'\ofi{\Theta (1)}'
{$\textbf{Justificacion:}$ De acuerdo a la interfaz de Lista, HaySiguiente(it) es $\Theta (1)$.
\end{lstlisting}

\begin{lstlisting}[mathescape]
'\alg{iSiguiente}{\In{it}{itDiccNat($\alpha$)}}{tupla(nat,$\alpha$)}'
{$\textbf{Pre}$ $\equiv$ HaySiguiente?(it)}
	res $\leftarrow$ Siguiente(it)
{$\textbf{Pre}$ $\equiv$ alias(res $=_{obs}$ Siguiente?(it))}
'\ofi{\Theta (1)}'
{$\textbf{Justificacion:}$ De acuerdo a la interfaz de Lista, Siguiente(it) es $\Theta (1)$.
\end{lstlisting}

\begin{lstlisting}[mathescape]
'\alg{iSiguienteSignificado}{\In{it}{itDiccNat($\alpha$)}}{$\alpha$}'
{$\textbf{Pre}$ $\equiv$ HaySiguiente?(it)}
	res $\leftarrow$ Siguiente(it)
{$\textbf{Pre}$ $\equiv$ alias(res $=_{obs}$ Siguiente(it).significado)}
'\ofi{\Theta (1)}'
{$\textbf{Justificacion:}$ De acuerdo a la interfaz de Lista, Siguiente(it) es $\Theta (1)$.
\end{lstlisting}

\begin{lstlisting}[mathescape]
'\alg{iAvanzar}{\Inout{it}{itDiccNat($\alpha$)}}{}'
{$\textbf{Pre}$ $\equiv$ it $=_{obs}$ it$_0$ $\land$ res $=_{obs}$ haySiguiente?(it)}
	res $\leftarrow$ Avanzar(it)
{$\textbf{Pre}$ $\equiv$ it $=_{obs}$ Avanzar?(it$_0$)}
'\ofi{\Theta (1)}'
{$\textbf{Justificacion:}$ De acuerdo a la interfaz de Lista, Avanzar(it) es $\Theta (1)$.
\end{lstlisting}
