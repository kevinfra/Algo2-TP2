\documentclass[10pt, a4paper, spanish]{article}
\usepackage[paper=a4paper, left=1.5cm, right=1.5cm, bottom=1.5cm, top=3.5cm]{geometry}
\usepackage[spanish]{babel}
\selectlanguage{spanish}
\usepackage[utf8]{inputenc}
\usepackage[T1]{fontenc}
\usepackage{indentfirst}
\usepackage{fancyhdr}
\usepackage{latexsym}
\usepackage{lastpage}
\usepackage{aed2-symb,aed2-itef,aed2-tad,caratula}
\usepackage[colorlinks=true, linkcolor=blue]{hyperref}
\usepackage{calc}
\usepackage{ifthen}
\usepackage{caratula/caratula/caratula}

\usepackage{xspace}
\usepackage{xargs}
\usepackage{algorithm}% http://ctan.org/pkg/algorithms
\usepackage{algpseudocode}% http://ctan.org/pkg/algorithmicx
\usepackage{verbatim}
\usepackage{listings}

% Estilo para Algoritmos
\lstdefinestyle{alg}{tabsize=4, frame=single, escapeinside=\'\', framesep=10pt}

\newcommand{\alg}[3]{\hangindent=\parindent#1 (#2) \ifx#3\empty\else$\rightarrow$ res: #3\fi}
\newcommand\ote[1]{\hspace*{\fill}~\mbox{O(#1)}\penalty -9999 }
\newcommand\ofi[1]{\ensuremath{\textbf{Complejidad}: #1}}

% Afanado
\newcommand{\moduloNombre}[1]{\textbf{#1}}

\let\NombreFuncion=\textsc
\let\TipoVariable=\texttt
\let\ModificadorArgumento=\textbf
\newcommand{\res}{$res$\xspace}
\newcommand{\tab}{\hspace*{7mm}}

\newcommandx{\TipoFuncion}[3]{%
  \NombreFuncion{#1}(#2) \ifx#3\empty\else $\to$ \res\,: \TipoVariable{#3}\fi%
}
\newcommand{\In}[2]{\ModificadorArgumento{in} \ensuremath{#1}\,: \TipoVariable{#2}\xspace}
\newcommand{\Out}[2]{\ModificadorArgumento{out} \ensuremath{#1}\,: \TipoVariable{#2}\xspace}
\newcommand{\Inout}[2]{\ModificadorArgumento{in/out} \ensuremath{#1}\,: \TipoVariable{#2}\xspace}
\newcommand{\Aplicar}[2]{\NombreFuncion{#1}(#2)}

\newlength{\IntFuncionLengthA}
\newlength{\IntFuncionLengthB}
\newlength{\IntFuncionLengthC}
%InterfazFuncion(nombre, argumentos, valor retorno, precondicion, postcondicion, complejidad, descripcion, aliasing)
\newcommandx{\InterfazFuncion}[9][4=true,6,7,8,9]{%
  \hangindent=\parindent
  \TipoFuncion{#1}{#2}{#3}\\%
  \textbf{Pre} $\equiv$ \{#4\}\\%
  \textbf{Post} $\equiv$ \{#5\}%
  \ifx#6\empty\else\\\textbf{Complejidad:} #6\fi%
  \ifx#7\empty\else\\\textbf{Descripción:} #7\fi%
  \ifx#8\empty\else\\\textbf{Aliasing:} #8\fi%
  \ifx#9\empty\else\\\textbf{Requiere:} #9\fi%
}

\newenvironment{Interfaz}{%
  \parskip=2ex%
  \noindent\textbf{\Large Interfaz}%
  \par%
}{}

\newenvironment{Representacion}{%
  \vspace*{2ex}%
  \noindent\textbf{\Large Representación}%
  \vspace*{2ex}%
}{}

\newenvironment{Algoritmos}{%
  \vspace*{2ex}%
  \noindent\textbf{\Large Algoritmos}%
  \vspace*{2ex}%
}{}


\newcommand{\Titulo}[1]{
  \vspace*{1ex}\par\noindent\textbf{\large #1}\par
}

\newenvironmentx{Estructura}[2][2={estr}]{%
  \par\vspace*{2ex}%
  \TipoVariable{#1} \textbf{se representa con} \TipoVariable{#2}%
  \par\vspace*{1ex}%
}{%
  \par\vspace*{2ex}%
}%

\newboolean{EstructuraHayItems}
\newlength{\lenTupla}
\newenvironmentx{Tupla}[1][1={estr}]{%
    \settowidth{\lenTupla}{\hspace*{3mm}donde \TipoVariable{#1} es \TipoVariable{tupla}$($}%
    \addtolength{\lenTupla}{\parindent}%
    \hspace*{3mm}donde \TipoVariable{#1} es \TipoVariable{tupla}$($%
    \begin{minipage}[t]{\linewidth-\lenTupla}%
    \setboolean{EstructuraHayItems}{false}%
}{%
    $)$%
    \end{minipage}
}

\newcommandx{\tupItem}[3][1={\ }]{%
    %\hspace*{3mm}%
    \ifthenelse{\boolean{EstructuraHayItems}}{%
        ,#1%
    }{}%
    \emph{#2}: \TipoVariable{#3}%
    \setboolean{EstructuraHayItems}{true}%
}

\newcommandx{\RepFc}[3][1={estr},2={e}]{%
  \tadOperacion{Rep}{#1}{bool}{}%
  \tadAxioma{Rep($#2$)}{#3}%
}%

\newcommandx{\Rep}[3][1={estr},2={e}]{%
  \tadOperacion{Rep}{#1}{bool}{}%
  \tadAxioma{Rep($#2$)}{true \ssi #3}%
}%

\newcommandx{\Abs}[5][1={estr},3={e}]{%
  \tadOperacion{Abs}{#1/#3}{#2}{Rep($#3$)}%
  \settominwidth{\hangindent}{Abs($#3$) \igobs #4: #2 $\mid$ }%
  \addtolength{\hangindent}{\parindent}%
  Abs($#3$) \igobs #4: #2 $\mid$ #5%
}%

\newcommandx{\AbsFc}[4][1={estr},3={e}]{%
  \tadOperacion{Abs}{#1/#3}{#2}{Rep($#3$)}%
  \tadAxioma{Abs($#3$)}{#4}%
}%


%FIN Afanado

\newcommand{\f}[1]{\text{#1}}
\renewcommand{\paratodo}[2]{$\forall~#2$: #1}

\sloppy

\hypersetup{%
 % Para que el PDF se abra a página completa.
 pdfstartview= {FitH \hypercalcbp{\paperheight-\topmargin-1in-\headheight}},
 pdfauthor={Grupo 1 - Algoritmos y Estructuras de Datos II},
 pdfkeywords={Trabajo Practico 2},
 pdfsubject={Diseño}
}

\parskip=5pt % 5pt es el tamaño de fuente

% Pongo en 0 la distancia extra entre ítemes.
\let\olditemize\itemize
\def\itemize{\olditemize\itemsep=0pt}

% Acomodo fancyhdr.
\pagestyle{fancy}
\thispagestyle{fancy}
\addtolength{\headheight}{1pt}
\lhead{Algoritmos y Estructuras de Datos II}
\rhead{$2^{\mathrm{do}}$ cuatrimestre de 2015}
\cfoot{\thepage /\pageref{LastPage}}
\renewcommand{\footrulewidth}{0.4pt}

\author{Grupo 2 - Algoritmos y Estructuras de Datos II}
\date{2do cuatrimestre 2015}
\title{Trabajo Practico 2}

\begin{document}

%caratula
\titulo{Trabajo Pr\'actico 2}
\fecha{06 de Septiembre de 2015}
\materia{Algoritmos y Estructuras de Datos II}
\subtitulo{Grupo N\'umero 1}

\integrante{Joel Esteban Camera}{257/14}{joel.e.camera@gmail.com}
\integrante{Manuel Mena}{313/14}{manuelmena1993@gmail.com}
\integrante{Kevin Frachtenberg Goldsmit}{247/14}{kevinfra94@gmail.com}
\integrante{Nicol\'as Bukovits}{546/14}{nicobuk@gmail.com}

\maketitle

%Indice
\tableofcontents

\pagebreak
\section{Módulo CampusSeguro}

\subsection{Interfaz}

\textbf{se explica con}: \tadNombre{CampusSeguro}.

\textbf{géneros}: \TipoVariable{campusSeguro}.

\subsubsection{Operaciones básicas de CampusSeguro}

\InterfazFuncion{comenzarRastrillaje}{\In{c}{campus}, \In{d}{dicc(placa,AS)}}{campusSeguro}
[($\forall a$ : agente) (def?($a$,$d$) $\Rightarrow_L$ (posVálida(obtener($a$,$d$)) $\land$ $\neg$ocupada?(obtener($a$,$d$,$c$))) $\land$ ($\forall a, a2$ : agente) ((def?($a$,$d$) $\land$ def?($a2$,$d$) $\land a \neq a2$) $\Rightarrow_L$ obtener($a$,$d$) $\neq$ obtener($a2$,$d$))]
{res $=_{obs}$ comenzarRastrillaje(c,d)}
[$O((f*c)^2 + N_a)$]
[Crea un nuevo campusSeguro tomando un campus y un diccionario con agentes.]
[]

~

\InterfazFuncion{ingresarEstudiante}{\In{e}{nombre}, \In{p}{posición}, \Inout{cs}{campusSeguro}}{}
[$cs =_{obs} cs_o \land e \not\in$ (estudiantes($cs$) $\bigcup$ hippies($cs$)) $\land$ esIngreso?($p$,campus($cs$)) $\land$ $\neg$estaOcupada?($p$,$cs$)]
{$cs =_{obs}$ ingresarEstudiante($e$,$p$,$cs_o$)}
[$O(|n_m|) + O(log(N_a))$]
[Ingresa un nuevo estudiante al campusSeguro]
[]

~

\InterfazFuncion{ingresarHippie}{\In{h}{nombre}, \In{p}{posición}, \Inout{cs}{campusSeguro}}{}
[$cs =_{obs} cs_o \land h \not\in$ (estudiantes($cs$) $\bigcup$ hippies($cs$)) $\land$ esIngreso?($p$,campus($cs$)) $\land$ $\neg$estaOcupada?($p$,$cs$)]
{$cs =_{obs}$ ingresarHippie($h$,$p$,$cs_o$)}
[$O(|n_m|)$]
[Ingresa un nuevo hippie el campusSeguro]
[]

~

\InterfazFuncion{moverEstudiante}{\In{e}{nombre}, \In{dir}{dirección}, \Inout{cs}{campusSeguro}}{}
[$cs =_{obs} cs_o \land e \in$ estudiantes($cs$) $\land$ (seRetira($e$,$dir$,$cs$) $\lor$ \\
(posValida(proxPosicion(posEstudianteYHippie($e$,$cs$),$dir$,campus($cs$)), campus($cs$)) $\land$ $\neg$estaOcupada?(proxPosicion(posEstudianteYHippie($e$,$cs$),$dir$,campus($cs$)),$cs$)]
{$cs =_{obs}$ moverEstudiante($e$,$dir$,$cs_o$)}
[$O(|n_m|)$]
[Mueve un estudiante dentro del campus o lo hace salir y se actualizan los atrapados, sanciones (si es que las hay) y hippies atrapados (si es que los hay).]
[]

~

\InterfazFuncion{moverHippie}{\In{h}{nombre}, \In{d}{direccion}, \Inout{cs}{campusSeguro}}{}
[$cs =_{obs} cs_o \land h \in$ hippies($cs$) $\land$ $\neg$todasOcupadas?(vecinos(posEstudianteYHippie($h$,$cs$),campus($cs$)), $cs$)]
{$cs =_{obs}$ moverHippie($h$,$d$,$cs_o$}
[$O(|n_m|) + O(N_e)$]
[Mueve un hippie dentro del campus y se actualizan los atrapados, sanciones(si es que las hay) y hippies atrapados (si es que los hay).]
[]

~

\InterfazFuncion{moverAgente}{\In{a}{agente}, \Inout{cs}{campusSeguro}}{}
[$cs =_{obs} cs_o \land a \in$ agentes($cs$) $\land_L$ cantSanciones($a$,$cs$) $\leq 3 \land$ \\
$\neg$todasOcupadas?(vecinos(posEstudianteYHippie($h$,$cs$),campus($cs$)),$cs$)]
{$cs =_{obs}$ moverAgente($a$,$cs_o$)}
[$O(|n_m|) + O(log(N_a)) + O(N_h)$]
[Mueve un agente dentro del campus y se actualizan los atrapados, sanciones (si es que las hay) y hippies atrapados (si es que los hay).]
[]

~

\InterfazFuncion{campus}{\In{cs}{campusSeguro}}{campus}
[true]
{$res =_{obs}$ campus($cs$)}
[$O(1)$]
[Devuelve el campus del campusSeguro.]
[$res$ es una referencia no modificable.]

~

\InterfazFuncion{estudiantes}{\In{cs}{campusSeguro}}{conj(nombre)}
[true]
{$res =_{obs}$ estudiantes($cs$)}
[$O(1)$]
[Devuelve el conjunto de los estudiantes que estan en el campus.]
[$res$ es una referencia no modificable.]

~

\InterfazFuncion{hippies}{\In{cs}{campusSeguro}}{conj(nombre)}
[true]
{$res =_{obs}$ hippies($cs$)}
[$O(1)$]
[Devuelve el conjunto de los hippies que estan en el campus.]
[$res$ es una referencia no modificable.]

~

\InterfazFuncion{agentes}{\In{cs}{campusSeguro}}{conj(agentes)}
[true]
{$res =_{obs}$ agentes($cs$)}
[$O(1)$]
[Devuelve el conjunto de los agentes que estan en el campus.]
[$res$ es una referencia no modificable.]

~

\InterfazFuncion{posEstudianteYHippie}{\In{id}{nombre}, \In{cs}{campusSeguro}}{posición}
[$id \in$ (estudiantes($cs$) $\bigcup$ hippies($cs$)]
{$res =_{obs}$ posEstudianteYHippie($id$,$cs$)}
[$O(|n_m|)$, donde $|m_n|$ es la longitud mas larga entre todos los nombres.]
[Devuelve la posicion del estudiante o hippie.]
[]

~

\InterfazFuncion{posAgente}{\In{a}{agente}, \In{cs}{campusSeguro}}{posición}
[$a \in$ agentes($cs$)]
{$res =_{obs}$ posAgente($a$,$cs$)}
[$O(1)$ en caso promedio.]
[Devuelve la posicion del agente pasado como parametro.]
[]

~

\InterfazFuncion{cantSanciones}{\In{a}{agente}, \In{cs}{campusSeguro}}{nat}
[$a \in$ agentes($cs$)]
{$res =_{obs}$ cantSanciones($a$,$cs$)}
[$O(1)$ en caso promedio.]
[Devuelve la cantidad de sanciones que posee el agente pasado como parametro.]
[]

~

\InterfazFuncion{cantHippiesAtrapados}{\In{a}{agente}, \In{cs}{campusSeguro}}{nat}
[$a \in$ agentes($cs$)]
{$res =_{obs}$ cantHippiesAtrapados($a$,$cs$)}
[$O(1)$ en caso promedio.]
[Devuelve la cantidad de hippies que atrapo el agente pasado como parametro.]
[]

~

\InterfazFuncion{másVigilante}{\In{cs}{campusSeguro}}{agente}
[true]
{$res =_{obs}$ másVigilante($cs$)}
[$O(1)$]
[Devuelve la placa del agente que ha atrapado mas hippies.]
[]

~

\InterfazFuncion{conMismasSanciones}{\In{a}{agente}, \In{cs}{campusSeguro}}{conj(agentes)}
[$a \in$ agentes($cs$)]
{$res =_{obs}$ conMismasSanciones($a$,$cs$)}
[$O(1)$]
[Devuelve el conjunto de los agentes que tienen el mismo numero de sanciones que el agente pasado como parametro.]
[$res$ es una referencia no modificable.]

~

\InterfazFuncion{conKSanciones}{\In{k}{nat}, \In{cs}{campusSeguro}}{conj(agentes)}
[true]
{$res =_{obs}$ conKSanciones($k$,$cs$)}
[$O(N_a)$ la primera vez que se la llama y $O(log(N_a))$ en futuras llamadas mientras no ocurran sanciones.]
[Devuelve el conjuto de agenes que tienen k sanciones.]
[$res$ es una referencia no modificable.]

~
\pagebreak

\subsubsection{Representación de campusSeguro}

\begin{Estructura}{campusSeguro}[estr]
	\begin{Tupla}[estr]
		\tupItem{campus}{campus}%
		\tupItem{\\ personalAS}{diccNat(agente,datosAgente)}%
		\tupItem{\\ posicionesAgente}{vector(As)}%
		\tupItem{\\ masVigilante}{As}%
		\tupItem{\\ listaMismasSanc}{lista(kSanc)}%
		\tupItem{\\ arregloMismasSanc}{arreglo(puntero(kSanc))}
		\tupItem{\\ mismasSancModificado}{bool}
		\tupItem{\\ hippies}{dicString(nombre,posicion)}%
		\tupItem{\\ estudiantes}{dicString(nombre, posicion)}%
		\tupItem{\\ posicionesHippies}{vector(nombre)}
		\tupItem{\\ posicionesEstudiantes}{vector(nombre)}
	\end{Tupla}

	~

	\begin{Tupla}[datosAgente]
		\tupItem{posicion}{posicion}%
		\tupItem{\\ cantSanc}{nat}%
		\tupItem{\\ cantAtrapados}{nat}%
		\tupItem{\\ itMismasSanc}{itLista(kSanc)}%
		\tupItem{\\ itConjMismasSanc}{itConj(agente)}%
	\end{Tupla}

	~

	\begin{Tupla}[As]
		\tupItem{agente}{nat}%
		\tupItem{datos}{puntero(datosAgente)}%
	\end{Tupla}

	~

	\begin{Tupla}[kSanc]
		\tupItem{sanc}{nat}
		\tupItem{agentes}{conj(agente)}
	\end{Tupla}

\end{Estructura}

\subsubsection{Invariante de Representación}

\renewcommand{\labelenumi}{(\Roman{enumi})}

\begin{enumerate}
	%done
	\item Las posiciones de todos los agentes son posiciones validas del campus.
	%done
	\item Las posiciones de los hippies son posiciones validas del campus.
	%done
	\item Las posiciones de los estudiantes son posiciones validas del campus.
	%done
	\item Las posiciones de los agentes son distintas a las posiciones de los hippies.
	%done
	\item Las posiciones de los estudiantes son distintas a las posiciones de los hippies.
	%done
	\item Las posiciones de los agentes son distintas a las posiciones de los estudiantes.
	%done
	\item El agente masVigilante esta definido en personalAs y tiene la mayor cantidad de sanciones.
	%done
	\item Para todo nombre que esta definido en el diccionario de hippies no puede estar definido en el diccionario de estudiantes y viceversa.
	%done
	\item La cantidad de elementos que tiene posicionesAgente, posicionesHippies y posicionesEstudiantes es la cantidad de coordenadas que tiene la grilla.
	%done
	\item Los agentes de posicionesAgente son los mismos que los de personalAs y viceversa.
	%done
	\item La posicion de todos los agentes de posicionesAgente se mapea con la posicion que tienen en personalAs y viceversa.
	%done
	\item Las posiciones en posicionesAgente que no tienen agente tienen un puntero a NULL.
	%done
	\item La dimension del campus es mayor que la cantidad de estudiantes, hippies, agentes y obstaculos .
	%done
	\item La longitud de listaMismasSanc es la cantidad de sanciones diferentes que hay.
	\item Para cada cantidad de sanciones distinta hay un nodo en listaMismasSanc con esa cantidad de sanciones.
	\item listaMismasSanc esta ordenada por cantidad de sanciones.
	\item En cada nodo de listaMismasSanc va a estar el conjunto de todos los agentes que tienen la cantidad de sanciones indicada en el nodo.
	\item Si mismasSancModificado es falso, vectorMismasSanc tiene misma cantidad de elementos que listaMismasSanc, y cada elemento apunta a un nodo de listaMismasSanc, respetando el orden de listaMismasSanc
	\item La posiciones de todos los hippies en posicionesHippies se mapea con la posicion que tienen en el diccString hippies y viceversa
	\item La posiciones de todos los estudiantes en posicionesEstudiantes se mapea con la posicion que tienen en el diccString estudiantes y viceversa
\end{enumerate}

\Rep[estr][e]{
	\\
	($\forall a$: nat)(def?($a$, $e$.personalAS) $\Rightarrow_L$ posValida(Obtener($a$,$e$.personalAS).posicion, e.campus) $\land$ $\neg$ocupada?(Obtener($a$,$e$.personalAS).posicion),e.campus)
	\\
	$\land$
	\\
	($\forall h$: string)(def?($h$, $e$.hippies) $\Rightarrow_L$ posValida(Obtener($h$,$e$.hippies), e.campus) $\land$ $\neg$ocupada?(Obtener($h$, $e$.hippies)),e.campus)
	\\
	$\land$
	\\
	($\forall est$: string)(def?($est$, $e$.estudiantes) $\Rightarrow_L$ posValida(Obtener($est$, $e$.estudiantes), e.campus) $\land$ $\neg$ocupada?(Obtener($est$, $e$.estudiantes)), e.campus)
	\\
	$\land$
	\\
	($\forall a$: nat)($\forall h$: string) def?($a$, $e$.personalAS) $\land$ def?($h$, $e$.hippies) $\Rightarrow_L$ Obtener($a$, $e$.personalAS).posicion $\neq$ Obtener($h$, $e$.hippies)
	\\
	$\land$
	\\
	($\forall est$: string)($\forall h$: string) def?($est$,$e$.estudiantes) $\land$ def?($h$,$e$.hippies) $\Rightarrow_L$ Obtener($est$,$e$.estudiantes) $\neq$ Obtener($h$, $e$.hippies)
	\\
	$\land$
	\\
	($\forall a$: nat)($\forall est$: string) def?($a$, $e$.personalAS) $\land$ def?($est$, $e$.estudiantes) $\Rightarrow_L$ Obtener($a$, $e$.personalAS).posicion $\neq$ Obtener($est$, $e$.estudiantes)
	\\
	$\land$
	\\
	def?(masVigilante.agente, $e$.personalAS) $\land_L$ Obtener(masVigilante.agente, $e$.personalAS) = *masVigilante.datos
	\\
	$\land$
	\\
	($\forall a$: nat) def?($a$, $e$.personalAS) $\Rightarrow_L$ Obtener($a$, $e$.personalAS).cantSanc $\leq$ *masVigilante.datos.cantSanc
	\\
	$\land$
	\\
	($\forall h$: string) def?($h$, $e$.hippies) $\Rightarrow$ $\neg$def?($h$, $e$.estudiantes)
	\\
	$\land$
	\\
	($\forall est$: string) def?($est$, $e$.estudiantes) $\Rightarrow$ $\neg$def?($est$, $e$.hippies)
	\\
	$\land$
	\\
	longitud($e$.posicionesAgente) = filas($e$.campus)$*$columnas($e$.campus)
	\\
	$\land$
	\\
	longitud($e$.posicionesHippies) = filas($e$.campus)$*$columnas($e$.campus)
	\\
	$\land$
	\\
	longitud($e$.posicionesEstudiantes) = filas($e$.campus)$*$columnas($e$.campus)
	\\
	$\land$
	\\
	($\forall a$: nat) def?($a$, $e$.personalAS) $\Leftrightarrow_L$ ($\exists i$: nat) $i <$ longitud($e$.posicionesAgente) $\land_L$ $e$.posicionesAgente[$i$].agente = $a$ $\land$ *$e$.posicionesAgente[$i$].datos = Obtener($a$, $e$.personalAS) $\land$ ($i$ $/$ columnas($e$.campus)) = Obtener($a$, $e$.personalAS).posicion.y $\land$ ($i$ $mod$ columnas($e$.campus)) = Obtener($a$, $e$.personalAS).posicion.x
	\\
	$\land$
	\\
	($\forall i$: nat) $i <$ longitud($e$.posicionesAgente) $\Leftrightarrow_L$ *$e$.posicionesAgente[$i$].datos = NULL $\lor$ ($\exists! a$: nat) def?($a$, $e$.personalAS) $\land_L$ *$e$.posicionesAgente[$i$].datos = Obtener($a$, $e$.personalAS)
	\\
	$\land$
	\\
	$\#$claves($e$.personalAS) $+$ $\#$claves($e$.hippies) $+$ $\#$claves($e$.estudiantes) $+$ $\#$Obstaculos($e$.campus) $\leq$ filas($e$.campus)*columnas($e$.campus)

}\mbox{}
\\
	$\land$
	\\
	long($e$.listaMismasSanc) = $\#$SancionesDistintas($e$.personalAS)
	\\
	$\land$
	\\
	($\forall$ sanc : nat)(sanc $\in$ conjSanciones(claves($e$.personalAS), $e$.personalAS) $\Rightarrow$ ($\exists$ nodo : kSanc)(esta?(kSanc, $e$.listaMismasSanc) $\land$ kSanc.sanc = sanc))
	\\
	$\land$
	\\
	ordenada?($e$.listaMismasSanc)
	\\
	$\land$
	\\
	($\forall$ nodo : kSanc)(esta?(kSanc, $e$.listaMismasSanc) $\Rightarrow$ agentes(kSanc) $=_{obs}$ agentesKSanc($e$.personalAS, claves($e$.personalAS), sanc(kSanc)))
	\\
	$\land$
	\\
	$e$.mismasSancModificado $\Rightarrow$ (long($e$.listaMismasSanc) = long(vectorMismasSanc) $\land_L$ mismosNodos($e$.listaMismasSanc, $e$.vectorMismasSanc))
	\\
	$\land$
	\\
	($\forall$ h : string)(def?(h, $e$.hippies) $\Leftrightarrow_L$ ($\exists$ i : nat)(i < longitud($e$.posicionesHippies) $\land_L$ $e$.posicionesHippies[i] = obtener(h, $e$.hippies) $\land$ (i / columnas($e$.campus) = obtener(j, $e$.hippies).y) $\land$ (i mod columnas($e$.campus) = obtener(j, $e$.hippies).y)))
	\\
	$\land$
	\\
	($\forall$ h : string)(def?(h, $e$.estudiantes) $\Leftrightarrow_L$ ($\exists$ i : nat)(i < longitud($e$.posicionesEstudiantes) $\land_L$ $e$.posicionesEstudiantes[i] = obtener(h, $e$.estudiantes) $\land$ (i / columnas($e$.campus) = obtener(j, $e$.estudiantes).y) $\land$ (i mod columnas($e$.campus) = obtener(j, $e$.estudiantes).y)))
	\\
	\\
\tadOperacion{$\#$Obstaculos}{campus}{nat}{}
\tadOperacion{$\#$ObstaculosAux}{nat,nat,campus}{nat}{}
\tadOperacion{$\#$SancionesDistintas}{diccNat(agente, datosAgente)}{nat}{}
\tadOperacion{conjSanciones}{conj(nat), diccNat(agente,datosAgente)}{conj(nat)}{}
\tadOperacion{cantAgentes}{secu(kSanc)}{nat}{}
\tadOperacion{ordenada?}{secu(kSanc)}{bool}{}
\tadOperacion{agentesKSanc}{diccNat(agente, datosAgente), conj(agente), nat}{conj(agente)}{}
\tadOperacion{mismosNodos}{secu(kSanc)/lista, secu(puntero(kSanc))/vec}{bool}{long(lista) = long(vec)}

~

\tadAxioma{$\#$Obstaculos(c)}{
	$\#$ObstaculosAux(filas($c$)$-1$, columnas($c$)$-1$, $c$)
}
\tadAxioma{$\#$ObstaculosAux(f,col,c)}{
	\IF ($f=0 \land col=0$) THEN	$\beta$(ocupada?(<$f$,$col$>, $c$)
	ELSE{
		\IF ($f \neq 0 \land col = 0$) THEN $\#$ObstaculosAux($f-1$, columnas($c$)$-1$) $+$ $\beta$(ocupada?(<$f$,$col$>, $c$)
		ELSE{
			\IF ($f = 0 \land col \neq 0$) THEN $\#$ObstaculosAux(filas($c$),$col-1$,$c$) $+$ $\beta$(ocupada?(<$f$,$col$>, $c$)
			ELSE{
				$\#$ObstaculosAux($f-1$, $col-1$, $c$) $+$ $\beta$(ocupada?(<$f$,$col$>, $c$)
			}
			FI
		}
		FI
	}
	FI
}
\tadAxioma{$\#$SancionesDistintas(d)}{
	$\#$conjSanciones(Claves($d$),$d$)
}
\tadAxioma{conjSanciones(c,d)}{
	\IF $\neg\emptyset$($c$) THEN $\emptyset$
	ELSE{
		Ag(Obtener(DameUno($c$),$d$).cantSanc, conjSanciones(SinUno($c$,$d$)))
	}
	FI
}
\tadAxioma{cantAgentes(s)}{
	\IF $s = <>$ THEN 0
	ELSE {
		Long(prim($s$).vectorAgente) + cantAgentes(fin($s$))
	}
	FI
}
\tadAxioma{ordenada?(lista)}{
	\IF vacia?(lista) THEN true
	ELSE{
		\IF vacia?(fin(lista)) THEN true
		ELSE{
			\IF sanc(prim(lista)) < sanc(prim(fin(lista))) THEN ordenada?(fin(lista))
			ELSE false
			FI
		}
		FI
	}
	FI
}
\tadAxioma{agentesKSanc(dicc, claves, k)}{
	\IF $\emptyset$?(claves) THEN $\emptyset$
	ELSE{
		\IF cantSanc(obtener(prim(claves), dicc)) = k THEN
			Ag(prim(claves), agentesKSanc(dicc, fin(claves), k))
		ELSE
			agentesKSanc(dicc, fin(claves), k)
		FI
	}
	FI
}
\tadAxioma{mismosNodos(lista, vec)}{
	\IF vacia?(lista) THEN true
	ELSE{
		\IF $\&prim(lista) = prim(vec)$ THEN
			mismosNodos(fin(lista), fin(vec))
		ELSE
			false
		FI
	}
	FI
}

\pagebreak

\subsubsection{Función de Abstracción}
\Abs[estr]{campusSeguro}[e]{cs}{
$e$.campus $=$ campus($cs$)
\\
$\land$
\\
$e$.estudiantes $=$ estudiantes($cs$)
\\
$\land$
\\
$e$.hippies $=$ hippies($cs$)
\\
$\land$
\\
$e$.personalAS $=$ agentes($cs$)
\\
$\land$
\\
($\forall a$: nat) (def?($a$, $e$.personalAS) $\Rightarrow_L$ posAgente($a$, $cs$) $=$ Obtener($a$, $e$.personalAS).posicion $\land$ cantSanciones($a$, $cs$) $=$ Obtener($a$, $e$.personalAS).cantSanc $\land$cantHippiesAtrapados($a$, $cs$) $=$ Obtener($a$, $e$.personalAS).cantAtrapados)
\\
$\land$
\\
($\forall id$: string) (def?($id$, $e$.estudiantes) $\land_L$ Obtener($id$, $e$.estudiantes) = posEstudianteYHippie(cs)) $\lor$ (def?($id$, $e$.hippies) $\land_L$ Obtener($id$, $e$.hippies) = posEstudianteYHippie(cs))
}

\subsection{Algoritmos}

\lstset{style=alg}

\begin{lstlisting}[mathescape]
'\alg{icomenzarRastrillaje}{\In{c}{campus}, \In{d}{diccNat(agente, datosAgente)}}{estr}'

	res.campus $\leftarrow$ c '\ote{1}'
	res.listaMismasSanc $\leftarrow$ generarListaMismasSanc(d) '\ote{$N_a$}'
	res.personalAS $\leftarrow$ d '\ote{1}'
	res.posicionesAgente $\leftarrow$ vectorizarPos(d,filas(c),columnas(c)) '\ote{(f$*$c)$^2$ + $N_a$}'
	res.masVigilante $\leftarrow$ menorPlaca(d) '\ote{$N_a$}'
	res.mismasSancModificado $\leftarrow$ true '\ote{1}'
	res.hippies $\leftarrow$ Vacio() '\ote{1}'
	res.estudiantes $\leftarrow$ Vacio() '\ote{1}'
	res.posicionesHippies $\leftarrow$ Vacio() '\ote{1}'
	res.posicionesEstudiantes $\leftarrow$ Vacio '\ote{1}'

	nat: i $\leftarrow$ 0
	while i $<$ filas(c)*columnas(c) do '\ote{1}'
		AgregarAtras(res.posicionesHippies, " ")
		AgregarAtras(res.posicionesEstudiantes, " ")
		i $\leftarrow$ i$+1$
	end while '\ote{(f*c)$^2$}'

'\ofi{O(N_a) + O(N_a) + O((f*c)^2 + N_a) + O((f*c)^2) = O((f*c)^2 + N_a)}'
\end{lstlisting}

\begin{lstlisting}[mathescape]
'\alg{vectorizarPos}{\In{d}{diccNat(agente, datosAgente)}, \In{f}{nat}, \In{c}{nat}}{vector(AS)}'
	res $\leftarrow$ Vacia() '\ote{1}'
	nat: i $\leftarrow$ 0 '\ote{1}'
	itDiccNat(agente, datosAgente): it $\leftarrow$ CrearIt(d) '\ote{1}'

	while i $<$ f$*$c do '\ote{1}'
		AgregarAtras(res, tupla(0,NULL)) '\ote{f$*$c}'
		i $\leftarrow$ i$+1$ '\ote{1}'
	end while '\ote{(f$*$c)$^2$}'

	while HaySiguiente(it) do '\ote{1}'
		res[Siguiente(it).significado.posicion.y $*$ c $+$
				Siguiente(it).significado.posicion.x] $\leftarrow$
			$\leftarrow$ tupla(Siguiente(it).clave, puntero(Siguiente(it).significado)) '\ote{1}'
	end while '\ote{$N_a$}'

'\ofi{O((f*c)^2 + N_a)}'
\end{lstlisting}

\begin{lstlisting}[mathescape]
'\alg{menorPlaca}{\In{d}{diccNat(agente, datosAgente)}}{AS}'
	itDiccNat(agente, datosAgente): it $\leftarrow$ CrearIt(d) '\ote{1}'
	nat: placaMenor $\leftarrow$ Siguiente(it).clave '\ote{1}'
	puntero(datosAgente): punt $\leftarrow$ puntero(Siguiente(it).datosAgente '\ote{1}'
	while HaySiguiente(d) do '\ote{1}'
		if Siguiente(it).clave < placaMenor	then '\ote{1}'
			placaMenor $\leftarrow$ Siguiente(it).clave '\ote{1}'
			punt $\leftarrow$ puntero(Siguiente(it).datosAgente) '\ote{1}'
		end if '\ote{1}'
		Avanzar(it) '\ote{1}'
	end while '\ote{$N_a$}'

	res.agente $\leftarrow$ placaMenor '\ote{1}'
	res.datos $\leftarrow$ punt '\ote{1}'

'\ofi{O(N_a)}'
\end{lstlisting}

\begin{lstlisting}[mathescape]
'\alg{generarListaMismasSanc}{\Inout{d}{diccNat(agente, datosAgente)}}{lista(kSanc)}'
	itDiccNat(agente, datosAgente): itDic $\leftarrow$ CrearIt(d) '\ote{1}'
	res $\leftarrow$ Vacia() '\ote{1}'
	AgregarAdelante(res, tupla(0,Vacio())) '\ote{1 (esta vacío el vector)}'
	itLista(kSanc): itL $\leftarrow$ CrearIt(res) '\ote{1}'

	while HaySiguiente(itDic) do '\ote{1}'
		itConj(agente): itC $\leftarrow$ AgregarRapido(res.agentes, Siguiente(itDic).clave) '\ote{1}'
		Siguiente(itDic).significado.itConjMismasSanc $\leftarrow$ itC '\ote{1}'
		Siguiente(itDic).significado.itMismasSanc $\leftarrow$ itL '\ote{1}'
		Avanzar(itDic) '\ote{1}'
	end while '\ote{$N_a$}'

'\ofi{O(N_a)}'
\end{lstlisting}

\begin{lstlisting}[mathescape]
'\alg{iIngresarEstudiante}{\In{e}{nombre}, \In{pos}{posicion}, \Inout{cs}{estr} } {}'
	if todasOcupadas?(vecinos(pos, $cs$.campus), $cs$ AND
	AND AlMenosUnAgente(vecinos(pos, $cs$.campus) then '\ote{1}'
		conj(As): conjAgParaSanc $\leftarrow$ AgParaPremSanc(vecinos(pos, $cs$.campus), $cs$) '\ote{1}'
		SancionarAgentes(conjAgParaSanc, $cs$) '\ote{1}'
	end if '\ote{1}'

	if CantHippiesVecinos(vecinos(pos, $cs$.campus), $cs$)$< 2$  then '\ote{$|n_m|$}'
		Definir($cs$.estudiantes, e, pos) '\ote{$|n_m|$}'
		$cs$.posicionesEstudiantes[pos.y * Columnas($cs$.campus) + pos.x] $\leftarrow$ e '\ote{1}'
	else
		Definir($cs$.hippies, e, pos) '\ote{$|n_m|$}'
		$cs$.posicionesHippies[pos.y * Columnas($cs$.campus) + pos.x] $\leftarrow$ e '\ote{1}'
	end if '\ote{$|n_m| + |n_m|$}'

	conj(nombre, posicion): conjHippiesRodEst $\leftarrow$
		$\leftarrow$ HippiesRodeadosEstudiantes(vecinos(pos, $cs$.campus), $cs$) '\ote{1}'

	if Cardinal(conjHippiesRodEst) $>$ 0 then '\ote{1}'
		itConj(nombre, posicion): itHEst $\leftarrow$ CrearIt(conjHippiesRodEst) '\ote{1}'

		while HaySiguiente(itHEst) do '\ote{1}'
			Definir($cs$.estudiantes, Siguiente(itHEst).nombre,
				Siguiente(itHEst).posicion) '\ote{$|n_m|$}'
			Eliminar($cs$.hippies, Siguiente(itHEst).nombre) '\ote{$|n_m|$}'

			$cs$.posicionesEstudiantes[Siguiente(itHEst).posicion.y*Columnas($cs$.campus) +
			+ Siguiente(itHEst).posicion.x] $\leftarrow$
			$\leftarrow$ $cs$.posicionesHippies[Siguiente(itHEst).posicion.y*Columnas($cs$.campus) +
			+ Siguiente(itHest).posicion.x] '\ote{1}'

			$cs$.posicionesHippies[Siguiente(itHEst).posicion.y*Columnas($cs$.campus) +
			+ Siguiente(itHest).posicion.x] $\leftarrow$ " " '\ote{1}'

			Avanzar(itHEst) '\ote{1}'
		end while '\ote{$2|n_m|$}'
	end if '\ote{$4 * 2|n_m|$}'

	conj(nombre, posicion): conjHippiesRodAs $\leftarrow$
		$\leftarrow$ HippiesRodeadosAs(vecinos(pos, $cs$.campus), $cs$) '\ote{1}'
	if Cardinal(conjHippiesRodAs) $>$ 0 then '\ote{1}'
		itConj(nombre, posicion): itHAs $\leftarrow$ CrearIt(conjHippiesRodAs) '\ote{1}'
		while HaySiguiente(itHAs) do '\ote{1}'
			Eliminar($cs$.hippies, Siguiente(itHAs).nombre) '\ote{$|n_m|$}'

			$cs$.posicionesHippies[Siguiente(itHAs).y*Columnas($cs$.campus) +
		+ Siguiente(itHAs).x] $\leftarrow$ " " '\ote{1}'
			Avanzar(itHAs) '\ote{1}'
		end while '\ote{$4|n_m|$}'
	end if '\ote{$4|n_m|$}'

	conj(nombre, posicion): conjEstRodHip $\leftarrow$
		$\leftarrow$ EstudiantesRodeadosHippies(vecinos(pos, $cs$.campus), $cs$) '\ote{1}'
	if Cardinal(conjEstRodHip) $>$ 0 then '\ote{1}'
		itConj(nombre, posicion): itEstH $\leftarrow$ CrearIt(conjEstRodHip) '\ote{1}'
		while HaySiguiente(itEstH) do '\ote{1}'
			Eliminar($cs$.estudiantes, Siguiente(itEstH).nombre) '\ote{$|n_m|$}'

			$cs$.posicionesEstudiantes[Siguiente(itEstH).y*Columnas($cs$.campus) +
		+ Siguiente(itEstH).x] $\leftarrow$ " " '\ote{1}'

			Definir($cs$.hippies, Siguiente(itEstH).nombre,
				Siguiente(itEstH).posicion) '\ote{$|n_m|$}'

			$cs$.posicionesHippies[Siguiente(itEstH).y*Columnas($cs$.campus) +
		+ Siguiente(itEstH).x] $\leftarrow$ Siguiente(itEstH).nombre '\ote{1}'

			Avanzar(itHAs) '\ote{1}'
		end while '\ote{$4*2|n_m|$}'
	end if '\ote{$4*2|n_m|$}'

	conj(posicion): conjEstRodAs $\leftarrow$
		$\leftarrow$ EstudiantesRodeadosAs(vecinos(pos, $cs$.campus), $cs$) '\ote{1}'
	if Cardinal(conjEstRodAs) $> 0$ then '\ote{1}'
		itConj(posicion): itEstAs $\leftarrow$ CrearIt(conjEstRodAs) '\ote{1}'

		while HaySiguiente(itEstAs) do '\ote{1}'
			if todasOcupadas?(vecinos(Siguiente(itEstAs), $cs$.campus), $cs$ AND
			AND AlMenosUnAgente(vecinos(Siguiente(itEstAs), $cs$.campus) then '\ote{1}'
				conj(As): conjAgParaSanc $\leftarrow$
					$\leftarrow$ AgParaPremSanc(vecinos(Siguiente(itEstAs), $cs$.campus), $cs$) '\ote{1}'
				SancionarAgentes(conjAgParaSanc, $cs$) '\ote{1}'
			end if '\ote{1}'
		end while '\ote{4}'

	end if '\ote{4}'

'\ofi{O(2|n_m|) + O(4*2|n_m|) + O(4|n_m|) + O(4*2|n_m|) = O(22|n_m|) = O(|n_m|)}'
\end{lstlisting}

\begin{lstlisting}[mathescape]
'\alg{EstudiantesRodeadosAs}{\In{c}{conj(posicion)}, \In{cs}{estr}}{conj(posicion)}'
	itConj(posicion): itC $\leftarrow$ CrearIt(c) '\ote{1}'
	res $\leftarrow$ Vacio()

	while HaySiguiente(itC) do '\ote{1}'
		if TodasOcupadas?(vecinos(Siguiente(itC), $cs$), $cs$) AND
		AND AlMenosUnAgente(vecinos(Siguiente(itC), $cs$), $cs$) then '\ote{1}'
		AgregarRapido(res, Siguiente(itC)) '\ote{1}'
		end if '\ote{1}'
		Avanzar(itC) '\ote{1}'
	end while '\ote{4}'

'\ofi{O(4) = O(1)}'
\end{lstlisting}

\begin{lstlisting}[mathescape]
'\alg{EstudiantesRodeadosHippies}{\In{c}{conj(posicion)}, \In{cs}{estr}}{conj(nombre, posicion)}'
	itConj(posicion): itC $\leftarrow$ CrearIt(c) '\ote{1}'
	res $\leftarrow$ Vacio() '\ote{1}'

	while HaySiguiente(itC) do '\ote{1}'
		if $cs$.posicionesEstudiantes[Siguiente(itC).y*Colummnas($cs$.campus) +
		+ Siguiente(itC).x] $\neq$ " " AND
		AND TodasOcupadas?(vecinos(Siguiente(itC), $cs$.campus), $cs$) AND
		AND HippiesAtrapando(vecinos(Siguiente(itC), $cs$.campus), $cs$) then '\ote{1}'
		AgregarRapido(res,
			tupla($cs$.posicionesEstudiantes[Siguiente(itC).y*Columnas($cs$.campus) +
			+ Siguiente(itC).x], Siguiente(itC)) '\ote{1}'

	end while '\ote{4}'

'\ofi{O(1)}'
\end{lstlisting}

\begin{lstlisting}[mathescape]
'\alg{HippiesAtrapando}{\In{c}{conj(posicion)}, \In{cs}{estr}}{bool}'
	nat: i $\leftarrow$ 0 '\ote{1}'
	itConj(posicion): itC $\leftarrow$ CrearIt(c) '\ote{1}'

	while HaySiguiente(itC) do '\ote{1}'
		if $cs$.posicionesHippies[Siguiente(itC).y*Columnas($cs$.campus) +
		+ Siguiente(itC).x] $\neq$ " " then '\ote{1}'
			i $\leftarrow$ i$+1$ '\ote{1}'
		end if '\ote{1}'
		Avanzar(itC) '\ote{1}'
	end while '\ote{4}'

	res $\leftarrow$ i $\geq 2$ '\ote{1}'

'\ofi{O(1)}'
\end{lstlisting}

\begin{lstlisting}[mathescape]
'\alg{SancionarAgentes}{\In{c}{conj(As)}, \Inout{cs}{estr}}{}'
	itConj(As): itC $\leftarrow$ CrearIt(c) '\ote{1}'

	while HaySiguiente(itC) do '\ote{1}'
		*Siguiente(itC).datos.cantSanccantSanc $\leftarrow$
			$\leftarrow$ *Siguiente(itC).datos.cantSanc$+1$ '\ote{1}'
		itLista(kSanc): itLis $\leftarrow$ *Siguiente(itC).datos.itMismasSanc'\ote{1}'

		if HaySiguiente(*Siguiente(itC).datos.itMismasSanc) then '\ote{1}'
			Avanzar(*Siguiente(itC).datos.itMismasSanc)
			if Siguiente(*Siguiente(itC).datos.itMismasSanc).sanc =
			*Siguiente(itC).datos.cantSanc then '\ote{1}'
			else
				AgregarComoAnterior(*Siguiente(itC).datos.itMismasSanc,
					tupla(*Siguiente(itC).datos.cantSanc, Vacio())) '\ote{1}'
				Retroceder(*Siguiente(itC).datos.itMismasSanc) '\ote{1}'
			end if

		else
			AgregarComoSiguiente(*Siguiente(itC).datos.itMismasSanc,
				tupla(*Siguiente(itC).datos.cantSanc, Vacio()) '\ote{1}'
			Avanzar(*Siguiente(itC).datos.itMismasSanc) '\ote{1}'
		end if

		EliminarSiguiente(*Siguiente(itC).datos.itConjMismasSanc) '\ote{1}'
		*Siguiente(itC).datos.itConjMismasSanc $\leftarrow$
			$\leftarrow$ AgregarAdelante(*Siguiente(itC).datos.itMismasSanc.agentes,
				*Siguiente(itC).agente) '\ote{1}'
	end while '\ote{1}'

'\ofi{O(1)}'
\end{lstlisting}

\begin{lstlisting}[mathescape]
'\alg{HippiesRodeadosAs}{\In{c}{conj(posicion)}, \In{cs}{estr}}{conj(nombre, posicion)}'
	itConj(posicion): itC $\leftarrow$ CrearIt(c) '\ote{1}'
	res $\leftarrow$ Vacio() '\ote{1}'

	while HaySiguiente(itC) do '\ote{1}'
		if $cs$.posicionesHippies[Siguiente(itC).y*Columnas($cs$.campus) +
		+ Siguiente(itC).y] $\neq$ " "  AND
		todasOcupadas?(vecinos(Siguiente(itDiccS), $cs$.campus)) AND
		AlMenosUnAgente(vecinos(Siguiente(itDiccS), $cs$.campus)) then '\ote{1}'
			AgregarRapido(res,
			tupla($cs$.posicionesHippies[Siguiente(itC).y*Columnas($cs$.campus) +
			+ Siguiente(itC).y],
				Siguiente(itC)) '\ote{1}'

			conj(As): conjAgPremiar $\leftarrow$
			$\leftarrow$ AgParaPremSanc(vecinos(Siguiente(itDiccS), $cs$), $cs$) '\ote{1}'
			PremiarAgentes(conjAgPremiar, $cs$) '\ote{1}'

		end if '\ote{1}'
		Avanzar(itC) '\ote{1}'
	end while '\ote{4}'

'\ofi{O(4) = O(1) \text{. La cantidad maxima del conjunto de posicion es 4.}}'
\end{lstlisting}

\begin{lstlisting}[mathescape]
'\alg{AgParaPremSanc}{\In{c}{conj(posicion)}, \In{cs}{estr}}{conj(As)}'
	itConj(posicion): itC $\leftarrow$ CrearIt(c) '\ote{1}'
	res $\leftarrow$ Vacio() '\ote{1}'

	while HaySiguiente(itC) do '\ote{1}'
		if $cs$.posicionesAgente[Siguiente(itC).y * Columnas($cs$.campus) + x].datos $\neq$
			$\neq$ NULL then '\ote{1}'
			AgregarRapido(res,
				$cs$.posicionesAgente[Siguiente(itC).y * Columnas($cs$.campus) + x]) '\ote{1}'
		end if '\ote{1}'
		Avanzar(itC) \ote{1}'
	end while '\ote{4}'

'\ofi{O(1) + O(1) + O(4) = O(1)}'
\end{lstlisting}

\begin{lstlisting}[mathescape]
'\alg{PremiarAgentes}{\In{c}{conj(As)}, \Inout{cs}{estr}}{}'
	itConj(As): itC $\leftarrow$ CrearIt(As) '\ote{1}'

	while HaySiguiente(itC) do '\ote{1}'
		*Siguiente(itC).datos.cantAtrapados $\leftarrow$
			$\leftarrow$ *Siguiente(itC).datos.cantAtrapados$+1$ '\ote{1}'
	end while '\ote{4}'

'\ofi{O(4) = O(1) \text{. La cantidad maxima de agentes va a ser 4.}}'
\end{lstlisting}

\begin{lstlisting}[mathescape]
'\alg{CantHippiesVecinos}{\In{c}{conj(posicion)}, \In{cs}{estr}}{nat}'
	itConj(posicion): itC $\leftarrow$ CrearIt(c) '\ote{1}'
	res $\leftarrow$ $0$

	while HaySiguiente(itC) do '\ote{1}'
		itDiccString(nombre, posicion): itDic $\leftarrow$ CrearIt($cs$.hippies) '\ote{1}'

		while HaySiguiente(itDic) do '\ote{1}'
			if Siguiente(itDic) = Siguiente(itC) '\ote{1}'
				res $\leftarrow$ res$+1$ '\ote{1}'
			end if '\ote{1}'
		end while '\ote{$|n_m|$}'

		Avanzar(itC) '\ote{1}'
	end while '\ote{$|n_m|$}'

'\ofi{O(|n_m|) \text{. Como vecinos como maximo tiene longitud 4, la complejidad sería } O(4*|n_m|) = O(|n_m|)}}'
\end{lstlisting}

\begin{lstlisting}[mathescape]
'\alg{HippiesRodeadosEstudiantes}{\In{c}{conj(posicion}, \In{cs}{estr}}{conj(nombre, posicion)}'
	itConj(posicion): itC $\leftarrow$ CrearIt(c) '\ote{1}'
	res $\leftarrow$ Vacio() '\ote{1}'

	while HaySiguiente(itC) do '\ote{1}'
		if $cs$.posicionesHippies[Siguiente(itC).y*Columnas($cs$.campus) +
		+ Siguiente(itC).x] $\neq$ " "  AND
		todasOcupadas?(vecinos(Siguiente(itC), $cs$.campus)) AND
		TodosEstudiantes(vecinos(Siguiente(itC), $cs$.campus)) then '\ote{1}'
			AgregarRapido(res,
				tupla(, $cs$.posicionesHippies[Siguiente(itC).y*Columnas($cs$.campus) +
		+ Siguiente(itC).x]
				Siguiente(itC)) '\ote{1}'
		end if '\ote{1}'
		Avanzar(itC) '\ote{1}'
	end while '\ote{4}'

'\ofi{O(4) = O(1) \text{. Como el conjunto c tiene como maximo longitud 4, la complejidad sería } O(1)}'
\end{lstlisting}

\begin{lstlisting}[mathescape]
'\alg{todasOcupadas?}{\In{c}{conj(posicion)}, \In{cs}{estr}}{bool}'
	res $\leftarrow$ false '\ote{1}'
	itConj(posicion): itC $\leftarrow$ CrearIt(c) '\ote{1}'

	while HaySiguiente(itC) do '\ote{1}'
		if $cs$.posicionesHippies[Siguiente(itC)*Columnas($cs$.campus) +
		+ Siguiente(itC).y] $\neq$ " " then '\ote{1}'
			res $\leftarrow$ true '\ote{1}'

		if $cs$.posicionesEstudiantes[Siguiente(itC)*Columnas($cs$.campus) +
		+ Siguiente(itC).y] $\neq$ " " then '\ote{1}'
			res $\leftarrow$ true '\ote{1}'

		Avanzar(itC) '\ote{1}'
	end while '\ote{1}'

	itC $\leftarrow$ CrearIt(c) '\ote{1}'
	while HaySiguiente(itC) do '\ote{1}'
		if $cs$.posicionesAgente[Siguiente(itC).y * Columnas($cs$.campus) + x].datos $\neq$
		$\neq$ NULL then '\ote{1}'
			res $\leftarrow$ true '\ote{1}'
		end if '\ote{1}'
	end while '\ote{4}'

	itC $\leftarrow$ CrearIt(c) '\ote{1}'
	while HaySiguiente(itC) do '\ote{1}'
		if Ocupada?($cs$.campus, Siguiente(itC)) '\ote{1}'
			res $\leftarrow$ true
		end if '\ote{1}'
		Avanzar(itC) '\ote{1}'
	end while '\ote{4}'

'\ofi{O(1) \text{. Como el conjunto c tiene como maximo longitud 4, la complejidad sería } O(1)}'
\end{lstlisting}

\begin{lstlisting}[mathescape]
'\alg{AlMenosUnAgente}{\In{c}{conj(posicion)}, \In{cs}{estr} }{bool}'
	itConj(posicion): itC $\leftarrow$ CrearIt(c) '\ote{1}'
	res $\leftarrow$ false '\ote{1}'

	while HaySiguiente(itC) do '\ote{1}'
		if $cs$.posicionesAgente[Siguiente(itC).y * Columnas($cs$.campus) + x].datos $\neq$
		$\neq$ NULL then '\ote{1}'
			res $\leftarrow$ true '\ote{1}'
		end if '\ote{1}'
	end while '\ote{4}'

'\ofi{O(4) = O(1) \text{. Como el conjunto c tiene maximo 4 posiciones, solo hace el ciclo 4 veces maximo.}}'
\end{lstlisting}

\begin{lstlisting}[mathescape]
'\alg{TodosEstudiantes}{\In{c}{conj(posicion)}, \In{cs}{estr} }{bool}'
	itConj(posicion): itC $\leftarrow$ CrearIt(c)
	res $\leftarrow$ true

	while HaySiguiente(itC) AND res = true do '\ote{1}'
		if $cs$.posicionesEstudiantes[Siguiente(itC).y*Columnas($cs$.campus) +
		+ Siguiente(itC).x] $=$ " " then '\ote{1}'
			res $\leftarrow$ false
		end if '\ote{1}'

		Avanzar(itC) '\ote{1}'
	end while '\ote{1}'

'\ofi{O(1) \text{. Como el conjunto c tiene como maximo longitud 4, la complejidad sería } O(1)}'
\end{lstlisting}

\begin{lstlisting}[mathescape]
'\alg{iIngresarHippie}{\In{h}{nombre}, \In{pos}{posicion}, \Inout{cs}{estr} }{}'
	if todasOcupadas?(vecinos(pos, $cs$.campus), $cs$) AND
	AND AlMenosUnAgente(vecinos(pos, $cs$.campus) then '\ote{1}'
		conj(As): conjAgParaPrem $\leftarrow$ AgParaPremSanc(vecinos(pos, $cs$.campus), $cs$) '\ote{1}'
		PremiarAgentes(conjAgParaSanc, $cs$) '\ote{1}'
	else if todasOcuapadas?(vecinos(pos, $cs$.campus), $cs$) AND
	AND TodosEstudiantes(vecinos(pos, $cs$.campus), $cs$) then '\ote{1}'
		Definir($cs$.estudiantes, h, pos) '\ote{$|n_m|$}'
		$cs$.posicionesEstudiantes[pos.y*Columnas($cs$.campus) + pos.x] $\leftarrow$ h
	else
		Definir($cs$.hippies, h, pos) '\ote{$|n_m|$}'
		$cs$.posicionesHippies[pos.y*Columnas($cs$.campus) + pos.x] $\leftarrow$ h
	end if '\ote{$|n_m|$}'

	conj(nombre, posicion): conjHippiesRodEst $\leftarrow$
		$\leftarrow$ HippiesRodeadosEstudiantes(vecinos(pos, $cs$.campus), $cs$) '\ote{1}'

	if Cardinal(conjHippiesRodEst) $>$ 0 then '\ote{1}'
		itConj(nombre, posicion): itHEst $\leftarrow$ CrearIt(conjHippiesRodEst) '\ote{1}'

		while HaySiguiente(itHEst) do '\ote{1}'
			Definir($cs$.estudiantes, Siguiente(itHEst).nombre,
				Siguiente(itHEst).posicion) '\ote{$|n_m|$}'
			Eliminar($cs$.hippies, Siguiente(itHEst).nombre) '\ote{$|n_m|$}'

			$cs$.posicionesEstudiantes[Siguiente(itHEst).posicion.y*Columnas($cs$.campus) +
			+ Siguiente(itHEst).posicion.x] $\leftarrow$
			$\leftarrow$ $cs$.posicionesHippies[Siguiente(itHEst).posicion.y*Columnas($cs$.campus) +
			+ Siguiente(itHest).posicion.x] '\ote{1}'

			$cs$.posicionesHippies[Siguiente(itHEst).posicion.y*Columnas($cs$.campus) +
			+ Siguiente(itHest).posicion.x] $\leftarrow$ " " '\ote{1}'

			Avanzar(itHEst) '\ote{1}'
		end while '\ote{$4*2|n_m|$}'
	end if '\ote{$4*2|n_m|$}'

	conj(nombre, posicion): conjHippiesRodAs $\leftarrow$
		$\leftarrow$ HippiesRodeadosAs(vecinos(pos, $cs$.campus), $cs$) '\ote{1}'
	if Cardinal(conjHippiesRodAs) $>$ 0 then '\ote{1}'
		itConj(nombre, posicion): itHAs $\leftarrow$ CrearIt(conjHippiesRodAs) '\ote{1}'
		while HaySiguiente(itHAs) do '\ote{1}'
			Eliminar($cs$.hippies, Siguiente(itHAs).nombre) '\ote{$|n_m|$}'

			$cs$.posicionesHippies[Siguiente(itHAs).y*Columnas($cs$.campus) +
		+ Siguiente(itHAs).x] $\leftarrow$ " " '\ote{1}'
			Avanzar(itHAs) '\ote{1}'
		end while '\ote{$4|n_m|$}'
	end if '\ote{$4|n_m|$}'

	conj(nombre, posicion): conjEstRodHip $\leftarrow$
		$\leftarrow$ EstudiantesRodeadosHippies(vecinos(pos, $cs$.campus), $cs$) '\ote{1}'
	if Cardinal(conjEstRodHip) $>$ 0 then '\ote{1}'
		itConj(nombre, posicion): itEstH $\leftarrow$ CrearIt(conjEstRodHip) '\ote{1}'
		while HaySiguiente(itEstH) do '\ote{1}'
			Eliminar($cs$.estudiantes, Siguiente(itEstH).nombre) '\ote{$|n_m|$}'

			$cs$.posicionesEstudiantes[Siguiente(itEstH).y*Columnas($cs$.campus) +
		+ Siguiente(itEstH).x] $\leftarrow$ " " '\ote{1}'

			Definir($cs$.hippies, Siguiente(itEstH).nombre,
				Siguiente(itEstH).posicion) '\ote{$|n_m|$}'

			$cs$.posicionesHippies[Siguiente(itEstH).y*Columnas($cs$.campus) +
		+ Siguiente(itEstH).x] $\leftarrow$ Siguiente(itEstH).nombre '\ote{1}'

			Avanzar(itHAs) '\ote{1}'
		end while '\ote{$4*2|n_m|$}'
	end if '\ote{$4*2|n_m|$}'

	conj(posicion): conjEstRodAs $\leftarrow$
		$\leftarrow$ EstudiantesRodeadosAs(vecinos(pos, $cs$.campus), $cs$) '\ote{1}'
	if Cardinal(conjEstRodAs) $> 0$ then '\ote{1}'
		itConj(posicion): itEstAs $\leftarrow$ CrearIt(conjEstRodAs) '\ote{1}'

		while HaySiguiente(itEstAs) do '\ote{1}'
			if todasOcupadas?(vecinos(Siguiente(itEstAs), $cs$.campus), $cs$ AND
			AND AlMenosUnAgente(vecinos(Siguiente(itEstAs), $cs$.campus) then '\ote{1}'
				conj(As): conjAgParaSanc $\leftarrow$
					$\leftarrow$ AgParaPremSanc(vecinos(Siguiente(itEstAs), $cs$.campus), $cs$) '\ote{1}'
				SancionarAgentes(conjAgParaSanc, $cs$) '\ote{1}'
			end if '\ote{1}'
		end while '\ote{4}'

	end if '\ote{4}'

'\ofi{O(|n_m|)}'


\end{lstlisting}

\begin{lstlisting}[mathescape]
'\alg{iMoverEstudiante}{\In{e}{nombre}, \In{d}{dirección}, \Inout{cs}{estr} }{}'

	Posicion: actualPos $\leftarrow$ Obtener(cs.estudiantes,e) '\ote{$|n_m|$}'
	Posicion: pos $\leftarrow$ actualPos '\ote{1}'
	if(d=Izquierda) then '\ote{1}'
		pos.x $\leftarrow$ pos.x - 1 '\ote{1}'
	else if (d=derecha) then '\ote{1}'
		pos.x $\leftarrow$ pos.x + 1 '\ote{1}'
	else if (d=Arriba) then '\ote{1}'
		pos.y $\leftarrow$ pos.y + 1 '\ote{1}'
 	else if (d=Abajo) then '\ote{1}'
 		pos.y $\leftarrow$ pos.y - 1 '\ote{1}'
 	end if

 	if(not (pos.y = 1 OR pos.y = cd.campus.filas)) then '\ote{1}'
		if CantHippiesVecinos(vecinos(pos, $cs$.campus), $cs$)$< 2$  then '\ote{$|n_m|$}'
			Definir($cs$.estudiantes, e, pos) '\ote{$|n_m|$}'
			$cs$.posicionesEstudiantes[pos.y * Columnas($cs$.campus) + pos.x] $\leftarrow$ e '\ote{1}'
		else
			Definir($cs$.hippies, e, pos) '\ote{$|n_m|$}'
			$cs$.posicionesHippies[pos.y * Columnas($cs$.campus) + pos.x] $\leftarrow$ e '\ote{1}'
		end if '\ote{$|n_m| + |n_m|$}'

		conj(nombre, posicion): conjHippiesRodEst $\leftarrow$
			$\leftarrow$ HippiesRodeadosEstudiantes(vecinos(pos, $cs$.campus), $cs$) '\ote{1}'

		if Cardinal(conjHippiesRodEst) $>$ 0 then '\ote{1}'
			itConj(nombre, posicion): itHEst $\leftarrow$ CrearIt(conjHippiesRodEst) '\ote{1}'

			while HaySiguiente(itHEst) do '\ote{1}'
				Definir($cs$.estudiantes, Siguiente(itHEst).nombre,
					Siguiente(itHEst).posicion) '\ote{$|n_m|$}'
				Eliminar($cs$.hippies, Siguiente(itHEst).nombre) '\ote{$|n_m|$}'

				$cs$.posicionesEstudiantes[Siguiente(itHEst).posicion.y*Columnas($cs$.campus) +
				+ Siguiente(itHEst).posicion.x] $\leftarrow$
				$\leftarrow$ $cs$.posicionesHippies[Siguiente(itHEst).posicion.y*Columnas($cs$.campus) +
				+ Siguiente(itHest).posicion.x] '\ote{1}'

				$cs$.posicionesHippies[Siguiente(itHEst).posicion.y*Columnas($cs$.campus) +
				+ Siguiente(itHest).posicion.x] $\leftarrow$ " " '\ote{1}'

				Avanzar(itHEst) '\ote{1}'
			end while '\ote{$2|n_m|$}'
		end if '\ote{$4 * 2|n_m|$}'

		conj(nombre, posicion): conjHippiesRodAs $\leftarrow$
			$\leftarrow$ HippiesRodeadosAs(vecinos(pos, $cs$.campus), $cs$) '\ote{1}'
		if Cardinal(conjHippiesRodAs) $>$ 0 then '\ote{1}'
			itConj(nombre, posicion): itHAs $\leftarrow$ CrearIt(conjHippiesRodAs) '\ote{1}'
			while HaySiguiente(itHAs) do '\ote{1}'
				Eliminar($cs$.hippies, Siguiente(itHAs).nombre) '\ote{$|n_m|$}'

				$cs$.posicionesHippies[Siguiente(itHAs).y*Columnas($cs$.campus) +
			+ Siguiente(itHAs).x] $\leftarrow$ " " '\ote{1}'
				Avanzar(itHAs) '\ote{1}'
			end while '\ote{$4|n_m|$}'
		end if '\ote{$4|n_m|$}'

		conj(nombre, posicion): conjEstRodHip $\leftarrow$
			$\leftarrow$ EstudiantesRodeadosHippies(vecinos(pos, $cs$.campus), $cs$) '\ote{1}'
		if Cardinal(conjEstRodHip) $>$ 0 then '\ote{1}'
			itConj(nombre, posicion): itEstH $\leftarrow$ CrearIt(conjEstRodHip) '\ote{1}'
			while HaySiguiente(itEstH) do '\ote{1}'
				Eliminar($cs$.estudiantes, Siguiente(itEstH).nombre) '\ote{$|n_m|$}'

				$cs$.posicionesEstudiantes[Siguiente(itEstH).y*Columnas($cs$.campus) +
			+ Siguiente(itEstH).x] $\leftarrow$ " " '\ote{1}'

				Definir($cs$.hippies, Siguiente(itEstH).nombre,
					Siguiente(itEstH).posicion) '\ote{$|n_m|$}'

				$cs$.posicionesHippies[Siguiente(itEstH).y*Columnas($cs$.campus) +
			+ Siguiente(itEstH).x] $\leftarrow$ Siguiente(itEstH).nombre '\ote{1}'

				Avanzar(itHAs) '\ote{1}'
			end while '\ote{$4*2|n_m|$}'
		end if '\ote{$4*2|n_m|$}'

		conj(posicion): conjEstRodAs $\leftarrow$
			$\leftarrow$ EstudiantesRodeadosAs(vecinos(pos, $cs$.campus), $cs$) '\ote{1}'
		if Cardinal(conjEstRodAs) $> 0$ then '\ote{1}'
			itConj(posicion): itEstAs $\leftarrow$ CrearIt(conjEstRodAs) '\ote{1}'

			while HaySiguiente(itEstAs) do '\ote{1}'
				if todasOcupadas?(vecinos(Siguiente(itEstAs), $cs$.campus), $cs$ AND
				AND AlMenosUnAgente(vecinos(Siguiente(itEstAs), $cs$.campus) then '\ote{1}'
					conj(As): conjAgParaSanc $\leftarrow$
						$\leftarrow$ AgParaPremSanc(vecinos(Siguiente(itEstAs), $cs$.campus), $cs$) '\ote{1}'
					SancionarAgentes(conjAgParaSanc, $cs$) '\ote{1}'
				end if '\ote{1}'
			end while '\ote{4}'

		end if '\ote{4}'
	else
		Borrar(cs.estudiantes,e) '\ote{$|n_m|$}'
	end if

'\ofi{O(|n_m|)}'

\end{lstlisting}

\begin{lstlisting}[mathescape]
'\alg{iMoverHippie}{\In{h}{nombre}, \Inout{cs}{estr} }{}'

'\ofi{}'
\end{lstlisting}

\begin{lstlisting}[mathescape]
'\alg{iMoverAgente}{\In{a}{agente}, \Inout{cs}{estr} }{}'

'\ofi{}'
\end{lstlisting}

\begin{lstlisting}[mathescape]
'\alg{iCampus}{\In{cs}{estr} }{campus}'
	$res$ $\leftarrow$ $cs$.campus'\ote{1}'
'\ofi{O(1)}'
\end{lstlisting}

\begin{lstlisting}[mathescape]
'\alg{iEstudiantes}{\In{cs}{estr} }{itDiccString(nombre,posición)}'
	$res$ $\leftarrow$ CrearIt($cs$.estudiantes)'\ote{1}'
'\ofi{O(1)}'
\end{lstlisting}

\begin{lstlisting}[mathescape]
'\alg{iHippies}{\In{cs}{estr} }{itDiccString(nombre,posición)}'
	$res$ $\leftarrow$ CrearIt($cs$.hippies)'\ote{1}'
'\ofi{O(1)}'
\end{lstlisting}

\begin{lstlisting}[mathescape]
'\alg{iAgentes}{\In{cs}{estr} }{itDiccNat(agente,datosAgente)}'
	$res$ $\leftarrow$ CrearIt($cs$.personalAS)'\ote{1}'
'\ofi{O(1)}'
\end{lstlisting}

\begin{lstlisting}[mathescape]
'\alg{iPosEstudianteYHippie}{\In{id}{nombre}, \In{cs}{estr} }{posición}'
	if Definido?($id$, $cs$.hippies) then '\ote{$|n_m|$}'
		$res$ $\leftarrow$ Obtener($id$, $cs$.hippies)'\ote{$|n_m|$}'
	else
		$res$ $\leftarrow$ Obtener($id$, $cs$.estudiantes)'\ote{$|n_m|$}'
	end if
'\ofi{O(|n_m|) + O(4*2|n_m|) = O(|n_m|)}'
\end{lstlisting}

\begin{lstlisting}[mathescape]
'\alg{iPosAgente}{\In{a}{agente}, \In{cs}{estr} }{posición}'
	$res$ $\leftarrow$ Obtener($a$, $cs$.personalAS).posicion '\ote{1 caso promedio}'
'\ofi{O(1) \text{ caso promedio}}'
\end{lstlisting}

\begin{lstlisting}[mathescape]
'\alg{iCantSanciones}{\In{a}{agente}, \In{cs}{estr} }{}'
	$res$ $\leftarrow$ Obtener($a$, $cs$.personalAS).cantSanc '\ote{1 caso promedio}'
'\ofi{O(1) \text{ caso promedio}}'
\end{lstlisting}

\begin{lstlisting}[mathescape]
'\alg{iCantHippiesAtrapados}{\In{a}{agente}, \In{cs}{estr} }{}'
	$res$ $\leftarrow$ Obtener($a$, $cs$.personalAS).cantAtrapados '\ote{1 caso promedio}'
'\ofi{O(1) \text{ caso promedio}}'
\end{lstlisting}

\begin{lstlisting}[mathescape]
'\alg{iConMismasSanciones}{\In{a}{agente}, \In{cs}{estr} }{lista(agente)}'
	$res$ $\leftarrow$ Obtener($a$, $cs$.personalAS).itMismasSanc.conjAgente '\ote{1 caso promedio}'
'\ofi{O(1) \text{ caso promedio}}'
\end{lstlisting}

\begin{lstlisting}[mathescape]
'\alg{iConKSanciones}{\In{k}{nat}, \In{cs}{estr} }{lista(agente)}'
	if $cs$.mismasSancModificado == true then '\ote{1}'
		hacerArregloMismasSanc($cs$) '\ote{$N_a$}'
		$cs$.mismasSancModificado $\leftarrow$ false
	end if

	nat: i $\leftarrow$ 0
	bool: esta $\leftarrow$ busqBinAgente($k$, i, $cs$.vectorMismasSanc) '\ote{$logN_a$}'
	if esta = true then '\ote{1}'
		res $\leftarrow$ *$cs$.vectorMismasSanc[i].agentes '\ote{1}'
	else
		res $\leftarrow$ Vacia() '\ote{1}'
	end if

'\ofi{\text{Si las sanciones no fueron modificadas: } O(N_a) \text{. Si no } O(logN_a)}'
\end{lstlisting}

\begin{lstlisting}[mathescape]
'\alg{hacerArregloMismasSanc}{\Inout{cs}{estr}}{}'
	arreglo(puntero(kSanc)): arregloNuevo $\leftarrow$ CrearArreglo(Longitud($cs$.listaMismasSanc))
	'\ote{$N_a$}'
	itLista(kSanc): it $\leftarrow$ CrearIt($cs$.listaMismasSanc) '\ote{1}'
	nat: i $\leftarrow$ $0$ '\ote{1}'

	while HaySiguiente(it) do '\ote{1}'
		puntero(kSanc): p $\leftarrow$ puntero(Siguiente(it)) '\ote{1}'
		arregloNuevo[i] $\leftarrow$ p '\ote{1}'
		i $\leftarrow$ i $+1$ '\ote{1}'
		Avanzar(it) '\ote{1}'
	end while '\ote{$N_a$}'

	$cs$.arregloMismasSanc $\leftarrow$ arregloNuevo '\ote{1}'

'\ofi{O(N_a) + O(N_a) = O(N_a) \text{ Como máximo va a ser la cantidad de agentes la lista listaMismasSanc} }'
\end{lstlisting}

\begin{lstlisting}[mathescape]
'\alg{busqBinAgente}{\In{k}{nat}, \Inout{i}{nat}, \In{v}{arreglo(puntero(kSanc))}}{bool}'
	nat: n $\leftarrow$ $0$ '\ote{1}'
	nat: m $\leftarrow$ Longitud(v) '\ote{1}'
	nat: med

	while n $\neq$ m$-1$ do '\ote{1}'
		med $\leftarrow$ $\frac{n+m}{2}$ '\ote{1}'
		if med $\leq$ k then '\ote{1}'
			n $\leftarrow$ med '\ote{1}'
		else
			m $\leftarrow$ med '\ote{1}'
		end if
	end while '\ote{$log(N_a)$}'

	if v[n] $=$ k then
		i $\leftarrow$ n
		$res$ $\leftarrow$ true
	else
		$res$ $\leftarrow$ false
	end if

'\ofi{O(logN_a)}'

\end{lstlisting}

\begin{lstlisting}[mathescape]

'\alg{Atrapado?}{\In{c}{campus}, \In{pos}{Posicion}}{bool}'
	$res$ $\leftarrow$ todasOcupadas?(vecinos(pos,cs.campus)) '\ote{1}'
'\ofi{O(1)}'

\end{lstlisting}

\begin{lstlisting}[mathescape]
'\alg{busqBinPorPlaca}{\In{a}{agente}, \In{v}{vector(tupla(clave : agente, p : puntero(datosAgente))}}{puntero(datosAgente)}'
	nat: inf $\leftarrow$ $0$ '\ote{1}'
	nat: sup $\leftarrow$ Longitud(v) '\ote{1}'
	nat: med $\leftarrow$ $\frac{inf+sup}{2}$ '\ote{1}'

	while inf $\neq$ sup$-1$ do '\ote{1}'
		med $\leftarrow$ $\frac{inf+sup}{2}$ '\ote{1}'
		if v[med].clave $\leq$ a then '\ote{1}'
			inf $\leftarrow$ med '\ote{1}'
		else
			sup $\leftarrow$ med '\ote{1}'
		end if
	end while '\ote{$log(N_a)$}'

	$res$ $\leftarrow$ v[inf].p
	
'\ofi{O(logN_a)}'
\end{lstlisting}


\pagebreak
\section{Módulo Campus}

\subsection{Interfaz}

\textbf{se explica con}: \tadNombre{Campus}.

\textbf{géneros}: \TipoVariable{campus}.

\subsubsection{Operaciones básicas de Campus}

\InterfazFuncion{crearCampus}{\In{alto}{nat}, \In{ancho}{nat}}{campus}
[true]
{res $=_{obs}$ crearCampus(alto, ancho)}
[$O(1)$]
[Crea un nuevo campus tomando un alto y un ancho]
[]

~

\InterfazFuncion{agregarObstaculo}{\Inout{c}{campus}, \In{pos}{posicion}}{}
[$c =_{obs} c_0$ $\land$ posValida?(pos,c) $\yluego$ $\neg$ocupada?(pos,c)]
{$c =_{obs}$ agregarObstaculo($pos$,$c_0$)}
[$O(1)$]
[Agrega un obstaculo al campus]
[]

~

\InterfazFuncion{filas}{\In{c}{campus}}{nat}
[true]
{res $=_obs$ filas(c)}
[$O(1)$]
[Devuelve la cantidad de filas del campus]
[]

~

\InterfazFuncion{columnas}{\In{c}{campus}}{nat}
[true]
{res $=_{obs}$ columnas(c)}
[$O(1)$]
[Devuelve la cantidad de columnas del campus]
[]

~


\InterfazFuncion{ocupada?}{\In{c}{campus}, \In{pos}{posicion}}{bool}
[posValida(pos,c)]
{res $=_{obs}$ ocupada(pos,c)}
[$O(n)$ donde n es la cantidad de obstaculos]
[Devuelve true si la posicion esta ocupada por un obstaculo]
[]

~

\InterfazFuncion{posValida?}{\In{c}{campus}, \In{pos}{posicion}}{bool}
[true]
{$res =_{obs}$ posValida?($c$,$pos$)}
[$O(1)$]
[Devuelve true si la posicion es valida]
[]

~

\InterfazFuncion{esIngreso?}{\In{c}{campus}, \In{pos}{posicion}}{bool}
[true]
{$res =_{obs}$ esIngreso?($pos$,$c$)}
[$O(1)$]
[Devuelve true si la posicion es un ingreso. No tiene en cuenta su validez]
[]

~

\InterfazFuncion{ingresoSuperior?}{\In{c}{campus}, \In{pos}{posicion}}{bool}
[true]
{$res =_{obs}$ ingresoSuperior?($pos$,$c$)}
[$O(1)$]
[Devuelve true si la posicion es un ingreso superior. No tiene en cuenta su validez]
[]

~

\InterfazFuncion{ingresoInferior?}{\In{c}{campus}, \In{pos}{posicion}}{bool}
[true]
{$res =_{obs}$ ingresoInferior?($pos$,$c$)}
[$O(1)$]
[Devuelve true si la posicion es un ingreso inferior. No tiene en cuenta su validez]
[]

~

\InterfazFuncion{vecinos}{\In{c}{campus}, \In{pos}{posicion}}{conj(posicion)}
[posValida?(pos,c)]
{$res =_{obs}$ vecinos($pos$,$c$)}
[$O(1)$]
[Devuelve el conjunto posiciones validas adyacentes a pos]
[$res$ es una referencia no modificable]

~

\InterfazFuncion{distancia}{\In{c}{campus}, \In{pos1}{posicion}, \In{pos2}{posicion}}{nat}
[true]
{$res =_{obs}$ distancia($p1$,$p2$,$c$)}
[$O(1)$]
[Devuelve la distancia entre dos posiciones]
[]

~

\InterfazFuncion{proxPosicion}{\In{c}{campus}, \In{pos}{posicion}, \In{dir}{direccion}}{posicion}
[posValida(pos,c)]
{$res =_{obs}$ proxPosicion(pos,dir,c)}
[$O(1)$]
[Devuelve la posicion resultante al avanzar en la direccion pasada como parametro. No tiene en cuenta su validez]
[]

~

\InterfazFuncion{ingresosMasCercanos}{\In{c}{campus}, \In{pos}{posicion}}{conj(posicion)}
[posValida(pos,c)]
{$res =_{obs}$ ingresosMasCercanos(pos,c)}
[$O(1)$]
[Devuelve un conjunto con las posiciones de los ingresos mas cercanos]
[$res$ es una referencia no modificable]

~

\pagebreak

\subsubsection{Representación de campus}

\begin{Estructura}{campus}[estr]
	\begin{Tupla}[estr]
		\tupItem{filas}{nat}
		\tupItem{columnas}{nat}
		\tupItem{obstaculos}{conj(posicion)}
	\end{Tupla}
\end{Estructura}

\subsubsection{Invariante de Representación}

\renewcommand{\labelenumi}{(\Roman{enumi})}

\begin{enumerate}
	\item{La cantidad de obstaculos no puede ser mayor al producto de filas por columnas}
	\item{Las posiciones de los obstaculos deben ser validas, es decir, estar dentro de la grilla}

\end{enumerate}

\Rep[estr][e]{
	\\
	\#obstaculos $\leq$ filas * columnas $\land$ \\
	($\forall pos$: posicion)(x(pos) $\leq$ columnas $\land$ y(pos) $\leq$ filas)
}\mbox{}

\subsubsection{Funci\'on de Abstracci\'on}

\Abs[estr]{campus}[e]{cs}{
	\\
	filas(cs) = e.filas $\land$ \\
	columnas(cs) = e.columnas $\land$ \\
	($\forall pos$: posicion)(posValida?(pos,cs) $\Rightarrow_L$ (ocupada?(pos, cs) = pos $\in$ cs))
}

\subsection{Algoritmos}

\lstset{style=alg}

\begin{lstlisting}[mathescape]
'\alg{crearCampus}{\In{alto}{nat}, \In{ancho}{nat}}{estr}'
	
    res.filas $\leftarrow$ alto '\ote{1}'
    res.columnas $\leftarrow$ ancho '\ote{1}'
    res.obstaculos $\leftarrow$ Vacio() '\ote{1}'
    
'\ofi{O(1)}'
\end{lstlisting}

\begin{lstlisting}[mathescape]
'\alg{agregarObstaculo}{\Inout{cs}{campus}, \In{pos}{posicion}}{}'

	AgregarRapido(cs.obstaculos, pos) '\ote{1}'	

'\ofi{O(1)}'
\end{lstlisting}

\begin{lstlisting}[mathescape]
'\alg{filas}{\In{cs}{campus}}{nat}'

	res $\leftarrow$ cs.filas '\ote{1}'

'\ofi{O(1)}'
\end{lstlisting}

\begin{lstlisting}[mathescape]
'\alg{columnas}{\In{cs}{campus}}{nat}'

	res $\leftarrow$ cs.columnas '\ote{1}'

'\ofi{O(1)}'
\end{lstlisting}

\begin{lstlisting}[mathescape]
'\alg{ocupada?}{\In{cs}{campus}, \In{pos}{posicion}}{bool}'

	res $leftarrow$ Pertenece?(cs.obstaculos, pos) '\ote{n}'

'\ofi{O(n)}'
\end{lstlisting}

\begin{lstlisting}[mathescape]
'\alg{posValida?}{\In{cs}{campus}, \In{pos}{posicion}}{bool}'

	res $\leftarrow$ (pos.x $\geq$ 0) $\land$ (pos.x < cs.columnas) $\land$ 
	       (pos.y $\geq$ 0) $\land$ (pos.y < cs.filas) '\ote{1}'

'\ofi{O(1)}'
\end{lstlisting}

\begin{lstlisting}[mathescape]
'\alg{esIngreso}{\In{cs}{campus}, \In{pos}{posicion}}{bool}'

	res $\leftarrow$ esIngresoSuperior(cs, pos) $\lor$ esIngresoInferior(cs, pos) '\ote{1}'

'\ofi{O(1)}'
\end{lstlisting}

\begin{lstlisting}[mathescape]
'\alg{esIngresoSuperior}{\In{cs}{campus}, \In{pos}{posicion}}{bool}'
	
	res $\leftarrow$ pos.y = 0 '\ote{1}'

'\ofi{O(1)}'
\end{lstlisting}

\begin{lstlisting}[mathescape]
'\alg{esIngresoInferior}{\In{cs}{campus}, \In{pos}{posicion}}{bool}'
	
	res $\leftarrow$ pos.y = cs.filas - 1 '\ote{1}'

'\ofi{O(1)}'
\end{lstlisting}

\begin{lstlisting}[mathescape]
'\alg{$\puntito$ $=_i$ $\puntito$}{\In{dcn_1}{dcnet}, \In{dcn_2}{dcnet}}{bool}'

	bool: boolTopo $\leftarrow$ $dcn_1$.topologia = $dcn_2$.topologia '\ote{n + $L^2$}'
	bool: boolVec $\leftarrow$ $dcn_1$.vectorCompusDCNet = $dcn_2$.vectorCompusDCNet '\ote{n * k * (k + n)}'
	bool: boolConj $\leftarrow$ $dcn_1$.conjPaquetesDCNet = $dcn_2$.conjPaquetesDCNet '\ote{$k^3$ * (k + n)}'
	bool: boolMasEnvio $\leftarrow$ *($dcn_1$.laQueMasEnvio) = *($dcn_2$.laQueMasEnvio) '\ote{1}'

	res $\leftarrow$ boolTopo $\land$ boolVec $\land$ boolTrie $\land$ boolConj $\land$ boolMasEnvio '\ote{1}'

'\ofi{O(n * k^3 * (k + n))}'
\end{lstlisting}

\begin{lstlisting}[mathescape]
'\alg{$\puntito$ $=_{compudcn}$ $\puntito$}{\In{c_1}{compuDCNet}, \In{c_2}{compuDCNet}}{bool}'

	bool: boolPC $\leftarrow$ *($c_1$.pc) = *($c_2$.pc) '\ote{1}'
	bool: boolConj $\leftarrow$ $c_1$.conjPaquetes = $c_1$.conjPaquetes '\ote{$k^2$}'
	bool: boolAVL $\leftarrow$ true '\ote{1}'
	bool: boolCola $\leftarrow$ true '\ote{1}'
	bool: boolPaq $\leftarrow$ Siguiente($c_1$.paqueteAEnviar) $=_{paqdcn}$ Siguiente($c_2$.paqueteAEnviar)
		'\ote{n}'
	bool: boolEnviados $\leftarrow$ $c_1$.enviados = $c_2$.enviados '\ote{1}'

	if boolConj then '\ote{1}'
		itConj: $itconj_1$ $\leftarrow$ CrearIt($c_1$.conjPaquetes) '\ote{1}'
		while HaySiguiente?($itconj_1$) do '\ote{1}'
			if Definido?($c_2$.diccPaquetesDCNet, Siguiente($itconj_1$)).id then '\ote{log(n)}'
				if $\neg$(Siguiente(Obtener($c_1$.diccPaquetesDCNet, Siguiente($itconj_1$).id))
					$=_{paqdcn}$
					Siguiente(Obtener($c_1$.diccPaquetesDCNet, Siguiente($itconj_1$).id)))
					then '\ote{n}'
					boolAVL $\leftarrow$ false '\ote{1}'
				end if
			else
				boolAVL $\leftarrow$ false '\ote{1}'
			end if
			Avanzar($itconj_1$) '\ote{1}'
		end while '\ote{n * k}'
	end if

	if EsVacia($c_1$.colaPrioridad) then '\ote{1}'
		if $\neg$EsVacia($c_2$.colaPrioridad) then '\ote{1}'
			boolCola $\leftarrow$ false '\ote{1}'
		end if
	else
		if EsVacia($c_1$.colaPrioridad) then '\ote{1}'
			boolCola $\leftarrow$ false '\ote{1}'
		else
			if $\neg$(Siguiente(Proximo($c_1$.colaPrioridad)) $=_{paqdcn}$
				Siguiente(Proximo($c_2$.colaPrioridad))) then '\ote{n}'
				boolCola $\leftarrow$ false '\ote{1}'
			end if
		end if
	end if

	res $\leftarrow$ boolPC $\land$ boolConj $\land$ boolAVL $\land$ boolCola $\land$ boolPaq $\land$ boolEnviados '\ote{1}'

'\ofi{O(k^2 + n * k) = O(k * (k + n))}'
\end{lstlisting}

\begin{lstlisting}[mathescape]
'\alg{$\puntito$ $=_{paqdcn}$ $\puntito$}{\In{p_1}{paqueteDCNet}, \In{p_2}{paqueteDCNet},}{bool}'

	bool: boolPaq $\leftarrow$ Siguiente($p_1$.it) = Siguiente($p_2$.it) '\ote{1}'
	bool: boolRecorrido $\leftarrow$ $p_1$.recorrido = $p_2$.recorrido '\ote{n}'

	res $\leftarrow$ boolPaq $\land$ boolRecorrido '\ote{1}'

'\ofi{O(n)}'
\end{lstlisting}


\pagebreak
\section{Módulo Diccionario Nat Fijo}

\subsection{Interfaz}

\textbf{se explica con}: \tadNombre{Diccionario(nat, $\alpha$ )}.

\textbf{géneros}: \TipoVariable{DiccNat($\alpha$)}, \TipoVariable{itDiccNat($\alpha$)}.

\subsubsection{Operaciones básicas de DiccNat($\alpha$)}

\InterfazFuncion{crearDiccionario}{\In{v}{vector(tupla(clave : nat, significado :  $\alpha$))}}{DiccNat($\alpha$)}
[true]
{(($\forall$ $t$ : tupla(nat, $\alpha$)) esta?($t$, v)) $\Rightarrow$ ((definido?(t.clave, res))
	$\land_L$ obtener(t.clave,res) $=_{obs}$ t.significado) $\land$ cantClaves(res) $=_{obs}$ longitud(v)}
[$O(copy(\alpha) * n)$ donde n es el largo del vector]
[Agrega las tuplas de clave-significado pasadas por parametro al diccionario $d$]
[Los elementos pasados por parámetros se copian al diccionario]

~

\InterfazFuncion{redefinir}{\Inout{d}{DiccNat($\alpha$), \In{n}{nat}, \In{a}{$\alpha$}}}{}
[definido?(n,d)]
{obtener(n,d) $=_{obs}$ a}
[$O(1)$ en caso promedio, $O(\# claves)$ en peor caso]
[]

~

\InterfazFuncion{obtener}{\In{n}{nat}, \In{d}{DiccNat($\alpha$)}}{puntero($\alpha$)}
[true]
{res $=_{obs}$ obtener(n,d)}
[$O(1)$ en caso promedio, $O(\# claves)$ en peor caso]
[Devuelve un puntero al significado de la clave pasada por parametro. Si no está definido, devuelve NULL]
[El puntero va ser una referencia al significado almacenado en el diccionario]

~

\InterfazFuncion{definido?}{\In{n}{nat}, \In{d}{DiccNat($\alpha$)}}{bool}
[true]
{res $=_{obs}$ def?(n,d)}
[$O(1)$ en caso promedio, $O(\# claves)$ en peor caso]
[Dice si está definida una clave en el diccionario]
[]

~

\InterfazFuncion{cantClaves}{\In{d}{DiccNat($\alpha$)}}{Nat}
[true]
{res $=_{obs} \ \# claves$($d$)}
[$O(1)$]
[Devuelve la cantidad de claves definidas en el diccionario.]
[]

~

\InterfazFuncion{CrearItClaves}{\In{d}{DiccNat($\alpha$)}}{itConj(nat)}
[true]
{alias(esPermutacion?(SecuSuby(res), c)) $\land$ vacia?(Anteriores(res))}
[$O(1)$]
[crea un iterador bidireccional del conjunto, de forma tal que HayAnterior evalúe a false (i.e.,
que se pueda recorrer los elementos aplicando iterativamente Siguiente). Luego, se pueden utilizar
todas las funciones del iterador de conjunto sobre res.]
[El iterador se invalida si y sólo si se elimina el elemento siguiente del iterador sin utilizar la función
EliminarSiguiente. Además, anteriores(res) y siguientes(res) podrían cambiar completamente ante cualquier
operación que modifique c sin utilizar las funciones del iterador. Esto funciona tal como se indica en la interfaz del iterador de conjunto.]

~

\subsubsection{Operaciones básicas del iterador}

Este iterador permite recorrer la tabla de hash sobre la que est\'{a} implementado el diccionario para obtener cada clave con su respectivo significado sin modificar ning\'{u}n dato del diccionario.

\InterfazFuncion{CrearIt}{\In{d}{DiccNat($\alpha$)}}{itDiccNat($\alpha$)}
[true]
{alias(esPermutación(SecuSuby($res$), $d$)) $\land$ vacia?(Anteriores($res$))}
[$\Theta(1)$]
[crea un iterador bidireccional del diccionario, de forma tal que HayAnterior evalúe a false (i.e.,
que se pueda recorrer los elementos aplicando iterativamente Siguiente).]
[El iterador se invalida si y sólo si se elimina el elemento siguiente del iterador sin utilizar la función
EliminarSiguiente. Además, anteriores(res) y siguientes(res) podrían cambiar completamente ante cualquier
operación que modifique d sin utilizar las funciones del iterador.]

~

\InterfazFuncion{HaySiguiente}{\In{it}{itDiccNat($\alpha$)}}{bool}
[true]
{res $=_{obs}$ haySiguiente?($it$)}
[$\Theta(1)$]
[devuelve true si y sólo si en el iterador todavía quedan elementos para avanzar.]
[]

~

\InterfazFuncion{Siguiente}{\In{it}{itDiccNat($\alpha$)}}{tupla(nat,$\alpha$)}
[haySiguiente?(it)]
{alias(res $=_{obs}$ Siguiente($it$))}
[$\Theta(1)$]
[devuelve el elemento siguiente del iterador.]
[res.significado es modificable si y sólo si it es modificable. En cambio, res.clave no es modificable.]

~

\InterfazFuncion{SiguienteSignificado}{\In{it}{itDiccNat($\alpha$)}}{$\alpha$}
[haySiguiente?(it)]
{alias(res $=_{obs}$ Siguiente($it$).significado)}
[$\Theta(1)$]
[devuelve el significado del elemento siguiente del iterador]
[res es modificable si y sólo si it es modficable.]

~

\InterfazFuncion{Avanzar}{\Inout{it}{itDiccNat($\alpha$)}}{}
[it $=$ it$_0$ $\land$ haySiguiente?(it)]
{it $=_{obs}$ Avanzar(it$_0$)}
[$\Theta(1)$]
[avanza a la posicion siguiente del iterador.]
[]

~

\subsubsection{Especificación de las operaciones auxiliares utilizadas en la interfaz}

\begin{tad}{\tadNombre{DiccNat($\alpha$) extendido}}
	\tadExtiende{Diccionario(nat,$\alpha$)}
	\textbf{otras operaciones (no exportadas)}

	\tadOperacion{esPermutacion?}{secu(tupla(nat,$\alpha$)), diccNat($\alpha$)}{bool}{}
	\tadOperacion{secuADiccNat}{secu(tupla(nat,$\alpha$))}{diccNat($\alpha$)}{}

	\textbf{axiomas}

	\tadAxioma{esPermutacion?(s,d)}{d = secuADiccNat $\land$ $\#$claves = long(s)}
	\tadAxioma{secuADiccNat(s)}{\IF vacia?(s) THEN vacio ELSE definir(prim(s).clave, prim(s).significado, secuADiccNat(fin(s))) FI }
\end{tad}

\pagebreak

\subsection{Representación de DiccNat($\alpha$)}

\begin{Estructura}{DiccNat}[estr]
	\begin{Tupla}[estr]
		\tupItem{tabla}{vector(lista(tupla(clave : nat, significado : $\alpha$)))}
		\tupItem{\\ listaIterable}{lista(puntero(tupla(clave : nat, significado : $\alpha$)))}
	\end{Tupla}
\end{Estructura}

\begin{enumerate}
	\item No existe dos veces el mismo nat $n$ en dos posiciones distintas del vector y ese nat va a estar en la posicion $n$ mod longitud(tabla)
	\item La suma del largo de todas las listas enlazadas que salen del vector, tiene que ser igual al largo del vector.
	\item Toda tupla de la tabla es apuntado por un elemento de listaIterable.
	\item El largo de listaIterable es igual al largo del vector.
\end{enumerate}

\subsubsection{Invariante de Representación}

\renewcommand{\labelenumi}{(\Roman{enumi})}

\Rep[estr][e]{
	\\
	($\forall$ $l$ : lista(tupla(nat, $\alpha$))) (( $\exists$ $k$ : nat ) ((e.tabla[k] = l)
	  \\ $\Rightarrow$ ($\not \exists$ $q$ : nat) (k $\neq$ q $\land$ e.tabla[q] = l)) $\land$ ($\forall$ $t1$, $t2$ : tupla(nat, $\alpha$)) (esta?($t1$,l) $\land$ esta?($t2$,l)
		\\ $\Rightarrow$ ($\forall$ n,m : nat) ($t1$.clave = n $\land$ $t2$.clave = m
		\\ $\Rightarrow$ n $\neq$ m $\land$ (n mod longitud(e.tabla) = k $\land$ m mod longitud(e.tabla) = k) ) ))
	$\\ \land \\$
	largosDeListas(e.tabla) $=$ longitud(e.tabla)
	$\\ \land \\$
	($\forall$ l : lista(tupla(nat, $\alpha$)))(esta?(l, e.tabla)
		\\ $\Rightarrow$ ($\forall$ t : tupla(nat, $\alpha$))(esta?(t, l)
		\\ $\Rightarrow$ ($\exists$ p : puntero($\alpha$)($\&$t $=$ p $\land$ esta?(p, e.listaIterable)))))
	$\\ \land \\$
	long(e.tabla) $=_{obs}$ long(e.listaIterable)
}\mbox{}

\tadOperacion{largosDeListas}{secu(secu(tupla(nat,$\alpha$)))}{nat}{}

~

\tadAxioma{largoDeListas($vector$)}{
	\IF vacía?($vector$)) THEN
		0
	ELSE
		longitud(prim($vector$)) + largoDeLista(fin($vector$))
	FI
}

\subsubsection{Funci\'on de Abstracci\'on}

\Abs[estr]{Diccionario}[e]{dicc}{
	($\forall$ $l$ : lista(tupla(nat, $\alpha$))) ((esta?($l$,e.tabla) $\\$
	\- \- $\Rightarrow$ ($\forall$ $t$ : tupla(nat, $\alpha$)) esta?($t$,l) $\\$
	\- $\Rightarrow_{L}$ ($\forall$ $n$ : nat) $n =_{obs} t$.clave $\Leftrightarrow$ $\\$ def?($n$, dicc) $\land_L$ obtener($n$, dicc) $=_{obs}$ $t$.significado))
}

\subsection{Representaci\'on del iterador de DiccNat}

\begin{Estructura}{itDiccNat($\alpha$)}[itDiccNat donde itDiccNat es $\newline$ \- \- \- \- itLista(tupla(nat,$\alpha$))]

\end{Estructura}

\subsection{Algoritmos}

\lstset{style=alg}

\begin{lstlisting}[mathescape]
'\alg{icrearDiccionario}{\In{v}{vector(tupla(clave : nat, significado :  $\alpha$))}}{estr}'
{$\textbf{Pre}$ $\equiv$ true}
		nat : i $\leftarrow$ 0 '\ote{1}'
		while i < longitud(v) do '\ote{1}'
			AgregarAtras(res.tabla, vacia()) '\ote{1}'
		end while '\ote{1}'
		i $\leftarrow$ 0 '\ote{1}'
		while i < longitud(v) do '\ote{1}'
			nat : k $\leftarrow$ v[i].clave mod longitud(v) '\ote{1}'
			AgregarAtras(res.tabla[k], v[i]) '\ote{copy($\alpha$)}'
			nat : q $\leftarrow$ longitud(res.tabla[k]) '\ote{longitud(res.tabla[k])}'
			AgregarAtras(res.listaIterable, puntero(res.tabla[k][q-1])) '\ote{q}'
			i++ '\ote{1}'
		end while '\ote{longitud(v) * (copy($\alpha$) + q)}'
'{$\textbf{Post}$ $\equiv$ (($\forall$ $t$ : tupla(nat, $\alpha$)) esta?($t$, v)) $\Rightarrow$ ((definido?(t.clave, res))
 $\land_L$ obtener(t.clave,res) $=_{obs}$ t.significado) $\land$ cantClaves(res) $=_{obs}$ longitud(v)}'
'\ofi{O(longitud(v) * (copy(\alpha) + q))} donde q es la cantidad de elementos que pueden encontrarse en una misma posicion de la tabla'
'$\textbf{Justificacion:}$ las veces que se va a realizar la operacion de copia del elemento $\alpha$ es equivalente al largo del vector, ya que va a hacer una vez cada una. El valor de $q$ va a ser 1 en caso promedio gracias a la funcion de Hash, pero en el peor caso va a ser igual a la longitud del vector de entrada.'
\end{lstlisting}

\begin{lstlisting}[mathescape]
'\alg{iredefinir}{\Inout{d}{estr}, \In{n}{nat}, \In{a}{$\alpha$}}{}'
{$\textbf{Pre}$ $\equiv$ definido?(n,d)}
	nat : k $\leftarrow$ n mod longitud(d) '\ote{1}'
	itLista($\alpha$) : it $\leftarrow$ crearIt(d.tabla[k]) '\ote{1}'
	while haySiguiente?(it) do '\ote{1}'
		IF siguiente(it).clave = n THEN '\ote{1}'
			siguiente(it).significado $\leftarrow$ a '\ote{1}'
		FI
		avanzar(it) '\ote{1}'
	end while '\ote{longitud(d.tabla[k])}'
{$\textbf{Post}$ $\equiv$ obtener(n,d) $=_{obs}$ a}
'\ofi{O(1)} en caso promedio, $O(\#claves(d))$ en peor caso'
'$\textbf{Justificacion:}$ En el peor caso, van a estar todos los elementos en la misma posicion de la tabla y el elemento que queremos redefinir	va a estar en la ultima posicion. Ademas, cada acceso a una posicion de	una lista, es O(i) siendo i la posicion y vamos a recorrer todos los elementos de la tabla (o sea de la lista) hasta encontrar el que queremos. Entonces,	en el peor caso, i va a ser igual al numero de claves del diccionario, pero en caso promedio, i va a ser 1 gracias a la funcion de hash.'
\end{lstlisting}

\begin{lstlisting}[mathescape]
'\alg{iobtener}{\In{n}{nat}, \In{d}{estr}}{puntero($\alpha$)}'
{$\textbf{Post}$ $\equiv$ true}
	nat : k $\leftarrow$ n mod longitud(d) '\ote{1}'
	itLista($\alpha$) : it $\leftarrow$ crearIt(d.tabla[k]) '\ote{1}'
	res $\leftarrow$ NULL '\ote{1}'
	while haySiguiente?(it) do '\ote{1}'
		IF Siguiente(it).clave = n THEN '\ote{1}'
			res $\leftarrow$ Siguiente(it).significado '\ote{1}'
		FI
		Avanzar(it) '\ote{1}'
	end for '\ote{i}'
{$\textbf{Post}$ $\equiv$ res $=_{obs}$ obtener(n,d)}
'\ofi{O(1)} en caso promedio, $O(\#claves(d))$ en peor caso'
'$\textbf{Justificacion:}$ Misma justificacion que para redefinir(d,n,a)'
\end{lstlisting}

\begin{lstlisting}[mathescape]
'\alg{idefinido?}{\In{n}{nat}, \In{d}{estr}}{bool}'
{$\textbf{Pre}$ $\equiv$ true}
	nat : k $\leftarrow$ n mod longitud(d) '\ote{1}'
	itLista($\alpha$) : it $\leftarrow$ crearIt(d.tabla[k]) '\ote{1}'
	res $\leftarrow$ false '\ote{1}'
	while haySiguiente?(it) do '\ote{1}'
		IF Siguiente(it).clave $=$ n THEN '\ote{i}'
			res $\leftarrow$ true '\ote{1}'
		FI
		Avanzar(it) '\ote{1}'
	end while '\ote{i}'
{$\textbf{Post}$ $\equiv$ res $=_{obs}$ def?(n,d)}
'\ofi{O(1)} en caso promedio, $O(\#claves(d))$ en peor caso'
'$\textbf{Justificacion:}$ Misma justificacion que para redefinir(d,n,a)'
\end{lstlisting}

\begin{lstlisting}[mathescape]
'\alg{icantClaves}{\In{d}{estr}}{nat}'
{$\textbf{Pre}$ $\equiv$ true}
	res $\leftarrow$ longitud(d.tabla) '\ote{1}'
{$\textbf{Post}$ $\equiv$ res $=_{obs}$ $\#$claves(d)}
'\ofi{O(1)}'
'$\textbf{Justificacion:}$ Saber la longitud de un vector es O(1) y por el InvRep, sabemos que el largo del vector representa la cantidad de claves del dicc.'
\end{lstlisting}

\begin{lstlisting}[mathescape]
'\alg{iCrearIt}{\In{d}{DiccNat($\alpha$)}}{itDiccNat($\alpha$)}'
{$\textbf{Pre}$ $\equiv$ true}
	res $\leftarrow$ crearIt(d.listaIterable) '\ote{1}'
{$\textbf{Post}$ $\equiv$ alias(esPermutacion(SecuSuby(res), d)) $\land$ vacia?(Anteriores(res))}
'\ofi{\Theta (1)}'
'$\textbf{Justificacion:}$ De acuerdo a la interfaz de Lista, crear un itLista($\alpha$) es $\Theta (1)$.'
\end{lstlisting}

\begin{lstlisting}[mathescape]
'\alg{iHaySiguiente}{\In{it}{itDiccNat($\alpha$)}}{bool}'
{$\textbf{Pre}$ $\equiv$ true}
	res $\leftarrow$ HaySiguiente(it)
{$\textbf{Pre}$ $\equiv$ res $=_{obs}$ haySiguiente?(it)}
'\ofi{\Theta (1)}'
'$\textbf{Justificacion:}$ De acuerdo a la interfaz de Lista, HaySiguiente(it) es $\Theta (1)$.'
\end{lstlisting}

\begin{lstlisting}[mathescape]
'\alg{iSiguiente}{\In{it}{itDiccNat($\alpha$)}}{tupla(nat,$\alpha$)}'
{$\textbf{Pre}$ $\equiv$ HaySiguiente?(it)}
	res $\leftarrow$ Siguiente(it)
{$\textbf{Pre}$ $\equiv$ alias(res $=_{obs}$ Siguiente?(it))}
'\ofi{\Theta (1)}'
'$\textbf{Justificacion:}$ De acuerdo a la interfaz de Lista, Siguiente(it) es $\Theta (1)$.'
\end{lstlisting}

\begin{lstlisting}[mathescape]
'\alg{iSiguienteSignificado}{\In{it}{itDiccNat($\alpha$)}}{$\alpha$}'
{$\textbf{Pre}$ $\equiv$ HaySiguiente?(it)}
	res $\leftarrow$ Siguiente(it)
{$\textbf{Pre}$ $\equiv$ alias(res $=_{obs}$ Siguiente(it).significado)}
'\ofi{\Theta (1)}'
'$\textbf{Justificacion:}$ De acuerdo a la interfaz de Lista, Siguiente(it) es $\Theta (1)$.'
\end{lstlisting}

\begin{lstlisting}[mathescape]
'\alg{iAvanzar}{\Inout{it}{itDiccNat($\alpha$)}}{}'
{$\textbf{Pre}$ $\equiv$ it $=_{obs}$ it$_0$ $\land$ res $=_{obs}$ haySiguiente?(it)}
	res $\leftarrow$ Avanzar(it)
{$\textbf{Pre}$ $\equiv$ it $=_{obs}$ Avanzar?(it$_0$)}
'\ofi{\Theta (1)}'
'$\textbf{Justificacion:}$ De acuerdo a la interfaz de Lista, Avanzar(it) es $\Theta (1)$.'
\end{lstlisting}


\pagebreak
\section{Módulo Diccionario String($\alpha$)}

\subsection{Interfaz}

\textbf{se explica con}: \tadNombre{Diccionario(String,$\alpha$)}.

\textbf{géneros}: \TipoVariable{diccString($(\alpha)$)}.

Se representa mediante un árbol n-ario con invariante de trie

~

\subsubsection{Operaciones básicas de Diccionario String($\alpha$)}

\InterfazFuncion{CrearDiccionario}{\In{c}{campus}, \In{d}{diccString$(\alpha)$}}{true}
{res $=_{obs}$ vacío()}
[$O(1)$]
[Crea un diccionario vacío.]
[]

~



\InterfazFuncion{Definido?}{\In{d}{diccString$(\alpha)$}, \In{c}{string})}{bool}
[true]
{$res$ $\igobs$ def?($d$, $c$)}
[$O(L)$]
[Devuelve true si la clave está definida en el diccionario y false en caso contrario.]
[]

~

\InterfazFuncion{Definir}{\In{d}{diccString$(\alpha)$}, \In{c}{string}, \In{s}{$\alpha$}}{}
[$ d \igobs d_0 $]
{$ d \igobs$ definir($c$, $s$, $d_0$)}
[$O(L)$ ]
[Define la clave $c$ con el significado $s$]
[Almacena una copia de $s$.]

~

\InterfazFuncion{Obtener}{\In{d}{diccString$(\alpha)$}, \In{c}{string}}{$\alpha$}
[def?($c$, $d$)]
{alias($res$ $\igobs$ obtener($c$, $d$))}
[$O(L)$]
[Devuelve el significado correspondiente a la clave $c$.]
[Devuelve el significado almacenado en el diccionario, por lo que $res$ es modificable si y sólo si $d$ lo es.]

~

\InterfazFuncion{• = •}{\Inout{d}{diccString($\alpha$)}, \Inout{d'}{diccString($\alpha$)}}{bool}
[true]
{$res$ $\igobs$ (d $\igobs$ d')}
[$O(L*n*(\alpha  \igobs \alpha'))$]
[Indica si d es igual d']
[]

~

\InterfazFuncion{Copiar}{\In{dicc}{diccString$(\alpha)$}}{diccString$(\alpha)$}
[true]
{$res$ $\igobs$ dicc}
[$O(n * L * copy(\alpha))$]
[Devuelve una copia del diccionario]
[]


~

\pagebreak

\subsubsection{Representación de Diccionario String($\alpha$)}

\begin{Estructura}{ Diccionario String($\alpha$)}[estr]
	\begin{Tupla}[estr]
		\tupItem{raiz}{arreglo(puntero(Nodo))}%
		\tupItem{listaIterable}{lista(puntero(Nodo))}%
	\end{Tupla}

	~

	\begin{Tupla}[Nodo]
		\tupItem{arbolTrie}{arreglo(puntero(Nodo))}%
		\tupItem{\\
					info}{$\alpha$}%
		\tupItem{\\
					infoValida}{bool}%
		\tupItem{\\
					infoEnLista}{iterador(listaIterable)}
	\end{Tupla}

\end{Estructura}

\subsubsection{Invariante de Representación}

\renewcommand{\labelenumi}{(\Roman{enumi})}

\begin{enumerate}
	\item Raiz es la raiz del arbol con invariante de trie y es un arreglo de 27 posiciones.
	\item Cada uno de los elementos de la lista tiene que ser un puntero a un Nodo del trie.
	\item Nodo es una tupla que contiene un arreglo de 27 posiciones con un puntero a otro Nodo en cada posicion
	,un elemento info que es el alfa que contiene esa clave del arbol, y un elemento iterador que es un puntero
	a un nodo de la lista enlazada.
	\item El iterador a la lista enlazada de cada nodo tiene que apuntar al elemento de la lista que apunta al 
	mismo Nodo.

\end{enumerate}

\pagebreak

\Rep[estr][e]
	\\
	($\forall c$: diccString($(\alpha)$))()

\Rep[estr][e]{
	\\
	tamaño(raiz)==27 $\land$
	\\
	($\forall i$ $\in$ [0..tamaño(raiz)($c$ $\in$ computadoras($e$.topologia) $\Leftrightarrow$ \\
	\- ( \\
	\- \- ($\exists cd$: compuDCNet)
	(está?($cd$, $e$.vectorCompusDCNet) $\land$ ($cd$.pc = puntero($c$)) $\land$ \\
	\- \- ($\exists s$: string)(def?($s$, $e$.diccCompusDCNet) $\land$ ($s$ = $c$.ip))) \\
	\- ) \\
	) $\yluego$\\
	($\forall cd$: compuDCNet)(está?($cd$, $e$.vectorCompusDCNet) $\Leftrightarrow$ \\
	\- ($\exists s$: string)(($s$ = $cd$.pc$\rightarrow$ip) $\land$
		def?($s$, $e$.diccCompusDCNet) $\yluego$ \\
	\- obtener($s$, $e$.diccCompusDCNet) = puntero($cd$)) \\
	) $\yluego$\\
	($\exists cd$: compuDCNet)(está?($cd$, $e$.vectorCompusDCNet) $\yluego$ \\
	*($cd$.pc) = compuQueMásEnvió($e$.vectorCompusDCNet) $\land$ $e$.laQueMásEnvió = puntero($cd$)) $\yluego$\\
	($\forall cd_1$: compuDCNet)(está?($cd_1$, $e$.vectorCompusDCNet) $\implies$ \\
	\- ($\forall p_1$: paquete)($p_1$ $\in$ $cd_1$.conjPaquetes $\implies$ \\
	\- \- ($\forall cd_2$: compuDCNet)((está?($cd_2$, $e$.vectorCompusDCNet)
		$\land$ $cd_1$ $\neq$ $cd_2$) $\implies$ \\
	\- \- \- ($\forall p_2$: paquete)($p_2$ $\in$ $cd_2$.conjPaquetes $\implies$
		$p_1$.id $\neq$ $p_2$.id) \\
	\- \- ) \\
	\- ) \\
	) $\yluego$ \\
	($\forall cd$: compuDCNet)(está?($cd$, $e$.vectorCompusDCNet) $\implies$ \\
	\- ( \\
	\- \- ($\forall p$: paquete)($p$ $\in$ $cd$.conjPaquetes $\Leftrightarrow$ \\
	\- \- \- ( \\
	\- \- \- \- (($p$.origen $\in$ computadoras($e$.topologia) $\land$ $p$.destino
		$\in$ computadoras($e$.topologia) $\land$ \\
	\- \- \- \- $p$.destino $\neq$ *($cd$.pc)) $\yluego$ \\
	\- \- \- \- ($\exists sc$: secu(compu))($sc$ $\in$
		caminosMinimos($e$.topologia, $p$.origen, $p$.destino) $\land$
		está(*($cd$.pc), $sc$))) $\land$ \\
	\- \- \- \- ($\exists n$: nat)
		((def?($n$, $cd$.diccPaquetesDCNet) $\land$ $p$.id = $n$) $\yluego$ \\
	\- \- \- \- \- ($\exists pdn$: paqueteDCNet)($pdn$ $\in$ $e$.conjPaquetesDCNet $\land$ Siguiente($pdn$.it) = $p$ $\land$ \\
	\- \- \- \- \- \- (($p$.origen = *($cd$.pc) $\land$ $pdn$.recorrido = *($cd$.pc) $\puntito$ <>) $\lor$ \\
	\- \- \- \- \- \- ($p$.origen $\neq$ *($cd$.pc) $\land$
		$pdn$.recorrido $\in$ caminosMinimos($e$.topologia, $p$.origen,
		*($cd$.pc)))) $\land$ \\
	\- \- \- \- \- \- Siguiente(obtener($n$, $cd$.diccPaquetesDCNet)) = $pdn$ \\
	\- \- \- \- \- ) \\
	\- \- \- \- ) \\
	\- \- \- ) \\
	\- \- ) $\yluego$ \\
	\- \- ($\neg$vacía?($cd$.colaPaquetesDCNet) $\Leftrightarrow$ \\
	\- \- \- ($\exists p$: paquete)(($p$ $\in$ $cd$.conjPaquetes) $\land$
		($p$ = paqueteMásPrioridad($cd$.conjPaquetes)) $\land$ \\
	\- \- \- \- ($\exists pdn$: paqueteDCNet)(($pdn$ $\in$
		$e$.conjPaquetesDCNet) $\land$ (Siguiente($pdn$.it) = $p$) $\land$ \\
	\- \- \- \-	(Siguiente(proximo($cd$.colaPaquetesDCNet)) = $pdn$)) \\
	\- \- \- ) \\
	\- \- ) $\yluego$ \\
	\- \- ($cd$.enviados $\geq$ enviadosCompu(*($cd$.pc), $e$.vectorCompusDCNet)) $\land$\\
	\- \- ($\neg$HaySiguiente?(cd.paqueteAEnviar))
	\- ) \\
	)
}\mbox{}

\tadOperacion{compuQueMásEnvió}{secu(compuDCNet)/scd}{compu}{$\neg$vacía?($scd$)}
\tadOperacion{maxEnviado}{secu(compuDCNet)/scd}{nat}{$\neg$vacía?($scd$)}
\tadOperacion{enviaronK}{secu(compuDCNet),nat}{conj(compu)}{}
\tadOperacion{paqueteMásPrioridad}{conj(paquete)/cp}{paquete}{$\neg \emptyset?(cp)$}
\tadOperacion{paquetesConPrioridadK}{conj(paquete),nat}{conj(paquete)}{}
\tadOperacion{altaPrioridad}{conj(paquetes)/cp}{nat}{$\neg \emptyset?(cp)$}
\tadOperacion{enviadosCompu}{compu,secu(compuDCNet)}{nat}{}
\tadOperacion{aparicionesCompu}{compu,conj(nat)/cn,dicc(nat,itConj(paqueteDCNet))/dp}{nat}{claves($dp$) $\subseteq$ $cn$}

~

\tadAxioma{compuQueMásEnvió($scd$)}{dameUno(enviaronK($scd$, maxEnviado($scd$)))}
\tadAxioma{maxEnviado($scd$)}{
	\IF vacía?(fin($scd$)) THEN
		prim($scd$).enviados
	ELSE
		max(prim($scd$), maxEnviado(fin($scd$)))
	FI
}
\tadAxioma{enviaronK($scd$, $k$)}{
	\IF vacía?($scd$) THEN
		$\emptyset$
	ELSE {
		\IF prim($scd$).enviados = $k$ THEN
			Ag(*(prim($scd$).pc), enviaronK(fin($scd$), $k$))
		ELSE
			enviaronK(fin($scd$), $k$)
		FI
		}
	FI
}
\tadAxioma{paqueteMásPrioridad($dcn$, $cp$)}{dameUno(paquetesConPrioridadK($cp$, altaPrioridad($cp$)))}

\tadAxioma{altaPrioridad($cp$)}{
	\IF $\emptyset$?(sinUno($cp$)) THEN
		dameUno($cp$).prioridad
	ELSE
		min(dameUno($cp$).prioridad, altaPrioridad(sinUno($cp$)))
	FI
}

\tadAxioma{paquetesConPrioridadK($cp$, $k$)}{
	\IF $\emptyset$?($cp$) THEN
		$\emptyset$
	ELSE {
		\IF dameUno($cp$).prioridad = $k$ THEN
			 Ag(dameUno($cp$), paquetesConPrioridadK(sinUno($cp$), $k$))
		ELSE
			paquetesConPrioridadK(sinUno($cp$), $k$)
		FI
		}
	FI
}

\tadAxioma{enviadosCompu($c$, $scd$)}{
	\IF vacía?($scd$) THEN
		0
	ELSE {
			\IF prim($scd$) = $c$ THEN
				enviadosCompu($c$, fin($scd$))
			ELSE {
				aparicionesCompu($c$, claves(prim($scd$).diccPaquetesDCNet), \\
				prim($scd$).diccPaquetesDCNet) + enviadosCompu($c$, fin($scd$))
				}
			FI
		}
	FI
}

\tadAxioma{aparicionesCompu($c$, $cn$, $dpd$)}{
	\IF	$\emptyset$?($cn$) THEN
		0
	ELSE {
			\IF está?($c$, Siguiente(obtener(dameUno($cn$), $dpd$)).recorrido) THEN
				1 + aparicionesCompu($c$, sinUno($cn$), $dpd$)
			ELSE {
				aparicionesCompu($c$, sinUno($cn$), $dpd$)
				}
			FI
		}
	FI
}

\pagebreak

\subsubsection{Funci\'on de Abstracci\'on}

\Abs[estr]{dcnet}[e]{dcn}{
red($dcn$) = $e$.topología $\land$ \\
($\forall cdn$: compuDCNet)(está?($cdn$, $e$.vectorCompusDCNet) $\impluego$ \\
\- enEspera($dcn$, *($cdn$.pc)) = $cdn$.conjPaquetes $\land$ \\
\- cantidadEnviados($dcn$, *($cdn$.pc)) = $cdn$.enviados $\land$ \\
\- ($\forall p$: paquete)($p$ $\in$ $cdn$.conjPaquetes $\impluego$ \\
\- \- caminoRecorrido($dcn$, $p$) = Siguiente(obtener($p$.id,
	$cdn$.diccPaquetesDCNet)).recorrido \\
\- ) \\
)
}

\subsection{Algoritmos}

\lstset{style=alg}

\begin{lstlisting}[mathescape]
'\alg{crearDiccionario}{}{estr}'
	
    arreglo(puntero(Nodo)): res.raiz $\leftarrow$ CrearArreglo(27) '\ote{1}'
	nat: i $\leftarrow$ 0 '\ote{1}' 
    while i < long(res.raiz) do '\ote{1}'
    	res.raiz[i] $\leftarrow$ null '\ote{1}'
    end while '\ote{1}'
    res.listaIterable $\leftarrow$ Vacia() '\ote{1}'

'\ofi{O(1)}'

'\alg{definido?}{\In{d}{diccString($\alpha$)}, \In{c}{string}}{bool}'

	nat: i $\leftarrow$ 0 '\ote{1}'
	nat: letra $\leftarrow$ posicion(c[0]) '\ote{1}'
	puntero(Nodo): arr $\leftarrow$ d.raiz[letra] '\ote{1}'
	while(i <= longitud(c) and not arr = null) do '\ote{1}'
		i $\leftarrow$ i + 1 '\ote{1}'
		letra $\leftarrow$ posicion(c[i]) '\ote{1}'
		arr $\leftarrow$ (*arr).arbolTrie[letra] '\ote{1}'
	end while '\ote{$|n_m|$}'
	if(i=longitud(c)) then '\ote{1}'
		res $\leftarrow$ (*arr).infoValida '\ote{1}'
	else
		res $\leftarrow$ false '\ote{1}'
	end if
	
'\ofi{O(|n_m|)}'

\end{lstlisting}

\begin{lstlisting}[mathescape]
'\alg{definir}{\Inout{d}{diccString($\alpha$)}, \In{c}{string}, \In{s}{$\alpha$}}{}'

	nat: i $\leftarrow$ 0 '\ote{1}'
	nat: letra $\leftarrow$ posicion(c[0]) '\ote{1}'
	if (d.raiz[letra] = null) then '\ote{1}'
		Nodo: nuevo '\ote{1}'
		arreglo(puntero(Nodo)): nuevo.arbolTrie $\leftarrow$ CrearArrelgo(27) '\ote{1}'
		nuevo.infoValida $\leftarrow$ false '\ote{1}'
		d.raiz[letra] $\leftarrow$ puntero(nuevo) '\ote{1}'
	end if
	puntero(Nodo): arr $\leftarrow$ d.raiz[letra] '\ote{1}'
	while(i <= longitud(c)) do '\ote{1}'
		i $\leftarrow$ i + 1 '\ote{1}'
		letra $\leftarrow$ posicion(c[i]) '\ote{1}'
		if (arr.arbolTrie[letra] = null) then '\ote{1}'
			Nodo: nuevoHijo '\ote{1}'
			arreglo(puntero(Nodo)): nuevoHijo.arbolTrie $\leftarrow$ CrearArrelgo(27) '\ote{1}'
			nuevoHijo.infoValida $\leftarrow$ false '\ote{1}'
			arr.arbolTrie[letra] $\leftarrow$ puntero(nuevoHijo) '\ote{1}'	
		end if			
		arr $\leftarrow$ (*arr).arreglo[letra] '\ote{1}' '\ote{1}'
	end while '\ote{$|n_m|$}'
    (*arr).info $\leftarrow$ s '\ote{copy(s)}'
    if(not (*arr).infoValida = true) then '\ote{1}'
    	itLista(puntero(Nodo)) it $\leftarrow$ AgregarAdelante(d.listaIterable,null) '\ote{1}'
    	(*arr).infoValida $\leftarrow$ true '\ote{1}'
    	(*arr).infoEnLisa $\leftarrow$ it '\ote{1}'
    	siguiente(it) $\leftarrow$ puntero(*arr) '\ote{1}'
    end if
    	
'\ofi{O(|n_m| + copy(s))}'
\end{lstlisting}

\begin{lstlisting}[mathescape]
'\alg{obtener}{\In{d}{diccString($\alpha$)}, \In{c}{string}}{$\alpha$}'

	nat: i $\leftarrow$ 0 '\ote{1}'
	nat: letra $\leftarrow$ posicion(c[0]) '\ote{1}'
	puntero(Nodo): arr $\leftarrow$ d.raiz[letra] '\ote{1}'
	while(i <= longitud(c)) do '\ote{1}'
		i $\leftarrow$ i + 1 '\ote{1}'
		letra $\leftarrow$ posicion(c[i]) '\ote{1}'
		arr $\leftarrow$ (*arr).arbolTrie[letra] '\ote{1}'
	end while '\ote{$|n_m|$}'
	res $\leftarrow$ (*arr).info '\ote{1}'
	
'\ofi{O(|n_m|)}'
\end{lstlisting}

\begin{lstlisting}[mathescape]
'\alg{CaminoRecorrido}{\In{dcn}{dcnet}, \In{p}{paquete}}{lista(compu)}'

	nat: i $\leftarrow$ 0 '\ote{1}'
	while i < Longitud(dcn.vectorCompusDCNet) do '\ote{1}'
		if Definido?(dcn.vectorCompusDCNet[i].diccPaquetesDCNet, p.id) then '\ote{log(k)}'
			res $\leftarrow$ Siguiente(Obtener(dcn.vectorCompusDCNet[i].diccPaquetesDCNet,
				p.id)).recorrido '\ote{log(k)}'
		end if
		i++ '\ote{1}'
	end while '\ote{n * log(k)}'

'\ofi{O(n * log(k))}'
\end{lstlisting}

\begin{lstlisting}[mathescape]
'\alg{CantidadEnviados}{\In{dcn}{dcnet}, \In{c}{compu}}{nat}'

	res $\leftarrow$ Obtener(dcn.diccCompusDCNet, c.ip)$\rightarrow$enviados '\ote{L}'

'\ofi{O(L)}'
\end{lstlisting}

\begin{lstlisting}[mathescape]
'\alg{EnEspera}{\In{dcn}{dcnet}, \In{c}{compu}}{nat}'

	res $\leftarrow$ Obtener(dcn.diccCompusDCNet, c.ip)$\rightarrow$conjPaquetes '\ote{L}'

'\ofi{O(L)}'
\end{lstlisting}

\begin{lstlisting}[mathescape]
'\alg{PaqueteEnTransito}{\In{dcn}{dcnet}, \In{p}{paquete}}{bool}'

	res $\leftarrow$ false
	nat: i $\leftarrow$ 0 '\ote{1}'
	while i < Longitud(dcn.vectorCompusDCNet) do '\ote{1}'
		if Definido?(dcn.vectorCompusDCNet[i].diccPaquetesDCNet, p.id) then '\ote{log(k)}'
			res $\leftarrow$ true '\ote{1}'
		end if
		i++ '\ote{1}'
	end while '\ote{n * log(k)}'

'\ofi{O(n * log(k))}'
\end{lstlisting}

\begin{lstlisting}[mathescape]
'\alg{LaQueMasEnvio}{\In{dcn}{dcnet}}{compu}'

	res $\leftarrow$ *(dcn.laQueMasEnvio$\rightarrow$pc) '\ote{1}'

'\ofi{O(1)}'
\end{lstlisting}

\begin{lstlisting}[mathescape]
'\alg{$\puntito$ $=_i$ $\puntito$}{\In{dcn_1}{dcnet}, \In{dcn_2}{dcnet}}{bool}'

	bool: boolTopo $\leftarrow$ $dcn_1$.topologia = $dcn_2$.topologia '\ote{n + $L^2$}'
	bool: boolVec $\leftarrow$ $dcn_1$.vectorCompusDCNet = $dcn_2$.vectorCompusDCNet '\ote{n * k * (k + n)}'
	bool: boolConj $\leftarrow$ $dcn_1$.conjPaquetesDCNet = $dcn_2$.conjPaquetesDCNet '\ote{$k^3$ * (k + n)}'
	bool: boolMasEnvio $\leftarrow$ *($dcn_1$.laQueMasEnvio) = *($dcn_2$.laQueMasEnvio) '\ote{1}'

	res $\leftarrow$ boolTopo $\land$ boolVec $\land$ boolTrie $\land$ boolConj $\land$ boolMasEnvio '\ote{1}'

'\ofi{O(n * k^3 * (k + n))}'
\end{lstlisting}

\begin{lstlisting}[mathescape]
'\alg{$\puntito$ $=_{compudcn}$ $\puntito$}{\In{c_1}{compuDCNet}, \In{c_2}{compuDCNet}}{bool}'

	bool: boolPC $\leftarrow$ *($c_1$.pc) = *($c_2$.pc) '\ote{1}'
	bool: boolConj $\leftarrow$ $c_1$.conjPaquetes = $c_1$.conjPaquetes '\ote{$k^2$}'
	bool: boolAVL $\leftarrow$ true '\ote{1}'
	bool: boolCola $\leftarrow$ true '\ote{1}'
	bool: boolPaq $\leftarrow$ Siguiente($c_1$.paqueteAEnviar) $=_{paqdcn}$ Siguiente($c_2$.paqueteAEnviar)
		'\ote{n}'
	bool: boolEnviados $\leftarrow$ $c_1$.enviados = $c_2$.enviados '\ote{1}'

	if boolConj then '\ote{1}'
		itConj: $itconj_1$ $\leftarrow$ CrearIt($c_1$.conjPaquetes) '\ote{1}'
		while HaySiguiente?($itconj_1$) do '\ote{1}'
			if Definido?($c_2$.diccPaquetesDCNet, Siguiente($itconj_1$)).id then '\ote{log(n)}'
				if $\neg$(Siguiente(Obtener($c_1$.diccPaquetesDCNet, Siguiente($itconj_1$).id))
					$=_{paqdcn}$
					Siguiente(Obtener($c_1$.diccPaquetesDCNet, Siguiente($itconj_1$).id)))
					then '\ote{n}'
					boolAVL $\leftarrow$ false '\ote{1}'
				end if
			else
				boolAVL $\leftarrow$ false '\ote{1}'
			end if
			Avanzar($itconj_1$) '\ote{1}'
		end while '\ote{n * k}'
	end if

	if EsVacia($c_1$.colaPrioridad) then '\ote{1}'
		if $\neg$EsVacia($c_2$.colaPrioridad) then '\ote{1}'
			boolCola $\leftarrow$ false '\ote{1}'
		end if
	else
		if EsVacia($c_1$.colaPrioridad) then '\ote{1}'
			boolCola $\leftarrow$ false '\ote{1}'
		else
			if $\neg$(Siguiente(Proximo($c_1$.colaPrioridad)) $=_{paqdcn}$
				Siguiente(Proximo($c_2$.colaPrioridad))) then '\ote{n}'
				boolCola $\leftarrow$ false '\ote{1}'
			end if
		end if
	end if

	res $\leftarrow$ boolPC $\land$ boolConj $\land$ boolAVL $\land$ boolCola $\land$ boolPaq $\land$ boolEnviados '\ote{1}'

'\ofi{O(k^2 + n * k) = O(k * (k + n))}'
\end{lstlisting}

\begin{lstlisting}[mathescape]
'\alg{$\puntito$ $=_{paqdcn}$ $\puntito$}{\In{p_1}{paqueteDCNet}, \In{p_2}{paqueteDCNet},}{bool}'

	bool: boolPaq $\leftarrow$ Siguiente($p_1$.it) = Siguiente($p_2$.it) '\ote{1}'
	bool: boolRecorrido $\leftarrow$ $p_1$.recorrido = $p_2$.recorrido '\ote{n}'

	res $\leftarrow$ boolPaq $\land$ boolRecorrido '\ote{1}'

'\ofi{O(n)}'
\end{lstlisting}


\end{document}
